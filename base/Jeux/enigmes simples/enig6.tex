Un homme va chez un ami qui a trois enfants, il lui demande l'âge de ceux-ci. L'ami répond: "Le produit des âges de mes enfants est égal à 72 et leur somme est égale au numéro de la maison d'en face". L'homme réfléchit et dit:\\

- "Tu as du oublier une donnée !"\\
- "Ah oui, mon aîné joue au football"\\
- "D'accord j'ai trouvé, c'est facile."\\

Qu'en pensez-vous ?\\

\textbf{Réponse :}\\

C'est un grand classique, il faut écrire toutes les combinaisons de nombres 
dont le produit est 72 !

Comme l'homme connaît le "numéro de la maison d'en face" il peut logiquement 
en déduire la combinaison cherchée. Mais il lui manque une donnée, cela 
signifie qu'il y a plusieurs solutions possibles.

Or les seules combinaisons qui ont une somme égales sont (2,6,6) et (3,3,8). 
(le numéro de la maison d'en face doit être 14)

Comme il y a un aîné les âges cherchés sont: 3 ans 3 ans et 8 ans.
