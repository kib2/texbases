%@Titre:Parallélogramme de Wittenbauer.
%@Auteur:d'après IREM Lille.\par
On considère un quadrilatère {\em non croisé} $ABCD$.\\ On appelle $I$
et $P$ les points du segment $[BC]$ tels que $BI=IP=PC$. On appelle $Q$
et $K$ les points du segment $[CD]$ tels que $CQ=QK=KD$. On appelle $L$
et $S$ les points du segment $[DA]$ tels que $DL=LS=SA$. On appelle $R$
et $J$ les points du segment $[AB]$ tels que $AR=RJ=JB$.
\\On appelle $E$ le point d'intersection des droites $(RS)$ et
$(IJ)$. On appelle $F$ le point d'intersection des droites $(IJ)$ et
$(PQ)$. On appelle $E$ le point d'intersection des droites $(PQ)$ et
$(KL)$. On appelle $H$ le point d'intersection des droites $(KL)$ et
$(RS)$.
\begin{myenumerate}
  \item Après avoir fait plusieurs figures manuellement (et d'autres
    éventuellement à l'aide d'un logiciel de géométrie dynamique --
    {\em on fournira des impressions de ces constructions.}), quelle
    conjecture peut-on faire sur la nature du quadrilatère $EFGH$ ?
  \item Démontre cette conjecture.
\end{myenumerate}
