%@Titre:Asie -- 2007
\paragraph{Première partie}\hfill\newline
\begin{myenumerate}
\item On considère le \emph{tableau de proportionnalité} ci-dessous.\\
\begin{center}
\begin{pspicture}(6,2)
\psframe(4,2)
\psline(2,0)(2,2) \psline(0,1)(4,1)
\rput(1,0.5){$70$} \rput(1,1.5){$20$}
\rput(3,0.5){$b$}
\rput(3,1.5){$30$}
\rput(4,0.5){\pnode{C}} \rput(4,1.5){\pnode{D}} 
\nccurve{->}{D}{C}
\rput(4.6,1){$\times a$}
\end{pspicture}
\end{center}
\begin{enumerate}
\item Calculer $b$.
\item On appelle $a$ le coefficient de proportionnalité. Calculer $a$.
\end{enumerate}
\item On considère la fonction linéaire $f$ définie par : $f~: x \longmapsto3,5x$.\\
Sur la feuille de papier millimétré, tracer la droite $d$ représentant la fonction $f$.\\
\emph{On prendra un repère orthonormé ; l'origine sera placée en bas et à gauche de la feuille ;sur chaque axe : $1$ cm représentera $10$ unités.}
\end{myenumerate}
\paragraph{Deuxième partie}\hfill\newline
\begin{myenumerate}
\item Dans le repère précédent, placer les points $A(20 ; 70)$ et $B(60 ; 90)$.
\item Déterminer la fonction affine $g$ dont la représentation graphique est la droite $(AB)$.
\item \begin{enumerate}
\item Résoudre le système $\left\{ \begin{array}{l c l}
y & = & 3,5x\\
y &= &	0,5 x + 60\\
\end{array}\right.$.
\item Que représente le couple $(x;y)$, solution de ce système, pour les droites $d$ et $(AB)$ ?
\end{enumerate}
\end{myenumerate}
\paragraph{Troisième partie}\hfill\newline
On dispose d'un ressort de $60$~mm. Quand on lui suspend une masse de 20~g, il s'allonge de 10~mm.
\begin{myenumerate}
\item On admet que l'allongement du ressort est toujours proportionnel à la masse accrochée.\\
Démontrer que la longueur totale du ressort pour une masse de 80~g est 100~mm.
\item Soit $x$ la masse suspendue en grammes.\\
Exprimer l'allongement du ressort en fonction de $x$.
\item Exprimer la longueur totale du ressort en fonction de $x$.
\item Sachant que la masse volumique de l'or est $19,5$~g/cm$^3$, calculer la masse d'un cube en or de 2~cm d'arête.
\item On suspend ce cube à ce ressort.\\
Déterminer la longueur totale du ressort.
Retrouver cette longueur sur le graphique. Faire apparaître les pointillés nécessaires.
\end{myenumerate}