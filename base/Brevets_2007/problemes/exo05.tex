%@Titre:Polynésie -- 2007
Teva roule en scooter et tout à coup, il aperçoit un piéton.\\
La distance de réaction est la distance parcourue entre le temps où Teva voit l'obstacle et le moment où il va ralentir ou freiner. \\
Teva est en bonne santé, il lui faut 1 seconde en moyenne pour réagir.
\paragraph{Première partie}\hfill\newline
\begin{myenumerate}
\item Si Teva roule à 54~km/h.
\begin{enumerate}
\item Quelle distance en mètre parcourt-il en une heure ? 
\item Quelle distance en mètre parcourt-il en 1 seconde ?\\
En déduire la distance de réaction de Teva, s'iI roule à 54~km/h.
\end{enumerate}
\item On admettra que la distance de réaction se calcule avec la formule suivante :\\
$D_R=V\times\dfrac{5}{18}$, où $D_R$ est la distance de réaction en mètre  et $V$ est la vitesse en km/h.\\
Reproduire et compléter le tableau suivant :
\par
\begin{tabularx}{\linewidth}{|l|*{4}{>{\centering \arraybackslash}X|}}\hline
Vitesse en km/h&	45& 	54&90&108\\ \hline
Distance de réaction en mètre&&&&  \\ \hline
\end{tabularx}
\end{myenumerate}
\paragraph{Deuxième partie}\hfill\newline
On appelle $x$ la vitesse à laquelle peut rouler un conducteur.
\begin{myenumerate}
\item  Exprimer en fonction de $x$, la distance de réaction $d(x)$.
\item \begin{enumerate}
\item  Sur la feuille de papier millimétré, placer l'origine $O$ en bas et à gauche.\\
Prendre pour unités :
\begin{itemize}
\item  en abscisse, 1~cm pour 10~km/h ;
\item  en ordonnée, 1~cm pour 2~m.
\end{itemize}
\item Dans le repère précédent, tracer la représentation graphique de la fonction $d$
définie par $d(x)=\dfrac{5}{18}x$. (on pourra utiliser le tableau de la première partie).
\end{enumerate}
\item Un conducteur roule à la vitesse de 30~km/h.
\begin{enumerate}
\item Déterminer graphiquement la distance de réaction de ce conducteur.\\
(On laissera apparents les traits de construction )
\item Retrouver le résultat de la question précédente par le calcul. Le présenter sous forme de fraction irréductible, puis arrondir à l'unité.
 \end{enumerate}
\item En utilisant le graphique (on laissera les traits apparents), donner la vitesse à partir de laquelle la distance de réaction est supérieure à 20~m.
\end{myenumerate}