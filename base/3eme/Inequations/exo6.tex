$ABC$ est un triangle tel que $AC=20$~cm; $BC=16$~cm;
$AB=12$~cm. $F$ est un point du segment $[BC]$. La perpendiculaire à
la droite $(BC)$ passant par $F$ coupe le segment $[CA]$ en $E$.
\begin{myenumerate}
\item
\begin{enumerate}
\item Démontre que le triangle $ABC$ est rectangle en $B$ et calcule
son aire.
\item Démontre que la droite $(EF)$ est parallèle à la droite $(AB)$.
\end{enumerate}
\item On se place dans le cas où $CF=12$~cm.
\par Démontre que $EF=9$~cm puis calcule l'aire du triangle $EBC$.
\item On se place dans le cas où $F$ est un point quelconque du
segment $[BC]$, distinct de $B$ et $C$. Dans cette question, on pose
$CF=x$ ($x$ étant un nombre tel que $0<x<16$).
\begin{enumerate}
\item Montre que la longueur $EF$, exprimée en cm, est égale à $0,75x$.
\item Montre que l'aire du triangle $EBC$, exprimée en cm$^2$, est
égale à $6x$.
\item Pour quelle valeur de $x$, l'aire du triangle $EBC$, exprimée en
cm$^2$, est-elle égale à 30 ?
\end{enumerate}
\end{myenumerate}