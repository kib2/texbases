%@metapost: 301dm14.mp
\paragraph{Partie A} On considère la fonction $f$ qui à $x$ fait
correspondre le nombre $40-4x$.
\begin{myenumerate}
\item Quelle est l'expression de $f(x)$ ?
\item Quelle est l'image du nombre $0$ par la fonction $f$ ?
\item Quel nombre a pour image 16 par la fonction $f$ ?
\item Quel nombre a pour image 10 par la fonction $f$ ?
\end{myenumerate}
\par
\paragraph{Partie B}\subitem{}\par
\compo{1}{301dm14}{0.8}{Les dimensions de ce pavé droit sont :
$EH=8$~cm, $DH=10$~cm, $GH=12$~cm.
\par{\em La figure ci-dessus n'est pas en vraie grandeur.}
\par $I$ est un point du segment $[DH]$. La pyramide $\cal P$ de
sommet $D$ est de base $EFGH$ est coupée par un plan parallèle à la
base passant par le point $I$. La section est un quadrilatère $IJKL$, $J$, $K$ et $L$ appartenant respectivement aux segments $[DE]$, $[DF]$ et $[DG]$.
}
\begin{myenumerate}
\item
\begin{enumerate}
\item Précise la nature du quadrilatère $EFGH$ et calcule son aire.
\item Détermine alors le volume de la pyramide $\cal P$.
\end{enumerate}
\item Quelle est la nature du quadrilatère $IJKL$ ?
\item Représente la section $IJKL$ en perspective cavalière sur le
dessin ci-dessus.
\item Le plan de section étant parallèle à la base, les droites $(IJ)$
et $(EH)$ sont parallèles, ainsi que les droites ($IL)$ et $(GH)$.
 Dans cette question, on pose $IH=4$~cm.
\begin{enumerate}
\item Calcule la longueur $DI$.
\item Montre que $IJ=4,8$~cm et $IL=7,2$~cm.
\item Calcule le périmètre $p$ du quadrilatère $IJKL$.
\end{enumerate}
\item Dans cette question, on considère maintenant que $IH=x$ (en
cm).
\begin{enumerate}
\item Exprime la longueur $DI$ en fonction de $x$.
\item Montre que $IJ=8-\dfrac45x$ et que $IL=12-\dfrac65x$
\item Exprime le périmètre $p$ du quadrilatère $IJKL$ en fonction de
$x$.
\item Où placer le point $I$ pour que le périmètre $p$ du quadrilatère
$IJKL$ soit égal à 10~cm ?
\end{enumerate}
\end{myenumerate}