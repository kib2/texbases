%@metapost: 303dm07.mp
\par\compo{1}{303dm07}{0.75}{Un cornet de glace en forme de cône est
constitué de deux parties :
\begin{itemize}
\item une partie inférieure composée de gaufre et remplie de crème
glacée,
\item une partie supérieure constituée de glace.
\end{itemize}
\par On donne $SO=16$~cm; $AB=5$~cm.
}
\paragraph{Première partie} On arrondira tous les résultats au dixième
prés.
\begin{myenumerate}
\item Calcule la longueur $SA$.
\item Détermine une valeur approchée au degré près de l'angle $\widehat{OSA}$.
\item Calcule le volume du cornet de glace.
\item On appelle $SA'B'$ le cône constitué de gaufre dont la base de
centre $O'$ est parallèle à la base du cône $SAB$. On donne
$SO'=12$~cm. Le cône $SA'B'$ est une réduction du cône $SAB$.
\begin{enumerate}
\item Calcule le coefficient de réduction.
\item Calcule la longueur $A'B'$.
\item Calcule l'aire du disque de diamètre $[A'B']$.
\item Calcule le volume de la partie supérieure en forme de tronc de
cône constituée uniquement de glace.
\end{enumerate}
\end{myenumerate}
\paragraph{Deuxième partie}
Un vendeur de glace propose à ses clients les cornets de glace décrits
ci-dessus. Il les achète 0,50~\textgreek{\euro} l'unité au fabricant Moki. Il en
achète 100 et les revend 1,50~\textgreek{\euro} pièce. Soit $x$ le nombre de
cornets vendus.
\begin{myenumerate}
\item Exprimer en fonction de $x$ le bénéfice réalisé par ce
vendeur. (On appelle bénéfice la différence entre le gain obtenu par
la vente et le coût d'achat des glaces.)
\item Combien de glaces doit-il vendre pour réaliser un bénéfice nul ?
\item Combien de glaces doit-il vendre pour réaliser un bénéfice de
 d'au moins 35~\textgreek{\euro} ?
\item Quel est son bénéfice s'il vend 70 glaces?
\end{myenumerate}