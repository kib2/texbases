{\em Dans ce problème, l'unité de longueur est le millimètre.}
\par $ABC$ est un triangle tel que $AB=42$, $AC=56$, $BC=70$. $M$ est
un point du segment $[BC]$.
\\La perpendiculaire à la droite $(AB)$ passant par $M$ coupe le
segment $[AB]$ en $H$.
\\La perpendiculaire à la droite $(AC)$ passant par $M$ coupe le
segment $[AC]$ en $K$.
\begin{myenumerate}
\item Démontre que $ABC$ est un triangle rectangle en $A$.
\item Démontre que le quadrilatère $AHMK$ est un rectangle.
\end{myenumerate}
\par\paragraph{Partie A} Dans cette partie, $BM=14$.
\begin{myenumerate}
\item Calcule les longueur $BH$ et $HM$.
\item En déduire la longueur $AH$.
\item Calcule le périmètre du rectangle $AHMK$.
\end{myenumerate}
\par\paragraph{Partie B} Dans cette partie, on pose $BM=x$ ($x$ en
mm).
\begin{myenumerate}
\item Donne un encadrement de $x$.
\item
\begin{enumerate}
\item Démontre que $HM=0,8\,x$.
\item Exprime la longueur $BH$ en fonction de $x$.
\item Déduis-en que $AH=42-0,6x$.
\end{enumerate}
\item
\begin{enumerate}
\item Calcule la valeur de $x$ pour laquelle $HM=AH$.
\item Pour la valeur obtenue, précise la nature du quadrilatère $AHMK$
et calcule son périmètre.
\end{enumerate}
\end{myenumerate}