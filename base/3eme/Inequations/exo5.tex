%@metapost: 302dm13.mp
%@metapost: 302dm13.1.mp
\par\compo{1}{302dm13}{1}{Sur la figure ci-contre, le triangle $ABC$ est
rectangle isocèle en $A$.\\On donne $BC=8,4$~cm. Le point $M$
appartient au segment $[BC]$. Le quadrilatère $MNPQ$ est un rectangle.
}
\begin{myenumerate}
\item Calcule la valeur de l'angle $\widehat{ABC}$. Déduis-en la
nature des triangles $BMN$ et $CPQ$.
\item On pose $BM=1,5$~cm. Calcule la longueur $MQ$ et l'aire du
rectangle $MNPQ$.
\item On pose $BM=x$.
\begin{enumerate}
\item Exprime les longueurs $MQ$ et $MN$ en fonction de $x$.
\item Déduis-en que l'aire du rectangle $MNPQ$, notée $\cal A$,
s'écrit
\[{\cal A}=8,4x-2x^2\]
\end{enumerate}
\end{myenumerate}
\par\compo{1}{302dm13.1}{0.8}{
\begin{myenumerate}
\setcounter{enumi}{3}
\item
\begin{enumerate}
\item Recopie et complète le tableau suivant :
\begin{center}
\begin{tabular}{|c|c|c|c|c|}
\hline
$x$ en cm&1&1,5&3&4\\
\hline
$\cal A$ en cm$^2$&\phantom{8,25}&\phantom{8,25}&\phantom{8,25}&\phantom{8,25}\\
\hline
\end{tabular}
\end{center}
\item Sur le graphique ci-dessous, on a tracé la représentation de
l'aire du rectangle $MNPQ$ en fonction de $x$. Place sur ce document
les points du tableau de la question précédente.
\end{enumerate}
\item Par lecture graphique, en faisant apparaître les traits de
construction, détermine :
\begin{enumerate}
\item pour quelles valeurs de $x$, l'aire du rectangle $MNPW$ est
4,9~cm$^2$.
\item pour quelle valeur de $x$, l'aire du rectangle est maximale.
\end{enumerate}
\end{myenumerate}
}