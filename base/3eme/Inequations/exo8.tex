%@metapost: 303dme13.mp
$ABCD$ est un rectangle tel que $AB=6$~cm et $AD=4$~cm.
\paragraph{Partie 1} $M$ est le point du segment $[BC]$ tel que
$BM=2$~cm ; $N$ est le point du segment $[CD]$ tel que $CN=2$~cm.
\begin{myenumerate}
\item
\begin{enumerate}
\item Construis la figure.
\item Calcule la longueur $AM$ sous la forme $a\sqrt b$ (b étant un
nombre entier le plus petit possible).
\end{enumerate}
\item Démontre que l'aire du quadrilatère $AMCN$ est 10~cm$^2$.
\end{myenumerate}
\paragraph{Partie 2} Les points $M$ et $N$ peuvent se déplacer
respectivement sur les segments $[BC]$ et $[CD]$ de façon que
$BM=CN=x$.
\begin{myenumerate}
\item Donne un encadrement de $x$ puis exprime l'aire des triangles
$ABM$ et $ADN$ en fonction de $x$.
\item Sur le graphique joint, représente graphiquement la fonction
linéaire $f$ définie par $f(x)=3x$.
\item Détermine graphiquement les coordonnées du point d'intersection
des 2 droites. Que représentent ces valeurs ?
\item
\begin{enumerate}
\item Pour quelle valeur de $x$ les aires des triangles $ABM$ et $ADN$
sont-elles égales ?
\item Pour cette valeur de $x$, calcule l'aire du quadrilatère $AMCN$.
\end{enumerate}
\end{myenumerate}
\[\includegraphics[scale=0.8]{303dme13.1}\]