{\em Dans cet exercice, l'unité de longueur est le centimètre}.
\par Soit une pyramide régulière :
\begin{itemize}
\item dont la base est un rectangle $ABCD$ de centre $O$ tel que
$AB=6$ et $BC=4$,
\item dont la hauteur $[OS]$ est perpendiculaire en $O$ au plan
$(ABC)$ et est telle que $OS=12$.
\end{itemize}
\par Un point $M$ décrit le segment $[OS]$ et on pose $OM=x$.
\begin{myenumerate}
\item Donne un encadrement de $x$.
\item Calcule le volume $\cal V$ de la pyramide ${\cal P}$ de sommet
$S$ et de base $ABCD$.
\item Exprime, en fonction de $x$, le volume ${\cal V}_1$ de la
pyramide ${\cal P}_1$ de sommet $M$ et de base $ABCD$.
\item Exprime, en fonction de $x$, le volume ${\cal V}_2$ de la partie
${\cal P}_2$ de la pyramide ${\cal P}$ extérieure à la pyramide ${\cal
P}_1$.
\item Le plan est rapporté à un repère orthogonal $(O,I,J)$. On
choisit pour unités : 1 centimètre sur l'axe des abscisses ; 1
millimètre sur l'axe des ordonnées.
\begin{enumerate}
\item Trace dans ce repère, la représentation graphique de la fonction
affine $f$ définie par \[f:x\mapsto96-8x\] Limite cette droite au
segment $[KH]$ qui correspond à $0\leqslant x\leqslant12$.
\item \`A l'aide du graphique précédent, répondre aux questions
suivantes (on laissera apparents les pointillés permettant les
réponses) :
\begin{itemize}
\item Combien vaut le volume ${\cal V}_2$ lorsque $x$ est égal à 7
centimètres ?
\item Où se situe le point $M$ si le volume ${\cal V}_2$ est égal à 32
centimètres cubes ?
\item Détermine un encadrement pour $x$ sachant que le volume ${\cal
V}_2$ est compris entre 55 et 67 centimètres cubes.
\\La précision est-elle satisfaisante ? Sinon, détermine cet
encadrement par des calculs appropriés (on donnera les résultats sous
forme exacte).
\end{itemize}
\end{enumerate}
\end{myenumerate}