\documentclass[twocolumn]{article}
\usepackage{kib_ds}
% ------------------------
% Crée par SyraBases
% ------------------------
\begin{document}

%----------------------
% Exercice numéro 1
%----------------------
\exo
\par
Nous avons tous les deux autant d'argent. Combien dois-je vous donner pour 
que vous ayez exactement $10$ \euro de plus que moi ?\\

\textbf{Réponse :}\\

$5$ \euro vous suffiront-ils ?

\par
\vspace{3mm}

%----------------------
% Exercice numéro 2
%----------------------
\exo
\par
Combien ai-je d'animaux domestiques, sachant que tous, sauf deux sont des 
chiens, tous sauf deux sont des chats, et tous sauf deux sont des perroquets ? 
(j'en ai plus de deux)\\

\textbf{Réponse :}\\

je n'ai que trois animaux, un chat, un chien, et un perroquet !

\par
\vspace{3mm}

%----------------------
% Exercice numéro 3
%----------------------
\exo
\par
Un homme va chez un ami qui a trois enfants, il lui demande l'âge de ceux-ci. L'ami répond: "Le produit des âges de mes enfants est égal à 72 et leur somme est égale au numéro de la maison d'en face". L'homme réfléchit et dit:\\

- "Tu as du oublier une donnée !"\\
- "Ah oui, mon aîné joue au football"\\
- "D'accord j'ai trouvé, c'est facile."\\

Qu'en pensez-vous ?\\

\textbf{Réponse :}\\

C'est un grand classique, il faut écrire toutes les combinaisons de nombres 
dont le produit est 72 !

Comme l'homme connaît le "numéro de la maison d'en face" il peut logiquement 
en déduire la combinaison cherchée. Mais il lui manque une donnée, cela 
signifie qu'il y a plusieurs solutions possibles.

Or les seules combinaisons qui ont une somme égales sont (2,6,6) et (3,3,8). 
(le numéro de la maison d'en face doit être 14)

Comme il y a un aîné les âges cherchés sont: 3 ans 3 ans et 8 ans.

\par
\vspace{3mm}

%----------------------
% Exercice numéro 4
%----------------------
\exo
\par
Le monstre du Loch Ness mesure 20 mètres plus la moitié de sa propre longueur.
Quelle est la taille de ce monstre ?\\

\textbf{Réponse :}\\

20 mètres plus la moitié de la longueur égal le tout! (comme la moitié+ la moitié = un tout)
donc 20 mètres représente la moitié de la longueur du monstre !\\
Le monstre mesure donc 40 mètres.

\par
\vspace{3mm}

%----------------------
% Exercice numéro 5
%----------------------
\exo
\par
Un libraire achète un livre 70 francs, le vend 80 francs, le rachète 90 
francs et le revend 100 francs quel est son bénéfice ?\\

\textbf{Réponse :}\\

Attention encore, on entends toutes les réponses 10f, 20f, 30f, 40 f ?
Mais quelle est la bonne ?\\

La réponse est 20 francs bien sur (10 francs de gains à chaque transaction)

\par
\vspace{3mm}

%----------------------
% Exercice numéro 6
%----------------------
\exo
\par
Il faut $56$ biscuits pour nourrir $10$ animaux. Il n'y a que des chats et des 
chiens. Les chiens mangent $6$ biscuits chacun, les chats n'en mangent que $5$. 
Combien y'a-t-il de chiens et de chats ?\\

\textbf{Réponse :}\\

Si tous les animaux étaient des chiens, il faudrait $60$ biscuits or il n'y 
en a que $56$, il en manque $4$.

Donc, comme les chats mangent un biscuit de moins, il faut remplacer $4$ 
chiens par des chats.

Il y a donc $4$ chats et $6$ chiens !!

On peut résoudre ce problème autrement, mais ce raisonnement est très joli.


\par
\vspace{3mm}

%----------------------
% Exercice numéro 7
%----------------------
\exo
\par
\textit{"Voici la tombe de Diophante, elle est merveilleuse car, en utilisant un artifice arithmétique, elle apprend toute sa vie. Il resta Enfant pendant le sixième de sa vie, après un autre douzième, ses joues se couvrirent de barbe, après un septième, il alluma le flambeau du mariage, cinq ans après il lui naquît un fils, mais celui-ci, enfant malheureux, quoique passionnément aimé, mourut arrivé à peine à la moitié de l'age de son père. Diophante vécut encore quatre ans, adoucissant sa douleur par des recherches sur la science des nombres"}\\

Déterminer toutes les étapes de la vie de Diophante.\\

\textbf{Réponse :}\\

Une petite équation fera l'affaire:

$A = A/6 + A/12 + A/7 + 5 + A/2 + 4$

qui donne: $A = 84$.

On retrouve alors les etapes de la vie de Diophante

- enfant jusqu'a 14 ans\\

- Quelques poils a 21 ans\\

- marié a 33 ans\\

- Père a 38 ans\\

- en deuil a 80 ans\\

- décédé a 84 ans\\

\par
\vspace{3mm}

%----------------------
% Exercice numéro 8
%----------------------
\exo
\par
Un nénuphar, qui double sa taille tous les jours, met 30 jours pour recouvrir 
la surface d'un étang. Combien de jours lui faut-il pour en recouvrir la 
moitié ?\\

\textbf{Réponse :}\\

29 jours.

\par
\vspace{3mm}

%----------------------
% Exercice numéro 9
%----------------------
\exo
\par
Un escargot grimpe le long d'un puits de 12 mètres de haut. Il parcourt 3 mètres chaque 
jour mais glisse de 2 mètres chaque nuit. Combien lui faudra-t-il de jours pour sortir 
de ce puits ?\\

\textbf{Réponse :}\\

En 24 heures, l'escargot grimpe d'un mètre (il monte de 3 mètres et il descend de 2 
mètres)\\

Donc au bout de 9 jours il est à 9 mètres, pendant la journée du dixième jour,
il grimpe de 3 mètres, ce qui fait 12 !! (il ne va pas redescendre arrivé en haut 
tout de même, soyez sérieux!).\\

Il lui faudra donc 10 jours.
\par
\vspace{3mm}

\end{document}
