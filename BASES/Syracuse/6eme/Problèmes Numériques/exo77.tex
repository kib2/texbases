%@metapost:6pbnumexo77.mp
\begin{myenumerate}
  \item\hfill\newline\compo{1}{6pbnumexo77}{0.5}{%
 Un dé à six faces est un cube. Sur chacune de ces faces, les
    chiffres de 1 à 6 sont indiqués en respectant la règle suivante :
    {\em la somme des faces opposées est toujours égale à
      7}.\\Indique la position des chiffres manquants sur ce schéma.
}
  \item François a lancé deux dés et a noté la somme des points obtenus à chaque lancer. Les résultats sont consignés dans le tableau ci-dessous.
    \begin{center}
      \begin{tabular}{|*{18}{c|}}
        \hline
        8&5&11&7&6&8&10&9&4&3&5&3&6&7&3&6&5&7\\
        \hline
        3&4&6&10&9&4&6&8&12&9&6&8&4&2&5&6&11&7\\
        \hline
      \end{tabular}
    \end{center}
    \begin{enumerate}
    \item Combien de fois François a-t-il obtenu une somme égale à 7 ?
    \item On veut savoir combien de fois est obtenue chaque somme.\\Fais un tableau permettant d'obtenir ces informations.
  \end{enumerate}
\item François change son jeu : il lance deux dés et note le
  produit des points obtenus à chaque lancer.
  \begin{enumerate}
  \item Quel est le plus petit résultat possible à obtenir ? Et le
    plus grand ?
  \item Peut-on obtenir tous les nombres entiers entre ces deux valeurs ?
    Explique pourquoi.
  \item Pour chaque nombre possible à obtenir, indique tous les
    lancers permettant d'atteindre ce nombre.
  \end{enumerate}
\end{myenumerate}