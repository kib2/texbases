%@Dif:3
Mado va faire ses courses au supermarché. Elle a dans son
porte-monnaie un billet de 50~\textgreek{\euro} et
3,75~\textgreek{\euro} en pièces.
\\Elle achète un rôti de b\oe uf de 800~g qu'elle paie
15,6~\textgreek{\euro}, puis 300~g de fromage de brebis qui coûtent
2,13~\textgreek{\euro} les 100~g et enfin 2~kg de noix qui coûtent en
tout 2,6~\textgreek{\euro}. Pour faire son repas, elle prendra la
salade dans son jardin. \og C'est une économie de
1,1~\textgreek{\euro}\fg\ se dit-elle en voyant le prix de la salade
au rayon \og légumes\fg.
\begin{myenumerate}
\item Combien a t-elle payé le fromage ?
\item Combien a t-elle payé en tout ses courses à la caisse du supermarché ?
\item Quelle somme restera t-il dans son porte-monnaie à la sortie du
supermarché ?
\item Combien coûte le kilogramme de noix ? et le kilogramme de rôti
de b\oe uf ?
\end{myenumerate}
%@Commentaire: Lecture difficile, on peut moduler l'énoncé de cet exercice en étant plus précis sur les prix.