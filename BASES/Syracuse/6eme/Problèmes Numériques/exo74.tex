%@Auteur:Thierry Joffredo\par
Résous les problèmes suivants sur la copie (\textit{on écrira les opérations en ligne, on fera une phrase de réponse par calcul, et on posera les opérations dans la marge}):
\begin{enumerate}
\item[\textbf{Pb1}] Aurélien collectionne les cartes Kopemon; il en a 251, c'est-à-dire 84 de plus que son ami Moussa. Combien de cartes possède Moussa?
\item[\textbf{Pb2}] Sur une cassette vidéo de 4 heures, j'enregistre un documentaire d'une durée de 1 h 45 min, puis une série de 26 min. Combien de temps d'enregistrement reste-t-il sur la cassette ? 
\item[\textbf{Pb3}] Une des unités de longueur couramment utilisée en Grande-Bretagne est le \textit{mile}; 1 \textit{mile} correspond à $1,609$ km environ. A combien de kilomètres correspond une distance de $3,2$ \textit{miles} ?
\item[\textbf{Pb4}] Dans une station-service, j'achète $40,5$ L d'essence au prix de $1,12$~\textgreek{\euro} le litre, ainsi que trois paquets de chewing-gum à $1,75$~\textgreek{\euro} le paquet. Si je donne 60~\textgreek{\euro}, combien de monnaie va-t'on me rendre?
\item[\textbf{Pb5}] Dans une salle de cinéma, il y a 35 rangées de 12 fauteuils. Le prix d'une place pour une séance est de $7,20$~\textgreek{\euro} pour un tarif plein, de $5,80$~\textgreek{\euro} pour un tarif réduit. Toutes les places sont occupées, et 250 personnes ont payé leur place au tarif plein.
\begin{enumerate}
\item Combien y a-t-il eu de spectateurs à cette séance ? Parmi eux, combien ont payé leur place au tarif réduit?
\item Quelle sera la recette de la séance (\textit{c'est-à-dire la somme totale gagnée par le gérant de la salle de cinéma})? 
\end {enumerate}
\end{enumerate}