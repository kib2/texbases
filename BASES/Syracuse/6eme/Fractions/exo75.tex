%@Auteur: François Meria\par
\begin{multicols}{2}
Recopier et calculer les produits suivants comme sur l'exemple. \\
\textit{Exemple} : \columnbreak
\begin{align*}
9\times \dfrac{8}{27}&= \dfrac{ 9\times8}{27}\\
                    &=\dfrac{9\times 8}{9\times 3}\\
                    &=\dfrac{8}{3}\\
\end{align*}
\end{multicols}
\begin{multicols}{4} \setlength{\columnseprule}{0.5pt}
$8\times \dfrac{16}{56}=$ \dotfill \\ \vskip 0.3cm $6\times
\dfrac{3}{54}=$ \dotfill \\ \vskip 0.3cm $8\times \dfrac{7}{64}=$
\dotfill \\ \vskip 0.3cm $5\times \dfrac{13}{40}=$ \dotfill \\
\vskip 0.3cm $8\times \dfrac{13}{72}=$ \dotfill \\ \vskip 0.3cm
$7\times \dfrac{1}{63}=$ \dotfill \\ \vskip 0.3cm $9\times
\dfrac{4}{54}=$ \dotfill \\ \vskip 0.3cm $7\times \dfrac{7}{7}=$
\dotfill \\ \vskip 0.3cm $6\times \dfrac{18}{30}=$ \dotfill \\
\vskip 0.3cm $6\times \dfrac{14}{24}=$ \dotfill \\ \vskip 0.3cm
$4\times \dfrac{2}{16}=$ \dotfill \\ \vskip 0.3cm $10\times
\dfrac{1}{90}=$ \dotfill \\ \vskip 0.3cm $3\times \dfrac{13}{15}=$
\dotfill \\ \vskip 0.3cm $4\times \dfrac{7}{12}=$ \dotfill \\
\vskip 0.3cm $2\times \dfrac{8}{10}=$ \dotfill \\ \vskip 0.3cm
$4\times \dfrac{2}{8}=$ \dotfill \\ \vskip 0.3cm $6\times
\dfrac{19}{36}=$ \dotfill \\ \vskip 0.3cm $2\times \dfrac{11}{4}=$
\dotfill \\ \vskip 0.3cm $1\times \dfrac{8}{7}=$ \dotfill \\
\vskip 0.3cm $5\times \dfrac{16}{5}=$ \dotfill \\ \vskip 0.3cm
\end{multicols}
\setlength{\columnseprule}{0pt}