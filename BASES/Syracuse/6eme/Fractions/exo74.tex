%@Auteur: François Meria\par
\begin{multicols}{2}
Recopier et calculer les produits suivants comme sur l'exemple. \\
\textit{Exemple} : \columnbreak
\begin{align*}
9\times \dfrac{8}{7}&= \dfrac{ 9\times8}{7}\\
                    &=\dfrac{72}{7}\\
\end{align*}
\end{multicols}
\begin{multicols}{4} \setlength{\columnseprule}{0.5pt}
    $9\times \dfrac{10}{6}=$ \dotfill \\ \vskip 0.3cm
$9\times \dfrac{3}{9}=$ \dotfill \\ \vskip 0.3cm $5\times
\dfrac{7}{7}=$ \dotfill \\ \vskip 0.3cm $4\times \dfrac{4}{2}=$
\dotfill \\ \vskip 0.3cm $5\times \dfrac{2}{3}=$ \dotfill \\
\vskip 0.3cm $10\times \dfrac{5}{6}=$ \dotfill \\ \vskip 0.3cm
$9\times \dfrac{9}{4}=$ \dotfill \\ \vskip 0.3cm $9\times
\dfrac{6}{4}=$ \dotfill \\ \vskip 0.3cm $5\times \dfrac{8}{1}=$
\dotfill \\ \vskip 0.3cm $7\times \dfrac{9}{3}=$ \dotfill \\
\vskip 0.3cm $12\times \dfrac{7}{19}=$ \dotfill \\ \vskip 0.3cm
$8\times \dfrac{17}{14}=$ \dotfill \\ \vskip 0.3cm $12\times
\dfrac{7}{14}=$ \dotfill \\ \vskip 0.3cm $12\times \dfrac{5}{19}=$
\dotfill \\ \vskip 0.3cm $16\times \dfrac{20}{15}=$ \dotfill \\
\vskip 0.3cm $6\times \dfrac{20}{10}=$ \dotfill \\ \vskip 0.3cm
$10\times \dfrac{19}{4}=$ \dotfill \\ \vskip 0.3cm $14\times
\dfrac{1}{19}=$ \dotfill \\ \vskip 0.3cm $13\times \dfrac{3}{16}=$
\dotfill \\ \vskip 0.3cm $12\times \dfrac{7}{11}=$ \dotfill \\
\vskip 0.3cm
\end{multicols}
\setlength{\columnseprule}{0pt}