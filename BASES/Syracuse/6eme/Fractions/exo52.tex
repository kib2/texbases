%@metapost:fractions604exo031.mp
%@Dif:2
Voici, ci-dessous, une droite graduée.
\[\includegraphics{fractions604exo031.1}\]
\begin{myenumerate}
  \item Colorie en rouge la longueur \og{}unité\fg.
  \item
    \begin{enumerate}
    \item Place, sur cette droite graduée, le point $M$ tel que:
      \begin{center}
        {\begin{cursive}La longueur $OM$ représente $\dfrac17$ de la longueur ``unité''.\end{cursive}}
      \end{center}
    \item Place, sur cette droite graduée, le point $N$ tel que:
      \begin{center}
        {\begin{cursive}La longueur $ON$ représente $\dfrac57$ de la longueur ``unité''.\end{cursive}}
      \end{center}
    \item Place, sur cette droite graduée, le point $P$ tel que:
      \begin{center}
        {\begin{cursive}La longueur $OP$ représente $\dfrac87$ de la longueur ``unité''.\end{cursive}}
      \end{center}
    \item Place, sur cette droite graduée, le point $R$ tel que:
      \begin{center}
        {\begin{cursive}La longueur $OR$ représente $\dfrac{15}7$ de la longueur ``unité''.\end{cursive}}
      \end{center}
    \end{enumerate}
  \item Lorsque l'on place le point $M$ sur la droite graduée tel que {\em la longueur $OM$ représente $\dfrac17$ de la longueur \og{}unité\fg{}}, on dit que {\bf l'abscisse du point $M$ est $\dfrac17$.}\par Donne alors l'abscisse des points $N$, $P$ et $R$.
  \item Place sur la droite graduée les points $A$ et $B$ d'abscisses respectives $\dfrac{22}7$ et $\dfrac{14}7$.
\end{myenumerate}