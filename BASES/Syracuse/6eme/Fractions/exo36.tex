%@metapost: 6emefractions604exo015.mp
Les segments ci-dessous ont tous été découpés en parts égales. Dans chacun des cas, place le point $M$ qui correspond à l'indication donnée.
\begin{myenumerate}
  \item {\begin{cursive}La longueur $AM$ représente $\dfrac35$ de la longueur $AB$.\end{cursive}}
\includegraphics{6emefractions604exo015.1}
  \item {\begin{cursive}La longueur $AM$ représente $\dfrac68$ de la longueur $AB$.\end{cursive}}
\includegraphics{6emefractions604exo015.2}
  \item {\begin{cursive}La longueur $AM$ représente $\dfrac86$ de la longueur $AB$.\end{cursive}}
\includegraphics{6emefractions604exo015.3}
  \item {\begin{cursive}La longueur $AM$ représente $\dfrac39$ de la longueur $AB$.\end{cursive}}
\includegraphics{6emefractions604exo015.4}
  \item {\begin{cursive}La longueur $AM$ représente $\dfrac{12}4$ de la longueur $AB$.\end{cursive}}
\includegraphics{6emefractions604exo015.5}
  \item {\begin{cursive}La longueur $AM$ représente $\dfrac{16}{20}$ de la longueur $AB$.\end{cursive}}
\includegraphics{6emefractions604exo015.6}
\end{myenumerate}