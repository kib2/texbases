%@metapost:6decimauxexo96.mp
\par\compo{1}{6decimauxexo96}{1}{
  \begin{myenumerate}
    \item Sur le segment $[AB]$, place le point $M_1$ d'abscisse 12.
    \item Sur le segment $[DC]$, place :
      \begin{itemize}
      \item le point $M_2$ d'abscisse 7,5;
      \item le point $M_3$ d'abscisse 8,1.
      \end{itemize}
    \item Sur le segment $[BC]$, place :
      \begin{itemize}
      \item le point $M_4$ d'abscisse 5,08;
      \item le point $M_5$ d'abscisse 5,14;
      \item le point $M_6$ d'abscisse 5,2;
      \item le point $M_7$ d'abscisse 5,24.
      \end{itemize}
    \item Sur le segment $[DA]$, place :
      \begin{itemize}
      \item le point $M_8$ d'abscisse 4,005;
      \item le point $M_9$ d'abscisse 4,025;
      \item le point $M_{10}$ d'abscisse 4,06;
      \item le point $M_{11}$ d'abscisse 4,09.
      \end{itemize}
  \end{myenumerate}
}
\begin{myenumerate}
  \setcounter{enumi}{4}
  \item
      \begin{itemize}
      \item Trace les segments $[M_1M_2]$ et $[M_1M_3]$.
      \item Trace les segments $[M_4M_{10}]$ et $[M_5M_{11}]$. Ils
        sont sécants en $N_1$.
      \item Trace les segments $[M_6M_8]$ et $[M_5M_9]$. Ils
        sont sécants en $N_2$.
      \item Trace la droite $(N_1N_2)$.{\em Si ta construction est
          correcte, cette droite passe par le point $M_1$.}
      \end{itemize}
\end{myenumerate}
