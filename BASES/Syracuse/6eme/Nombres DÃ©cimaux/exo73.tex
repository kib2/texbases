%@Titre: Numération égyptienne.
%@Auteur: D'après {\sc M. Rousselet}. {\em Le calcul et la géométrie au temps des pharaons.}\par
Les premiers nombres qui ont été écrits en \'Egypte datent de 5\,000
ans. La numération égyptienne utilise sept symboles pour écrire les
nombres.
\begin{center}
  \begin{minipage}{0.8\linewidth}
\begin{multicols}{3}
\begin{description}
\item[\huge\textpmhg{\Hone}] pour 1
\item[\huge\textpmhg{2}] pour 10
\item[\huge\textpmhg{3}] pour 100
\item[\huge\textpmhg{4}] pour 1\,000
\item[\huge\textpmhg{5}] pour 10\,000
\item[\huge\textpmhg{6}] pour 100\,000
\item[\huge\textpmhg{7}] pour 1\,000\,000
\end{description}
\end{multicols}
  \end{minipage}
\end{center}
Ces dessins représentent respectivement un bâton, une voûte, une corde
enroulée, une fleur de lotus, une doigt pointé, un tétard et un dieu
qui lève les bras vers le ciel.
\par Dans l'écriture d'un nombre, les hiéroglyphes sont juxtaposées
dans n'importe quel ordre mais le même hiéroglyphe ne peut être
dessiné plus de neuf fois. Pour lire un nombre, on additionne les
valeurs de tous les hiéroglyphes qui ont été utilisés dans son
écriture. Par exemple,
\begin{center}
  \begin{tabular}{cc}
    {\huge\pmglyph{\HCthousand-\HCthousand-{{\HXthousand-\HXthousand-\HXthousand}+{\HXthousand-\HXthousand}}-\Hthousand-\Hthousand-\Hhundred-\Hhundred-\Hten-{{\Hone-\Hone}+{\Hone-\Hone}}}}&252\,214\\
    {\huge\pmglyph{{{\Hone-\Hone}+{\Hone-\Hone}}-\Hten-\Hten-\Hhundred-\Hhundred-\Hhundred-\Hthousand-\Hthousand-\HXthousand-\HXthousand}}&22\,324\\
    {\huge\pmglyph{\Hhundred-\Hone-\Hthousand}}&1\,101\\
  \end{tabular}
\end{center}
\begin{myenumerate}
\item Lis ces trois nombres.
\par{\huge\pmglyph{\Hhundred-\Hhundred-\Hhundred-\Hten-\Hten-\Hten-\Hone-\Hone-\Hone-\Hone-\Hone}
\hfill
\pmglyph{\Hthousand-\Hthousand-{{\Hten-\Hten}+{\Hhundred}}-\Hone}
\hfill
\pmglyph{\Hmillion-\Hmillion-{{\HCthousand-\HCthousand}+{\Hthousand-\Hthousand-\Hthousand}}-\Hhundred-\Hten-\Hten}
}
\item \'Ecris les nombres 29;87;1\,437 et 24\,765 avec des
  hiéroglyphes.
\item On écrit un nombre avec les chiffres de ta date de
  naissance. Par exemple, si c'est le 15 octobre 1993 alors le nombre
  choisi sera 151\,093.
\\\'Ecris, à l'aide des chiffres égyptiens, le nombre qui correspond à
ta date de naissance.
\item Quel est le plus grand nombre écrit avec les six premiers
  chiffres égyptiens ?
\item Quelle différence {\em très importante} existe-t-il avec notre
  écriture des nombres ?
\end{myenumerate}