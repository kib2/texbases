Complète les tableaux suivants\par
\renewcommand{\arraystretch}{2}
\begin{tabularx}{0.48\linewidth}{|c|X|}
\hline
{\small Nombre décimal}&{\small \'Ecriture sous la forme d'une fraction décimale}\\
\hline
17,52&\\
\hline
8,632&\\
\hline
0,000\,004&\\
\hline
&\multicolumn{1}{c|}{$\dfrac{54\,832}{\strut 100}$}\\
\hline
&\multicolumn{1}{c|}{$\dfrac{25}{\strut 1\,000}$}\\
\hline
\end{tabularx}
\hfill
\begin{tabularx}{0.5\linewidth}{|c|c|}
\hline
{\small Nombre décimal}&{\small Décomposition décimale}\\
\hline
7,15&\\
\hline
1\,359,23&\\
\hline
500,042&\\
\hline
&\multicolumn{1}{c|}{$40+80+7+0,6+0,09+0,007$}\\
\hline
&\multicolumn{1}{c|}{$40\,000+10+0,5+0,000\,6$}\\
\hline
\end{tabularx}
\par
\begin{center}
\begin{tabularx}{0.70\linewidth}{|c|X|}
\hline
{\small Nombre décimal}&\multicolumn{1}{c|}{\small Décomposition fractionnaire}\\
\hline
7,789&\\
\hline
1\,080,023\,4&\\
\hline
&\multicolumn{1}{c|}{$20+5+\dfrac7{\strut 10}+\dfrac4{100}$}\\
\hline
&\multicolumn{1}{c|}{$600+2+\dfrac3{\strut 1\,000}$}\\
\hline
\end{tabularx}
\end{center}