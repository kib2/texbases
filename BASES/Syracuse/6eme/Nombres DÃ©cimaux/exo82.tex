\begin{myenumerate}
\item
\begin{enumerate}
\item Remplis les tableaux ci-dessous.
\[\begin{tabular}{|c|c|c|c|c|c}
\cline{1-5}
Poids (kg)&1&2&3,5&4,731&\rnode{A}{}\\
\cline{1-5}
Poids (g)&\phantom{4,731}&\phantom{4,731}&\phantom{4,731}&\phantom{4,731}&\rnode{B}{}\\
\cline{1-5}
\end{tabular}
\]
\[
\begin{tabular}{|c|c|c|c|c|c}
\cline{1-5}
Poids (g)&1&4&6,3&7,45&\rnode{C}{}\\
\cline{1-5}
Poids (cg)&\phantom{4,731}&\phantom{4,731}&\phantom{4,731}&\phantom{4,731}&\rnode{D}{}\\
\cline{1-5}
\end{tabular}
\]
\[
\begin{tabular}{|c|c|c|c|c|c}
\cline{1-5}
Poids (hg)&1&3&5,7&67&\rnode{E}{}\\
\cline{1-5}
Poids (dag)&\phantom{4,731}&\phantom{4,731}&\phantom{4,731}&\phantom{4,731}&\rnode{F}{}\\
\cline{1-5}
\end{tabular}
\]
\ncbar[angleA=0,angleB=0]{->}{A}{B}
\naput{$\times\ldots$}
\ncbar[angleA=0,angleB=0]{->}{C}{D}
\naput{$\times\ldots$}
\ncbar[angleA=0,angleB=0]{->}{E}{F}
\naput{$\times\ldots$}
\item Donne alors une règle pour multiplier par 10; 100; 1\,000.
\[\psshadowbox{\begin{minipage}{300pt}
\begin{cursive}
$\bullet$ Si je multiplie un nombre décimal par 10 alors \dotfill\par\dotfill
\par
$\bullet$ Si je multiplie un nombre décimal par 100 alors \dotfill\par\dotfill
\par
$\bullet$ Si je multiplie un nombre décimal par 1\,000 alors \dotfill\par\dotfill
\end{cursive}
\end{minipage}}
\]
\item Applique ces règles pour donner les résultats des multiplications suivantes :
\[\Eqalign{
12,5\times100&=\ldots\ldots\ldots&1,45\times10&=\ldots\ldots\ldots&54,2457\times1\,000&=\ldots\ldots\ldots\cr
36\times1\,000&=\ldots\ldots\ldots&3,5\times100&=\ldots\ldots\ldots&24,1\times10&=\ldots\ldots\ldots\cr
}\]
\end{enumerate}
\end{myenumerate}