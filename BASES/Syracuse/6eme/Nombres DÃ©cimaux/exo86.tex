%@metapost:6decimauxexo86.mp
Sur la figure ci-dessous, six points sont placés. Sur chacun d'eux, un
nombre décimal est indiqué.
\[\includegraphics{6decimauxexo86.1}\]
Sachant que :
\begin{itemize}
\item le point $A$ correspond au nombre cent fois plus grand que 0,0221;
\item le point $B$ correspond au nombre cent fois plus petit que 2\,210;
\item le point $R$ correspond au nombre dix fois plus petit que 22,01;
\item le point $E$ correspond au nombre mille fois plus grand que 0,221;
\item le point $I$ correspond au nombre dix fois plus grand que celui
  du point $R$;
\item le point $T$ correspond au nombre mille fois plus petit que
  celui du point $E$.
\end{itemize}
retrouve les points $A$, $B$, $R$, $E$, $I$ et $T$; puis trace :
\begin{itemize}
\item le cercle de centre $A$ et de rayon $AE$ (on gardera la partie à
  l'intérieur du cercle déjà tracé);
\item le cercle de centre $B$ et de rayon $BT$ (on gardera la partie à
  l'intérieur du cercle déjà tracé);
\item le cercle de centre $R$ et de rayon $AE$ (on gardera la partie à
  l'intérieur du cercle déjà tracé);
\item le cercle de centre $T$ et de rayon $AE$ (on gardera la partie à
  l'intérieur du cercle déjà tracé);
\item le cercle de centre $I$ et de rayon $IR$ (on gardera la partie à
  l'intérieur du cercle déjà tracé);
\item le cercle de centre $E$ et de rayon $AE$ (on gardera la partie à
  l'intérieur du cercle déjà tracé);
\end{itemize}
