%@P:exocorcp
\begin{myenumerate}
  \item Maman a acheté 10 tickets de cantine à 2,54~\textgreek{\euro}
    l'unité et 10 tickets de bus à 0,85~\textgreek{\euro}
    l'unité. Combien a-t-elle dépensé ?
  \item Le CDI doit acheter 100 livres de mathématiques à
    17,19~\textgreek{\euro} l'unité et 100 livres de français à
    15,8~\textgreek{\euro} l'unité. Combien le CDI va-t-il dépensé ?
  \item La SNCF doit poser 1\,000 rails de longueur 3,2~m l'unité et
    1\,000 rails de longueur 5,82~m l'unité. Mis bout à bout, quelle
    longueur représente tous ces rails ?
\end{myenumerate}
%@Correction:
\begin{multicols}{2}
\begin{myenumerate}
  \item\hfill\newline $\left.
      \begin{array}{l}
        10\times2,54=25,4\\
        10\times0,85=8,5\\
      \end{array}
      \right\}25,4+8,5=33,9~\mbox{\textgreek{\euro}}$.
    \item\hfill\newline$\left.
      \begin{array}{l}
        100\times17,19=1\,719\\
        100\times158=1\,580\\
      \end{array}
      \right\}1\,719+1580=3\,299~\mbox{\textgreek{\euro}}$.\par\columnbreak\par
    \item\hfill\newline$\left.
      \begin{array}{l}
        1\,000\times3,2=3\,200\\
        1\,000\times5,82=5\,820\\
      \end{array}
      \right\}3\,200+5\,820=9\,020~\mbox{m}$.
\end{myenumerate}
\end{multicols}