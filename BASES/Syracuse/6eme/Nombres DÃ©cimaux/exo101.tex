%@Auteur:Véronique Glaçon\par
Colorie de la même couleur les cases contenant les écritures qui représentent les mêmes nombres.
\renewcommand{\arraystretch}{1.5} 
\[\begin{tabular}{|>{\centering}m{3cm}|>{\centering}m{3cm}|>{\centering}m{3cm}|}
 \hline $\dfrac{\nombre{1101}}{\nombre{1000}}$ & $1+\dfrac{1}{10}$ & une unité et cent un millièmes \tabularnewline
 \hline une unité et onze centièmes & $1+\dfrac{1}{100}$ & $\dfrac{111}{100}$ \tabularnewline
 \hline $\dfrac{101}{100}$ & une unité et un centième & $1+\dfrac{110}{\nombre{1000}}$ \tabularnewline
 \hline $\dfrac{11}{10}$ & une unité et un dixième & $\dfrac{110}{100}$ \tabularnewline
 \hline
\end{tabular}\]
\renewcommand{\arraystretch}{1} 