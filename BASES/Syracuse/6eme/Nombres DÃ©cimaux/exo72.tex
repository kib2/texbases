%@Auteur: d'après {\sl perso.orange.fr/gerard.cissa/}\par
\begin{myenumerate}
  \item Thomas, Charles, Clarisse, Edwige et Florent ont participé à
    un concours de mathématiques.
\par Voici leurs scores:
\begin{description}
  \item[Thomas] 17,23 points
  \item[Charles] 17,308 points
  \item[Clarisse] 17,205 points
  \item[Edwige] 17,31 points
  \item[Florent] 17,298 points.
\end{description}
Qui est déclaré vainqueur? Donne leur classement.
\item
  \begin{enumerate}
  \item Thomas affirme que son score est l'arrondi de 17,239 à
    $\dfrac1{10}$ près. A-t-il raison?
  \item Charles constate qu'il aurait autant qu'Edwige si tous les
    scores avaient été arrondis au $\dfrac1{100}$.\par\'Ecris toi même
    les arrondis de leurs scores. Que penses-tu alors de l'affirmation
    de Charles?
  \item Florent est sûr que son score est compris entre 17,29 et
    17,31. Est-ce vrai?
  \end{enumerate}
\item Parmi les questions qui étaient posées à nos cinq candidats, il
  y avait celle-ci :
\begin{quote}
  \'Ecris sous la forme d'une fraction décimale chacun des nombres
  suivants :
  \begin{itemize}
  \item quarante cinq unités trois dixièmes sept millièmes
  \item huit dizaines deux centièmes
  \item quinze centièmes et vingt-cinq millièmes.
  \end{itemize}
\end{quote}
Qu'aurais-tu répondu ?
\end{myenumerate}