%@P:exocorcp
%@metapost:6proporexo31.mp
%@Dif:3
\begin{tabularx}{\linewidth}{cX}
\large\dbend&{\em Pour l'évaluation de cet exercice, le
    raisonnement représentera une très grande part de votre
    résultat. Attention à utiliser un vocabulaire précis; à faire des
    phrases courtes.}\\
\end{tabularx}
\par
Un trésor est caché dans une île. Pour le retrouver, une vieille
carte, {\em reproduite ci-dessous}, signale une source $S$, un rocher en forme d'éléphant $R$ et un arbre géant $A$.
\par Il est écrit : {\em le trésor est à l'intérieur du triangle $SAR$, à plus de 500~m du rocher, à plus de 300~m de l'arbre et à plus de 200~m de la source.}
\begin{myenumerate}
\item Robinson retrouve la source et le rocher. Il mesure la distance
  entre la source et le rocher et trouve 600~m.\\Quelle information
  cela donne-t-il sur la carte ?\\{\em Explique par des phrases claires
    et précises ton raisonnement}.
\item En t'aidant de la carte, détermine les distances réelles entre la source et l'arbre; entre le rocher et l'arbre.\\{\em Explique par des phrases claires
    et précises ton raisonnement}.
\item Représente le triangle $SAR$ de la carte.{\em Sur la carte ci-dessous}, hachure en bleu la zone où ne se trouve pas le trésor et colorie en vert la zone où il se trouve.\\{\em Explique par des phrases claires et précises ton raisonnement}.
\end{myenumerate}
\[\includegraphics[scale=0.707]{6proporexo31.1}\]
%@Correction:
\[\includegraphics[scale=0.707]{6proporexo31.2}\]
