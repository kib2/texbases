%@Dif:3
Un trésor est caché dans une île. Pour le retrouver, une vieille carte signale une source $S$, un rocher en forme d'éléphant $R$ et un arbre géant $A$.
\par Il est écrit : {\em le trésor est à l'intérieur du triangle $SAR$, à plus de 500~m du rocher, à plus de 300~m de l'arbre et à plus de 200~m de la source.}
\begin{myenumerate}
\item Robinson retrouve la source et le rocher. Il mesure la distance entre la source et le rocher et trouve 600~m. Sur la carte, la distance entre $S$ et $R$ est 6~cm. Quelle est l'échelle de la carte ?
\item La distance entre $S$ et $A$ sur la carte est de 7~cm et la distance entre $R$ et $A$ est de 8~cm. Quelles sont les distances réelles entre la source et l'arbre ? le rocher et l'arbre ?
\item Représente le triangle $SAR$ de la carte. Hachure en bleu la zone où ne se trouve pas le trésor et colorie en vert la zone où il se trouve.  
\end{myenumerate}

