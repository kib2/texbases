%@Auteur: Nathalie Lespinasse\par
Lorsque Jérôme conduit sans à-coups, la consommation du moteur de sa voiture et la distance parcourue sont proportionnelles. Jérôme a parcouru 120 kilomètres avec sa voiture et le moteur a consommé 7,2~L de carburant.
\begin{myenumerate}
  \item Recopie et complète le tableau :
    \begin{center}
      \begin{tabular}{|c|c|c|}
        \hline
        Consommation en L&7,2&\\
        \hline
        Distance parcourue en km&120&100\\
        \hline
      \end{tabular}
    \end{center}
  \item Combien le moteur consomme-t-il lorsque Jérôme parcourt 300~km avec sa voiture ?
  \item Quelle distance Jérôme peut-il parcourir dans les mêmes conditions avec 15 litres de carburant ?
\end{myenumerate}
{\em \'Ecris les calculs te permettant de répondre aux questions 1/, 2/ et 3/.}