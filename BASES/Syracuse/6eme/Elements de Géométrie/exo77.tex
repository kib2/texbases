%@Auteur:Thierry Joffredo\par
\'Ecris un programme de construction pour la figure suivante:
\begin{center}
\psset{xunit=0.5cm,yunit=0.5cm,dotstyle=*,dotsize=3pt 0,linewidth=0.8pt,arrowsize=3pt 2,arrowinset=0.25}
\begin{pspicture*}(-6,-6)(6,6)
\rput{30}{
\pscircle(0,0){2.5}
\psline(-5,0)(5,0)
\psline(2.5,-6.766666666666667)(2.5,10.5)
\psline(2.8,0)(2.8,0.3)
\psline(2.8,0.3)(2.5,0.3)
\psline(1.2,0.2)(1.3,-0.2)
\psline(3.7,0.2)(3.8,-0.2)
\psdots(0,0)
\psdots(-5,0)
\psdots(5,0)
\psdots(2.5,-4.330127018922193)
\psdots(2.5,4.330127018922194)
\psdots(2.5,0)
\psdots(-3,-4)
\psline[linestyle=dashed](0,0)(-3,-4)
}
\rput[bl](1.8,0){$H$}
\rput[bl](-0.5,4){$D$}
\rput[bl](3.4,-3){$C$}
\rput[bl](4.8,2){$B$}
\rput[bl](-5.2,-2.8){$A$}
\rput[bl](-0.2,0.4){$O$}
\rput[bl](-0,-3){2.5 cm}
\end{pspicture*}
\end{center}
{\em On utilisera le vocabulaire mathématique approprié. On fera des
  phrases courtes et précises.}