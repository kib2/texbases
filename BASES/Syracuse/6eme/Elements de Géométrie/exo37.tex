%necessite mathrsfs
{\em On fera une figure différente pour chaque question.}
\begin{myenumerate}
\item
\begin{enumerate}
\item Trace un segment $[BC]$ de longueur 6,4~cm.
\item Trace le cercle $\mathscr{C}_1$ de diamètre $[BC]$.
\item Trace le cercle $\mathscr{C}_2$ de centre $C$, de rayon
  4~cm. Il coupe $\mathscr{C}_1$ en $E$ et $F$.
\item Trace en bleu le petit arc de cercle de $E$ à $F$ du cercle
  $\mathscr{C}_1$ ; en vert le petit arc de cercle de $E$ à $F$ du cercle
  $\mathscr{C}_2$.
\end{enumerate}
\item
\begin{enumerate}
\item Trace un segment $[AB]$ de longueur 6~cm. Sur ce segment,
  place un point $C$ tel que $AC=4$~cm.
\item Trace le cercle de centre $B$ qui passe par $C$. Calcule son
  rayon.
\end{enumerate}
\item
\begin{enumerate}
\item Trace un segment $[AB]$ de longueur 6,5~cm.
\\Trace le cercle de centre $A$ et de rayon 4~cm ; ce cercle coupe
la droite $(AB)$ en deux points $E$ et $F$. On appelle $E$ celui qui
est le plus près de $B$.
\item Calcule les longueurs $EB$ et $FB$.
\end{enumerate}
\end{myenumerate}