%@geogebra:6elmtsgeoexo33.ggb
Place trois points $A$, $B$ et $C$ sur une droite $(d)$ et trois
points $D$, $E$ et $F$ sur une droite $(d_1)$.
\begin{myenumerate}
\item Trace en rouge les droites $(AE)$ et $(DB)$. Leur point
d'intersection s'appelle $J$.
\item Trace en bleu les segments $(AF)$ et $(DC)$. Leur point
d'intersection s'appelle $I$.
\item Trace en vert les demi-droites $(BF)$ et $(EC)$. Leur point
d'intersection s'appelle $K$.
\item Que peut-on dire des points $I$, $J$ et $K$ ?
\end{myenumerate}