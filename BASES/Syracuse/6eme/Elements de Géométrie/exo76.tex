%@metapost:6elmtsgeoexo76.mp
%@Auteur: Nathalie Lespinasse\par
\begin{multicols}{3}
  \begin{description}
    \item[Compétence \no1] : {\em Je sais tracer des rayons d'un cercle.}
\par Trace sur le cercle de centre $O$ ci-dessous trois rayons.
\\ Code ta figure.
\[\includegraphics{6elmtsgeoexo76.1}\]
\par\columnbreak\par
\item[Compétence \no2] : {\em Je sais tracer des cordes d'un cercle.}
\par Trace quatre cordes du cercle $\mathscr C$ ci-dessous.
\[\includegraphics{6elmtsgeoexo76.2}\]
\par\columnbreak\par
\item[Compétence \no3] : {\em Je sais tracer des diamètres d'un cercle.}
\par Trace trois diamètres du cercle $\mathscr C$ ci-dessous.
\[\includegraphics{6elmtsgeoexo76.3}\]
\end{description}
  \end{multicols}
\begin{description}
\item[Compétence \no4] : {\em Je connais le vocabulaire associé au cercle.}
  \begin{myenumerate}
    \item Rédige un petit texte pour décrire la figure ci-dessous, en utilisant les mots {\em centre}, {\em rayon}, {\em diamètre} et {\em corde}.
\par\compo{4}{6elmtsgeoexo76}{1}{\dotfill\par\dotfill\par\dotfill\par\dotfill\par\dotfill\par\dotfill}
\item Observe la figure ci-dessous et complète les phrases suivantes en utilisant les mots qui conviennent :\par
\compo{5}{6elmtsgeoexo76}{1}{
  \begin{enumerate}
  \item $I$ est \dotfill du cercle $\mathscr C$.
  \item $[ST]$ est \dotfill du cercle $\mathscr C$.
  \item $[IM]$ est \dotfill du cercle $\mathscr C$.
  \item $[AB]$ est \dotfill du cercle $\mathscr C$.
  \item $[IB]$ est \dotfill du cercle $\mathscr C$.
  \end{enumerate}
}
  \end{myenumerate}
\end{description}