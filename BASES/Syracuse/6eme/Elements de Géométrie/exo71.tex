%@Auteur: D'après IREM Strasbourg.\par
\begin{myenumerate}
  \item Trace un segment $[AB]$ de longueur 5~cm.
  \item Place un point $C$ situé à 5~cm du point $A$ (on écrira
    $AC=5$~cm). Trace le triangle $ABC$. Quelle est la particularité
    du triangle ABC ? Comment appelle-t-on un tel triangle ?
  \item Place sur la même figure quatre autres points situés à 5~cm du
    même point $A$. On les appellera $D$, $E$, $F$ et $G$.\par Avec
    quel instrument peut-on tracer une ligne qui passe par les points
    $C$, $D$, $E$, $F$ et $G$ ? Comment appelle-t-on cette ligne ?
  \item Place un point $H$ sur le cercle de centre $A$ et de rayon
    5~cm. Sans la mesurer, combien vaut la distance $AH$ ? Quelle est
    la particularité du triangle $ABH$ ?
  \item Place un point $J$ de telle façon que le triangle $ABJ$ soit
    isocèle en $A$. Où se trouve ce point $J$ ? Pourquoi ?
  \item Place deux points $K$ et $L$ situés à moins de 5~cm du point
    $A$. Où se situent-ils par rapport au cercle déjà tracé ?
  \item Place deux points $M$ et $N$ situés à plus de 5~cm du point
    $A$. Où se situent-ils par rapport au cercle déjà tracé ?
\end{myenumerate}