%@metapost:6elmtsgeoexo80.mp
%@Auteur: IREM Strasbourg\par
\begin{center}
  \includegraphics{6elmtsgeoexo80.1}
  \includegraphics{6elmtsgeoexo80.2}
  \includegraphics{6elmtsgeoexo80.3}
  \includegraphics{6elmtsgeoexo80.4}
  \includegraphics{6elmtsgeoexo80.5}\\
  \includegraphics{6elmtsgeoexo80.6}
  \includegraphics{6elmtsgeoexo80.7}
  \includegraphics{6elmtsgeoexo80.8}
  \includegraphics{6elmtsgeoexo80.9}
  \includegraphics{6elmtsgeoexo80.10}
\end{center}
On a reproduit ci-dessus le film de la construction de la figure
\ding{181}, c'est à dire les différentes étapes qui permettent de
construire cette figure. On donne les dimensions suivantes :
\[AB=AC=5~\mbox{cm}\kern0.1\linewidth BD=7~\mbox{cm}\kern0.1\linewidth
BC=4~\mbox{cm}\kern0.1\linewidth CD=6~\mbox{cm}\]
et on sait que $E$ est le milieu du segment $[BD]$.
\begin{myenumerate}
  \item Reproduis cette figure en respectant les dimensions indiquées, ainsi que l'ordre de construction donné.
  \item \'Ecrire les différentes instructions qui permettent de tracer
    la figure \ding{181}. Chaque instruction correspond à une étape
    numérotée de \ding{172} à \ding{181}. {\em Une seule instruction est à exécuter à chaque étape}.
\end{myenumerate}