\begin{myenumerate}
\item Recopie et complète les phrases suivantes avec le vocabulaire
convenable :
\begin{itemize}
\item[$\diamondsuit$] La \dots\dots\dots $(d)$ coupe le
\dots\dots\dots $[IJ]$ au \dots\dots\dots $K$ mais ce n'est pas le
milieu du \dots\dots\dots $[IJ]$
\item[$\diamondsuit$] La \dots\dots\dots $(d_1)$ coupe le
\dots\dots\dots $({\cal C})$ de centre $O$ et de rayon 4~cm en deux
\dots\dots\dots $A$ et $B$ mais le \dots\dots\dots $[AB]$ n'est pas un
diamètre du cercle $({\cal C})$.
\item[$\diamondsuit$] Le \dots\dots\dots $C$ appartient à la
\dots\dots\dots $[BA)$ mais il n'appartient pas au \dots\dots\dots
$[BA]$.
\item[$\diamondsuit$] Le \dots\dots\dots $A$ est le \dots\dots\dots
d'intersection de la \dots\dots\dots $(BC)$ et de la \dots\dots\dots
$(EF)$. Le \dots\dots\dots $E$ appartient au \dots\dots\dots $[FA]$
mais le \dots\dots\dots $A$ n'appartient pas au \dots\dots\dots
$[BC]$.
\end{itemize}
\item Pour chaque cas, fais une figure correspondante à la phrase.
\end{myenumerate}