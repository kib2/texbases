%@metapost:6elmtsgeoexo85.mp
%@Auteur:Véronique Glaçon\par
En utilisant le codage de la figure, complète le tableau ci-dessous :
\[\includegraphics{6elmtsgeoexo85.1}\]
\renewcommand{\arraystretch}{1.3}
\begin{center}
\begin{tabular}{|p{9cm}|c|c|c|}
  \hline \centering{Affirmation}& vrai & faux & je ne sais pas \\
  \hline Le triangle PTV est équilatéral &&& \\
  \hline [TS] est un rayon du cercle &&& \\
  \hline Le triangle IPS est isocèle en I &&& \\ 
  \hline Le quadrilatère PASI est un losange &&& \\   
  \hline Le quadrilatère TIKM est un losange &&& \\   
  \hline Le triangle PTW est un triangle isocèle en W &&& \\   
  \hline Le triangle PTV est isocèle en T &&& \\   
  \hline Le triangle TPI est isocèle en P &&& \\    
  \hline Les segments [MK] et [PI] ont la même longueur &&& \\    
  \hline
\end{tabular}
\end{center}
\renewcommand{\arraystretch}{1}