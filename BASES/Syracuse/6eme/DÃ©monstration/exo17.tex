%@P:exocorcp
%@metapost:6demoexo17.mp
\par\compo{1}{6demoexo17}{1}{{\em La figure ci-contre n'est pas en
    vraie grandeur.\par L'unité de longueur est le centimètre.}
\par
\begin{myenumerate}
  \item\`A l'aide de la figure ci-contre, explique la forme
    géométrique des polygones $IJL$, $IRK$ et $JTPK$.
  \item Calcule le périmètre du polygone $JIRKT$.
  \item Reproduis, {\em en vraie grandeur}, la figure ci-contre.
\end{myenumerate}
}
%@Correction:
\begin{myenumerate}
  \item Comme les droites $(IJ)$ et $(IL)$ sont perpendiculaires alors
    le triangle $IJL$ est rectangle en $I$.\par Comme les côtés du
    triangle $IJR$ ont tous la même mesure alors le triangle $IKR$ est
    équilatéral.\par Comme les côtés du quadrilatère $JTPK$ ont tous
    la même longueur alors le quadrilatère $JTPK$ est un losange.
  \item ${\cal P}=JI+IR+RK+KT+TJ=5+6+6+4+9=30$~cm.
\end{myenumerate}