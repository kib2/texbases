%@metapost:demo604exo010.mp
%@Auteur: d'après APMEP.\par
\og Voici trois textes d'exercices et les figures qu'un élève a dessinées. Je t'affirme que cet élève a mis dans son dessin plus d'informations, c'est-à-dire plus de choses, que dans le texte donné.\par Il a fait ce que l'on appelle {\em un cas particulier} de figure et il faut éviter cela.\fg
\begin{center}
  \begin{tabular}{|m{7cm}|m{5cm}|m{4cm}|}
    \hline
    \multicolumn{1}{|c|}{Cas \no1}&\multicolumn{1}{c|}{Cas \no2}&\multicolumn{1}{c|}{Cas \no3}\\
    \hline
\includegraphics{demo604exo010.1}&\multicolumn{1}{c|}{\includegraphics{demo604exo010.2}}&\multicolumn{1}{c|}{\includegraphics{demo604exo010.3}}\\
\hline
    $(d_1)$, $(d_2)$ et $(d_3)$ sont trois droites. $A$, $B$ et $C$ sont trois points de la droite $(d)$.\par
Par ces trois points, on mène trois droites parallèles entre elles, qui coupent la droite $(d_2)$ en $A'$, $B'$, $C'$.
\par Par $A'$, $B'$, $C'$, on trace trois droites parallèles entre elles qui coupent la droite $(d_2)$ en $E$, $F$, $G$.
\par Que remarques-tu ?
&Dessine deux segments $[AB]$ et $[CD]$ ayant le même milieu $I$.
\par Que remarques-tu ?
&Dessine un rectangle $ABCD$ et les milieux de ses quatre côtés. Relie ces milieux.
\par Que remarques-tu ?\\
\hline
  \end{tabular}
\end{center}