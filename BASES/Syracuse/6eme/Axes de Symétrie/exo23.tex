%@Dif:3
Sur un plan, les maisons de trois amis sont représentées par les
points $A$, $B$, $C$ tel que $AB=6$~cm; $AC=7$~cm et $BC=8$~cm. Ils
veulent construire un puits $P$ à égale distance de leur maison.
\begin{myenumerate}
\item Fais une figure que l'on complétera au fur et à mesure de
l'exercice.
\item Où placer le puits $P$ pour qu'il soit situé à égale distance de
$A$ et de $B$ ? Explique ta réponse.
\item Où placer le puits $P$ pour qu'il soit situé à égale distance de
$A$ et de $C$ ? Explique ta réponse.
\item Déduis-en la position du puits.
\end{myenumerate}
%@Commentaire: Utilisation de la propriété fondamentale de la médiatrice d'un segment. Exercice \og concret\fg.