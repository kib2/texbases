%@Auteur: François Meria\par
\begin{multicols}{2}
\begin{center}
\pspicture(5,4)
    \pstGeonode[PointSymbol=none,PosAngle={235,0,45}](0,0){S}(5,0){x}(4,4){y}
    \pstLineAB{S}{x} \pstLineAB{S}{y}
    \pstBissectBAC[linecolor=black,PointSymbol=none]{x}{S}{y}{t}
    \pstBissectBAC[linecolor=black,PointSymbol=none]{t}{S}{y}{t_1}
    \pstBissectBAC[linecolor=black,PointSymbol=none]{t_1}{S}{y}{t_2}
    \pstBissectBAC[linecolor=black,PointSymbol=none]{x}{S}{t}{t_3}
    \pstBissectBAC[linecolor=black,PointSymbol=none]{x}{S}{t_3}{t_4}
\endpspicture
\end{center}

\columnbreak

Parmi les demi-droites tracées à l'intérieur de l'angle
$\widehat{xSy}$, quelle est la bissectrice de l'angle
$\widehat{xSy}$ ? \dotfill\\
\null \dotfill\\
\null \dotfill\\
Vérifier à l'aide du rapporteur et compléter :\\
\begin{myenumerate}
    \item La bissectrice de l'angle $\widehat{xSy}$ est la
    demi-droite \dotfill
    \item Donc, les mesures des angles \phantom{blablabla} et
    \phantom{blablabla} sont égales.
\end{myenumerate}
\end{multicols}