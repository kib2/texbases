%@metapost:6axesymetrieexo48.mp
%@Auteur:Véronique Glaçon\par
Complète les phrases suivantes en utilisant les mots proposés :
\\
\compo{1}{6axesymetrieexo48}{1}{
\hspace{1cm} \psshadowbox{\begin{tabular}{c} équidistants \\ \end{tabular}}
\hspace{2cm} \psshadowbox{\begin{tabular}{c} symétrie \\ \end{tabular}}
\hspace{2cm} \psshadowbox{\begin{tabular}{c} médiatrice \\ \end{tabular}}
\vspace{0.5cm} \\
\hspace{2cm} \psshadowbox{\begin{tabular}{c} axe \\ \end{tabular}}
\hspace{2cm} \psshadowbox{\begin{tabular}{c} perpendiculaire \\ \end{tabular}}
\hspace{2cm} \psshadowbox{\begin{tabular}{c} milieu \\ \end{tabular}}}
\\\vspace{1cm}
\begin{itemize}
	\item[$\bullet$] L'\ldots\ldots\ldots\ldots\ldots\ldots  de \ldots\ldots\ldots\ldots\ldots\ldots  du segment $[AB]$ est la \ldots\ldots \ldots\ldots\ldots\ldots de ce segment.
	\vspace{0.3cm}
	\item[$\bullet$] La \ldots\ldots\ldots\ldots\ldots\ldots  du
          segment $[AB]$ est \ldots\ldots \ldots\ldots\ldots\ldots à
          la droite $(AB)$ et coupe le segment $[AB]$ en son \ldots\ldots\ldots\ldots\ldots\ldots 
	\vspace{0.3cm}
	\item[$\bullet$]  Tous les points de la \ldots\ldots \ldots\ldots\ldots\ldots du segment $[AB]$ sont \ldots\ldots \ldots\ldots\ldots\ldots des points $A$ et $B$.
	\vspace{0.3cm}
	\item[$\bullet$] Tous les points \ldots\ldots \ldots\ldots\ldots\ldots des points $A$ et $B$ appartiennent à la \ldots\ldots \ldots\ldots\ldots\ldots du segment $[AB]$. 
\end{itemize}