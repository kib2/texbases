%@Auteur: François Meria\par
Compléter les phrases suivantes avec les mots proposés :\\

\fbox{médiatrice} \hfill \fbox{axe de symétrie} \hfill
\fbox{bissectrice} \hfill \fbox{angle}
\begin{enumerate}[(a)]
    \item La droite $(d)$ est l'\dotfill de l'\dotfill
    $\widehat{HAK}$.
    \item La droite $(d)$ est donc sa \dotfill
    \item La droite $(d)$ est aussi la \dotfill du segment $[HK]$.
\end{enumerate}
\begin{center}
\pspicture(5,5) \rput{25}{
    \pstGeonode[PointSymbol=none,PosAngle={235,-60,60}](0,1){A}(4,-1){K}(4,3){H}
    \pstLineAB{H}{K}
    \pstMediatorAB[CodeFig=true,PointName=none,PointSymbol=none,nodesep=-1.5,CodeFigColor=black]{H}{K}{I}{M}
    \pstLineAB[nodesepA=-1]{A}{I}
    \pstLineAB[nodesepB=-1]{A}{H}
    \pstMarkAngle[Mark=MarkHash,MarkAngleRadius=1]{I}{A}{H}{}
    \pstLineAB[nodesepB=-1]{A}{K}
    \pstMarkAngle[Mark=MarkHash,MarkAngleRadius=0.8]{K}{A}{I}{}
    }
    \put(5,3){$(d)$}
\endpspicture
\end{center}