%@P:exocorcp
%@Dif:2
Calcule les sommes suivantes :
\begin{myenumerate}
  \item en utilisant l'écriture décimale;
  \item en utilisant l'écriture fractionnaire.
\end{myenumerate}
\[\Eqalign{
A&=\frac3{10}+\frac5{10}\kern2cm&B&=\frac{48}{100}+\frac{91}{100}\cr
\cr
C&=\frac3{10}+\frac4{100}&D&=\frac{46}{10}+\frac{248}{100}\cr
}\]
%@Correction:
\begin{myenumerate}
  \item
\[\Eqalign{
A&=\frac3{10}+\frac5{10}\kern2cm&B&=\frac{48}{100}+\frac{91}{100}\kern2cm&C&=\frac3{10}+\frac4{100}&D&=\frac{46}{10}+\frac{248}{100}\cr
A&=0,3+0,5&B&=0,48+0,91&C&=0,3+0,04&D&=4,6+2,48\cr
A&=0,8&B&=1,39&C&=0,34&D&=7,08\cr
}\]
\item 
\[\Eqalign{
A&=\frac3{10}+\frac5{10}\kern2cm&B&=\frac{48}{100}+\frac{91}{100}\cr
A&=3\mbox{ dixièmes}+5\mbox{ dixièmes}&B&=48\mbox{ centièmes}+91\mbox{ centièmes}\cr
A&=8\mbox{ dixièmes}&B&=139\mbox{ centièmes}\cr
A&=\frac8{10}&B&=\frac{139}{100}\cr
\cr
C&=\frac3{10}+\frac4{100}&D&=\frac{46}{10}+\frac{248}{100}\cr
C&=3\mbox{ dixièmes}+4\mbox{ centièmes}&D&=46\mbox{ dixièmes}+248\mbox{ centièmes}\cr
C&=30\mbox{ centièmes}+4\mbox{ centièmes}&D&=460\mbox{ centièmes}+248\mbox{ centièmes}\cr
C&=34\mbox{ centièmes}&D&=708\mbox{ centièmes}\cr
C&=\frac{34}{100}&D&=\frac{708}{100}\
}\]
\end{myenumerate}
%@Commentaire: Manipulation de différentes écritures des nombres décimaux. Pour la 2\ieme\ question, on peut passer par la langue naturelle.