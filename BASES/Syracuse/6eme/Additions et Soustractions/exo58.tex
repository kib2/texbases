%@P:exocorcp
%@Dif:2
\QCM{3}{Addition et soustraction de nombres décimaux.}{
\hline
\multicolumn{1}{|c|}{\bf Question}&\multicolumn{1}{c|}{\bf Réponse A}&\multicolumn{1}{c|}{\bf Réponse B}&\multicolumn{1}{c|}{\bf Réponse C}\\
\hline
La somme $14,41+7,7$ vaut\ldots&\F{21,39}&\F{21,111}&\V{22,11}\\
La différence $41-9,25$ vaut\ldots&\F{32,25}&\F{32,75}&\V{31,75}\\
Pierre a 44 ans, il a 19 ans de plus que Paul. Pour trouver l'âge de Paul, il faut poser :&\F{une addition}&\V{une soustraction}&\F{Réponse impossible}\\
Quelle opération permet de calculer le terme inconnu dans $89+\square=121,5$&\V{$121,5-89$}&\F{$121,5+89$}&\F{$89-121,5$}\\
Voici un énoncé : \og{}{\em J'avais 9,45~\textgreek{\euro}, il me reste 4,15~\textgreek{\euro}. Combien ai-je dépensé ?}\fg{}\par Quelle égalité traduit cet énoncé ?&\V{$9,45-?=4,15$}&\F{$?-4,15=9,45$}&\F{$9,45+?=4,15$}\\
Quelle opération permet de calculer le terme inconnu dans $54,6-?=24$ ?&\V{$54,6-24$}&\F{$54,6+24$}&\F{$24-54,6$}\\
\og{}{\em Avec 13~cm de plus, ce sportif aurait égalé le record du monde de saut en hauteur : $2,48$~m.}\fg. Quelle égalité permet de calculer la hauteur du saut de ce sportif ?&\F{$?-2,43=0,13$}&\V{$?+0,13=2,48$}&\F{$?-0,13=2,48$}\\
Quelle opération permet de calculer le terme inconnu dans $\square-54,6=24$ ?&\F{$54,6-24$}&\V{$54,6+24$}&\F{$24-54,6$}\\
Au 3\ieme\ étage, 12 personnes entrent dans l'ascenseur et 8 personnes en sortent. Combien de personnes sont dans l'ascenseur après la fermeture ?&\F{$12+8$}&\F{$12-8$}&\V{Réponse impossible}\\
}
%@Correction:
\QCMcor{3}{Addition et soustraction de nombres décimaux.}{
\hline
\multicolumn{1}{|c|}{\bf Question}&\multicolumn{1}{c|}{\bf Réponse A}&\multicolumn{1}{c|}{\bf Réponse B}&\multicolumn{1}{c|}{\bf Réponse C}\\
\hline
La somme $14,41+7,7$ vaut\ldots&\F{21,39}&\F{21,111}&\V{22,11}\\
La différence $41-9,25$ vaut\ldots&\F{32,25}&\F{32,75}&\V{31,75}\\
Pierre a 44 ans, il a 19 ans de plus que Paul. Pour trouver l'âge de Paul, il faut poser :&\F{une addition}&\V{une soustraction}&\F{Réponse impossible}\\
Quelle opération permet de calculer le terme inconnu dans $89+?=121,5$&\V{$121,5-89$}&\F{$121,5+89$}&\F{$89-121,5$}\\
Voici un énoncé : \og{}{\em J'avais 9,45~\textgreek{\euro}, il me reste 4,15~\textgreek{\euro}. Combien ai-je dépensé ?}\fg{}\par Quelle égalité traduit cet énoncé ?&\V{$9,45-?=4,15$}&\F{$?-4,15=9,45$}&\F{$9,45+?=4,15$}\\
Quelle opération permet de calculer le terme inconnu dans $54,6-?=24$ ?&\V{$54,6-24$}&\F{$54,6+24$}&\F{$24-54,6$}\\
\og{}{\em Avec 13~cm de plus, ce sportif aurait égalé le record du monde de saut en hauteur : $2,48$~m.}\fg. Quelle égalité permet de calculer la hauteur du saut de ce sportif ?&\F{$?-2,43=0,13$}&\V{$?+0,13=2,48$}&\F{$?-0,13=2,48$}\\
Quelle opération permet de calculer le terme inconnu dans $?-54,6=24$ ?&\F{$54,6-24$}&\V{$54,6+24$}&\F{$24-54,6$}\\
Au 3\ieme\ étage, 12 personnes entrent dans l'ascenseur et 8 personnes en sortent. Combien de personnes sont dans l'ascenseur après la fermeture ?&\F{$12+8$}&\F{$12-8$}&\V{Réponse impossible}\\
}
%@Commentaire: Un QCM. Travail surtout sur les équations relatives à l'addition et à la soustraction.