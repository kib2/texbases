%@P:exocorcp
%@Dif:2
Morgane pose dans son chariot 4 litres de jus de raisin coûtant
3,48~\textgreek{\euro}, un rôti de porc coûtant
6,83~\textgreek{\euro}, du fromage coûtant 1,96~\textgreek{\euro} et
un gâteau coûtant 4,70~\textgreek{\euro}. Elle souhaite aussi
acheter une tablette de chocolat à 1,22~\textgreek{\euro}. Elle dispose
de 20~\textgreek{\euro}.\\En prenant un ordre de grandeur pour chaque
prix, elle vérifie si cela est possible.
\\En faisant mentalement les mêmes calculs que Morgane, dis si elle
peut ou non acheter la tablette de chocolat.
%@Correction:
$\left.\begin{array}{l}
3,48~\mbox{\textgreek{\euro}}\longrightarrow3,5~\mbox{\textgreek{\euro}}\\
6,83~\mbox{\textgreek{\euro}}\longrightarrow7~\mbox{\textgreek{\euro}}\\
1,96~\mbox{\textgreek{\euro}}\longrightarrow2~\mbox{\textgreek{\euro}}\\
4,70~\mbox{\textgreek{\euro}}\longrightarrow4,70~\mbox{\textgreek{\euro}}\\
1,22~\mbox{\textgreek{\euro}}\longrightarrow1,30~\mbox{\textgreek{\euro}}\\
\end{array}
\right\}3,5+7+2+4,70+1,30=3,5+9+6=18,5~\mbox{\textgreek{\euro}}$
\par Morgane aura donc assez de ses 20~\textgreek{\euro}.
%@Commentaire: Utilisation des ordres de grandeurs dans un contexte familier pour l'élève. Doit être fait prioritairement à l'aide du calcul mental.