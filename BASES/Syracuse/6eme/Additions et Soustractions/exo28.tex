%@P:exocorcp
%@Dif:2
{\em Sans calculer les sommes}, pour chaque calcul, choisis parmi 60; 80;
100; 150 et 200 le nombre le plus proche du résultat :
\begin{multicols}{3}
\begin{myenumerate}
\item $37,25+119,40$
\item $28,73+32,29$
\item $120,28+69,2$
\item $14,96+63,7$
\item $27,89+119,28$
\item $118,79+79,19$
\end{myenumerate}
\end{multicols}
%@Correction:
\begin{multicols}{3}
  \begin{myenumerate}
    \item $40+120=160$ donc 150.
    \item $30+30=60$ donc 60.
    \item $120+70=190$ donc 200.
    \item $15+65=80$ donc 80.
    \item $30+120=150$ donc 150.
    \item $120+80=200$ donc 200
  \end{myenumerate}
\end{multicols}
%@Commentaire:
Travail sur les ordres de grandeurs.