%@P:exocorcp
%@Dif:2
Un quadrilatère a pour périmètre 450~m.\\Le premier côté mesure 60~m, le deuxième 110~m. La longueur du troisième côté est la somme de celles du premier et du second.\\Calcule la longueur du quatrième côté.
%@Correction:
\begin{multicols}{3}
\opadd*{60}{110}{a}\opadd{60}{110}
\par Le 3\ieme\ côté mesure \opprint{a}~m.\par
\opadd*{a}{a}{b}\opadd{a}{a}
\par Il y a déjà \opprint{b}~m.
\par
\opsub*{450}{b}{c}\opsub{450}{b}
\par le 4\ieme\ côté mesure \opprint{c}~m.
\end{multicols}
%@Commentaire: On peut le donner en début d'année pour vérifier les acquis sur les périmètres, ou au début du chapitre sur les périmètres ou en interrogation après avoir fait ce cours.