%@P:exocorcp
%@Dif:2
Antoine veut s'offrir un lecteur de CD qui coûte 110~\textgreek{\euro}
et un CD de son chanteur préféré. Ce CD vaut
18,75~\textgreek{\euro}. Antoine réunit ses économies et constate
qu'elles se montent à 53,35~\textgreek{\euro}. Son père participe à son
achat en lui donnant 15~\textgreek{\euro}.
\\Antoine envisage alors de retirer l'argent qui lui manque de son
livret de Caisse d'Epargne. Sur ce livret, il dispose de
191,73~\textgreek{\euro}.
\begin{myenumerate}
\item Quelle somme Antoine doit-il retirer de son livret de Caisse
  d'Epargne ?
\item Quel montant lui restera-t-il alors sur son livret ?
\end{myenumerate}
%@Correction:
\begin{myenumerate}
  \item\subitem{}\par
\begin{multicols}{3}
\opadd*{110}{18,75}{a}\opadd{110}{18,75}
\par Antoine a besoin de \opprint{a}~\textgreek{\euro}.
\par
\opadd*{53,35}{15}{b}\opadd{53,35}{15}
\par Antoine dispose de \opprint{b}~\textgreek{\euro}.
\par
\opsub*{a}{b}{c}\opsub[carrysub=true]{a}{b}
\par Il manque \opprint{c}~\textgreek{\euro} à Antoine.
\end{multicols}
\item\subitem{}\par
\opsub*{191,73}{c}{d}\opsub[carrysub=true]{191,73}{c}\kern2cm Il restera \opprint{d}~\textgreek{\euro} sur son livret.
\end{myenumerate}
%@Commentaire: Encore une situation concrète. Le sens des opérations est une nouvelle fois privilégié.