%@P:exocorcp
%@Dif:2
Pierre refait ses comptes : il a acheté un rôti à
15,08~\textgreek{\euro}; un coquelet à 3,25~\textgreek{\euro};
 un lapin et un demi-agneau. Il a dépensé en tout
 51,37~\textgreek{\euro}. Le lapin coûtait 4,5~\textgreek{\euro}
 de plus que le coquelet.
\par Calcule le prix du demi-agneau.
%@Correction:
\par\opadd*{3,25}{4,5}{a}\opadd{3,25}{4,5}\kern2cm Le prix du lapin est \opprint{a}~\textgreek{\euro}.
\par\opadd*{15,08}{3,25}{b}\opadd{15,08}{3,25}\kern2cm\opadd*{b}{a}{c}\opadd{b}{a}\kern2cm Le total des courses sans le demi-agneau est de \opprint{c}~\textgreek{\euro}.
\par
\opsub*{51,37}{c}{d}\opsub{51,37}{c}\kern2cm Le prix du demi-agneau est de \opprint{d}~\textgreek{\euro}.
%@Commentaire: Exercice facile, concret pour les élèves (c'est une situation habituelle).