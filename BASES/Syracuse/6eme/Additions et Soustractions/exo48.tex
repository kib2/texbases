%@P:exocorcp
%@Dif:2
Calcule les différences suivantes :
\begin{myenumerate}
  \item En utilisant l'écriture décimale.
  \item En utilisant l'écriture fractionnaire.
\end{myenumerate}
\[\Eqalign{
A&=\frac{29}{10}-\frac{11}{10}\kern2cm&B&=\frac{42}{100}-\frac2{100}\cr
\cr
C&=\frac5{10}-\frac{29}{100}&D&=\frac{428}{100}-\frac{12}{10}\cr
}\]
%@Correction:
\begin{myenumerate}
  \item
\[\Eqalign{
A&=\frac{29}{10}-\frac{11}{10}\kern2cm&B&=\frac{42}{100}-\frac2{100}\kern2cm&C&=\frac5{10}-\frac{29}{100}\kern2cm&D&=\frac{428}{100}-\frac{12}{10}\cr
A&=2,9-1,1&B&=0,42-0,02&C&=0,5-0,29&D&=4,28-1,2\cr
A&=1,8&B&=0,4&C&=0,21&D&=3,08\cr
}\]
\item 
\[\Eqalign{
A&=\frac{29}{10}-\frac{11}{10}\kern2cm&B&=\frac{42}{100}-\frac2{100}\cr
A&=29\mbox{ dixièmes}-11\mbox{ dixièmes}&B&=42\mbox{ centièmes}-2\mbox{ centièmes}\cr
A&=18\mbox{ dixièmes}&B&=40\mbox{ centièmes}\cr
A&=\frac{18}{10}&B&=\frac{40}{100}\cr
\cr
C&=\frac5{10}-\frac{29}{100}&D&=\frac{428}{100}-\frac{12}{10}\cr
C&=5\mbox{ dixièmes}-29\mbox{ centièmes}&D&=428\mbox{ centièmes}-12\mbox{ dixièmes}\cr
C&=50\mbox{ centièmes}-29\mbox{ centièmes}&D&=428\mbox{ centièmes}-120\mbox{ centièmes}\cr
C&=21\mbox{ centièmes}&D&=308\mbox{ centièmes}\cr
C&=\frac{21}{100}&D&=\frac{308}{100}\cr
}\]
\end{myenumerate}
%@Commentaire: Manipulation de différentes écritures des nombres décimaux.