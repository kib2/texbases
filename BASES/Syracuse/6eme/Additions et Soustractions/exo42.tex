%@P:exocorcp
%@Dif:3
Un croissant et un éclair coûtent ensemble 1,60~\textgreek{\euro}.\\Deux croissants et un éclair coûtent ensemble 2,30~\textgreek{\euro}.
\begin{myenumerate}
  \item Combien coûte un croissant ?
  \item Combien coûte un éclair ?
\end{myenumerate}
%@Commentaire: Problème intéressant où les élèves ont l'impression de manquer d'informations.
%@Correction:
\begin{myenumerate}
\item La différence entre les deux achats est :
  \begin{itemize}
  \item 1 croissant
  \item ou 0,70~\textgreek{\euro}.
  \end{itemize}
Donc 1 croissant coûte 0,70~\textgreek{\euro}.
\item Avec la 1\iere\ information, on peut écrire :
\[\Eqalign{
0,70+\mbox{1 éclair}&=1,60\cr
\mbox{1 éclair}&=1,60-0,70\cr
\mbox{1 éclair}&=0,90\cr
}\]
Un éclair coûte 0,90~\textgreek{\euro}.
\end{myenumerate}
\par
\vspace{2mm}
{\em Complément} : On peut vérifier que le prix du 2\ieme\ achat est correct.