%@P:exocorcp
%@Dif:2
Voici la table de multiplication de 78 :
\begin{center}
\begin{tabular}{|c|c|c|}
\hline
$1\times78=78$&$4\times78=312$&$7\times78=546$\\
\hline
$2\times78=156$&$5\times78=390$&$8\times78=624$\\
\hline
$3\times78=234$&$6\times78=468$&$9\times78=702$\\
\hline
\end{tabular}
\end{center}
\par \`A l'aide de cette table, donne une valeur approchée au centième
du quotient de 45\,017 par 78.
%@Correction:
\opdiv*{45017}{78}{q}{r}\opround{q}{2}{q}\opdiv[maxdivstep=6]{45017}{78}
\par Donc une valeur approchée au centième du quotient est \opprint{q}.
%@Commentaire: On cherche à vérifier si la technique de calcul est maîtrisée.