%@P:exocorcp
%@Dif:4
%@Auteur: Rallye Mathématique CM2 - 6\ieme\ Poitou- Charentes
Au départ d'une épreuve de 20~km de marche, il y avait 35 concurrents.\\Pendant, l'épreuve, les juges, qui sont très sévères, ont distribué 82 avertissements pour marche irrégulière. On sait qu'au troisième avertissement un marcheur est éliminé.\\Annabelle affirme que 10 marcheurs ont été éliminés.\\Tom certifie qu'il a vu les juges en éliminer 30 !\\Grégoire dit qu'il a assisté à l'élimination de 23 marcheurs.
\par Il y a au moins deux menteurs ! Lesquels ? Pourquoi ?
%@Correction:
Pour chaque affirmation, il faut compter le nombres d'avertissements distribués.
\begin{description}
\item[Annabelle] 10 marcheurs ont été éliminés donc cela fait 30 avertissements. Il reste 25 marcheurs. Ils ont pu avoir au maximum 2 avertissements, soit 50 avertissements. Soit un total de 80 avertissements. Ce qui est impossible.
\item[Tom] 30 marcheurs ont été éliminés donc 90 avertissements. Ce qui est impossible.
\item[Grégoire] 23 marcheurs éliminés soit 69 avertissements. Il reste 12 concurrents. C'est possible d'atteindre 82 avertissements avec ces 12 concurrents. Mais c'est possible de ne pas l'atteindre.
\end{description}
%@Commentaire: Utilisation de la division euclidienne dans un contexte vraiment différent.