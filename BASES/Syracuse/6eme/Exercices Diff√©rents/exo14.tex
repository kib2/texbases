%@Titre: La planète Codus.
%@Auteur: Académie d'Orléans (?)\par
L'agent Codus est un super agent de contrôle dont les missions
consistent à vérifier la codification des codes barres. Sur la planète
Codus, les codes barres ont 5 chiffres.
\par\'Etant débordé dans son travail, notre super héros fait appel à
tes services. Dès ton arrivée, il t'explique comment contrôler un code
barre sur une boîte de conserve qui fait fureur sur la planète Codus :
{\em Le ragoût de lézard}.
Voici le numéro du code barre de la boîte de conserve : 2 587 4.
\par L'agent Codus place ce code dans son super tableau d'analyse.
\par\vspace{2mm}\par\dispo{1}{
  \begin{tabular}{|c|c|c|c|c|}
    \hline
    Pays&\multicolumn{3}{c|}{Produit}&Clé\\
    \hline
    2&5&8&7&4\\
    \hline
    $P$&$A$&$B$&$C$&$K$\\
    \hline
  \end{tabular}
}{Pour vérifier que la clé est correcte, utilise la formule magique
\[F=P+(3\times A)+B+(3\times C)+K\]
Si $F$ est un multiple de 10 alors la clé est correcte et le produit peut-être mis en vente.
}
\par\vspace{2mm}\par
Pour la boîte de lézard, on a $F=2+(3\times5)+8+(3\times7)+4=50$. {\bf
  Comme 50 est un multiple de 10, donc la clé est juste et le
  code-barre est accepté. Dans ce cas, le produit a le droit d'être
  mis en vente.}
\begin{description}
  \item[Formation] On craint une erreur sur les bouteilles de lait de mammouth ! Le code-barre est 6 281 8. Est-il correct ?
\par
\dispo{1}{
  \begin{tabular}{|c|c|c|c|c|}
    \hline
    Pays&\multicolumn{3}{c|}{Produit}&Clé\\
    \hline
    &&&&\\
    \hline
    $P$&$A$&$B$&$C$&$K$\\
    \hline
  \end{tabular}
}{\[\Eqalign{
F&=P+(3\times A)+B+(3\times C)+K\cr
F&=\ldots+(3\times\ldots)+\ldots+(3\times\ldots)+\ldots\cr
F&=\ldots\cr
}\]}
\item[Mission \no 1] L'agent Codus m'emmène avec lui au
  SUPERMARCHIX. Le directeur nous demande de détecter les codes faux
  puis de changer la clé pour qu'ils deviennent corrects.
  \begin{itemize}
  \item Shampooing anti-tique : 4 785 7
  \item Confiture de limaces : 0 147 2
  \item Bonbons Codus : 7 143 9
  \end{itemize}
\item[Mission \no2] \`A peine fini, que le directeur arrive en criant
  : \og notre ordinateur a un virus. Il ne marque plus la clé. Nous
  courons à la ruine. Faites quelque chose !\fg.
  \begin{itemize}
  \item Pomme animée : 1 547 ?
  \item Balai magique : 9 704 ?
  \item Lampe automatique : 2 036 ?
  \end{itemize}
\item[Mission \no3] Le directeur vérifie que les caissières frappent
  bien le code 8 167 2. Trois d'entre elles se sont trompées et le
  directeur demande à Codus de vérifier si pour chacune d'elles,
  l'ordinateur détecte l'erreur.
  \begin{itemize}
  \item Hôtesse \no1 : 8 167 3
  \item Hôtesse \no2 : 8 176 2
  \item Hôtesse \no3 : 8 617 2
  \end{itemize}
\end{description}