%@metapost:octaedrecolore.mp
%@Auteur:d'après {\sl www.mathsenzep.com}\par
Le travail consiste à construire un patron qui représentera un assemblage de
tétraèdres : un rouge et un jaune.
\\Tous les triangles sont des {\em triangles équilatéraux} qui ont
pour côté 4~cm.
\\La construction doit être très précise, sinon l'assemblage du patron
ne sera pas satisfaisant.
\paragraph{Programme de tracé}\hfill\newline
\begin{itemize}
\item Sur la feuille non quadrillée fournie, prise en format portrait,
  trace une droite $(d)$ parallèle au bas de la feuille à 8~cm de
  distance. Sur cette droite placer un point $A$ à 0,5~cm du bord
  gauche.
\item Place un point $B$ sur la droite $(d)$ à 4~cm de $A$. Trace le
  triangle équilatéral $ABC$ (au dessus de la droite).
\item Trace un triangle équilatéral $BCD$ (qui ne se superpose pas à
  $ABC$).
\item Trace {\em côte à côte} les triangles équilatéraux $BDE$, puis
  $DEF$, puis $EFG$, puis $FGH$, puis $GHI$, puis $HIJ$ et enfin
  $IJK$.
\item Colorie les triangles $ABC$, $BCD$, $BDE$ puis $GHI$, $HIJ$ et
  $IJK$ en jaune. Colorie les trois autres en rouge.
\item Sous le triangle $EFG$, trace un autre triangle équilatéral
  $EGL$. Puis un triangle équilatéral $ELM$ et enfin un triangle
  équilatéral $MLN$. Colorie ces trois triangles en jaune.
\item Au dessus du triangle $HIJ$, construire des triangles
  équilatéraux $HJO$, puis $OHP$ et $OPQ$. Colorie ces trois triangles
  en rouge.
\item Trace des triangles équilatéraux $PQR$ puis $QRS$ et
  $SRT$. Colorie ces trois triangles en jaune.
\item Trace des triangles équilatéraux $SQU$, puis $SUV$ et
  $UVW$. Colorie ces trois triangles en rouge.
\item Trace des triangles équilatéraux $STX$, puis $TXY$ et enfin
  $YXZ$. Colorie ces trois triangles en rouge.
\end{itemize}
\paragraph{Assemblage du patron}\hfill\newline
\begin{itemize}
\item Construis des pattes de collage sur les segments $[AB]$, $[AC]$,
  $[CD]$, $[DF]$, $[ME]$, $[MN]$, $[NL]$, $[IK]$, $[JK]$, $[OQ]$,
  $[PR]$, $[UW]$ et $[ZX]$.
\item Découpe le patron.
\item Plie les lignes entre les triangles de même couleur vers
  l'arrière.
\item Plie les lignes entre les triangles de couleurs différentes vers l'avant (les deux couleurs doivent être l'une contre l'autre).
\item Assembler le patron.
\end{itemize}
\compo{1}{octaedrecolore}{0.5,angle=90}{
{\em La figure n'est pas à l'échelle et elle est tournée d'un quart de
  tour.\par Les triangles \ding{172}
  sont en jaune; les triangles \ding{173} sont en rouge.}}