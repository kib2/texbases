%@Auteur: d'après IREM Strasbourg
%@Dif:2
\begin{myenumerate}
\item Recopie et complète les schémas de calculs suivants :
\par\vspace{1cm}\par
\begin{center}
  \schop{4,5}{$\div\ldots$}{\ldots}{$\div10$}{\ldots}{$\div10$}\kern2cm\schop{0,7}{$\div1\,000$}{\ldots}{$\div100$}{\ldots}{\ldots}
\par\vspace{2cm}\par
\schop{0,7}{$\div\ldots$}{\ldots}{$\div\ldots$}{0,07}{$\div100$}\kern2cm\schop{830}{$\div\ldots$}{8,3}{$\div10$}{\ldots}{$\div\ldots$}
\par\vspace{2cm}\par
\schop{\ldots}{$\div1\,000$}{0,007\,5}{$\div100$}{\ldots}{$\div\ldots$}\kern2cm\schop{6,9}{$\times\ldots$}{\ldots}{$\times100$}{\ldots}{$\div10$}
\par\vspace{2cm}\par
\schop{8,5}{$\div10$}{\ldots}{$\div100$}{\ldots}{$\times\ldots$}\kern2cm\schop{\ldots}{$\div10$}{89}{$\times\ldots$}{\ldots}{$\div1\,000$}
\end{center}
\par\vspace{1cm}\par
\item Complète les phrases ci-dessous après avoir observé les schémas ci-dessus :
\begin{itemize}
\item Diviser par 10, puis diviser par 10 revient à \dotfill
\item Diviser par 100, puis diviser par 10 revient à \dotfill
\item Diviser par 10, puis diviser par 100 revient à \dotfill
\item Multiplier par 100, puis diviser par 10 revient à \dotfill
\item Diviser par 100, puis multiplier par 10 revient à \dotfill
\end{itemize}
\end{myenumerate}