%@P:exocorcp
%@Dif:3
Un commerçant achète 12 cageots d'oranges. Chaque cageot contient
5,5~kg d'oranges et coûte 6,71~\textgreek{\euro}.
\\Le commerçant revend les oranges 1,91~\textgreek{\euro} le
kilogramme.
\par Quel bénéfice fera-t-il s'il réussit à tout vendre ?
%@Correction:
\begin{center}
\begin{tabular}{c|l}
Opérations&\multicolumn{1}{c}{Solutions}\\
\hline
\opmul*{12}{5,5}{a}\opmul*{12}{6,71}{b}\opmul{12}{5,5}\kern1cm\opmul{12}{6,71}&Il a acheté \opprint{a}~kg d'oranges et a payé \opprint{b}~\textgreek{\euro}.\\
&\\
\opmul*{1,91}{a}{c}\opmul{1,91}{a}&S'il réussi à tout vendre, il récupérera \opprint{c}~\textgreek{\euro}.\\
&\\
\opsub*{c}{b}{d}\opsub{c}{b}&Son bénéfice est de \opprint{d}~\textgreek{\euro}.\\
\end{tabular}
\end{center}
%@Commentaire: Que veut dire le mot {\em bénéfice} ? Il y a plusieurs étapes de calculs.