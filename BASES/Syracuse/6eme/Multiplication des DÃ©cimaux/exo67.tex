%@Auteur: Nathalie Lespinasse\par
{\em Lis attentivement le texte suivant :}
\begin{quote}
Quatre copains préparent un repas. Ils vont au supermarché avec un
billet de 20~\textgreek{\euro}. Ils achètent quatre pains à
0,60~\textgreek{\euro} chacun, trois boîtes de salade mexicaine à
1,35~\textgreek{\euro} pièce, un rôti de dinde à 7,5~\textgreek{\euro}
et une grande bouteille de jus de fruits de deux litres à
1,46~\textgreek{\euro} le litre.
\end{quote}
\begin{myenumerate}
  \item
    \begin{enumerate}
    \item Recopie la partie du texte qui parle des pains :
\par\dotfill\par\dotfill
\item Calcule le prix des pains :
\par\dotfill\par\dotfill
\par Les pains coûtent \dotfill
    \end{enumerate}
  \item
    \begin{enumerate}
    \item  Recopie la partie du texte qui parle des salades :
\par\dotfill\par\dotfill
\item Calcule le prix des salades :
\par\dotfill\par\dotfill
    \end{enumerate}
  \item
    \begin{enumerate}
    \item Recopie la partie du texte qui parle du jus de fruits :
\par\dotfill\par\dotfill
\item Calcule le prix du jus de fruits :
\par\dotfill\par\dotfill
    \end{enumerate}
  \item Calcule le prix total à payer :
\par\dotfill\par\dotfill
\item Calcule la somme rendue :
\par\dotfill\par\dotfill
\end{myenumerate}