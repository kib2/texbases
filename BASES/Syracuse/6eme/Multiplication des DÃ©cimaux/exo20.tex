%@P:exocorcp
%@Dif:2
\opset{voperator=bottom}
Recopie et complète les multiplications suivantes :
\\
\opmul[displayshiftintermediary=all,intermediarystyle.1=\troudot,
intermediarystyle.2=\troudot,intermediarystyle.3=\troudot,
resultstyle=\troudot]{23}{153}\hfill
\opmul[displayintermediary=all,displayshiftintermediary=all,
intermediarystyle.1=\troudot,intermediarystyle.2=\troudot,
intermediarystyle.3=\troudot,resultstyle=\troudot]{37}{105}\hfill
\opmul[displayshiftintermediary=all,intermediarystyle.1=\troudot,
intermediarystyle.2=\troudot,intermediarystyle.3=\troudot,
resultstyle=\troudot]{78}{936}\hfill
\opmul[displayshiftintermediary=all,intermediarystyle.1=\troudot,
intermediarystyle.2=\troudot,intermediarystyle.3=\troudot,
resultstyle=\troudot]{98}{246}
%@Correction:
\opmul{23}{153}\hfill
\opmul{37}{105}\hfill
\opmul{78}{936}\hfill
\opmul{98}{246}
%@Commentaire: Chaque calcul est décomposé. Cet exercice permet de détailler la méthode pour effectuer la multiplication.