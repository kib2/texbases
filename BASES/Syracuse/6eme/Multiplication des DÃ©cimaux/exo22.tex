%@P:exocorcp
%@Dif:3
\opset{voperator=bottom,decimalsepsymbol={,}}
Recopie et complète les multiplications suivantes :
\\\opmul[displayshiftintermediary=all,intermediarystyle.1=\troudot,
intermediarystyle.2=\troudot,resultstyle.3=\troudot,
resultstyle.2=\troudot,resultstyle.1=\troudot,resultstyle.-1=\troudot]
{23}{4,5}\hfill
\opmul[displayshiftintermediary=all,intermediarystyle.1=\troudot,
intermediarystyle.2=\troudot,intermediarystyle.3=\troudot,
resultstyle.3=\troudot,resultstyle.2=\troudot,resultstyle.1=\troudot,
resultstyle.-1=\troudot,resultstyle.-2=\troudot]{34}{3,59}\hfill
\opmul[displayshiftintermediary=all,intermediarystyle.1=\troudot,
intermediarystyle.2=\troudot,resultstyle.3=\troudot,
resultstyle.2=\troudot,resultstyle.1=\troudot,resultstyle.-1=\troudot]{78}{8,9}\hfill
\opmul[displayshiftintermediary=all,intermediarystyle.1=\troudot,
intermediarystyle.2=\troudot,intermediarystyle.3=\troudot,
resultstyle.3=\troudot,resultstyle.2=\troudot,resultstyle.1=\troudot,
resultstyle.-1=\troudot,resultstyle.-2=\troudot]{89}{5,17}
%@Correction:
\opset{voperator=bottom,decimalsepsymbol={,}}
\opmul{23}{4,5}\hfill
\opmul{34}{3,59}\hfill
\opmul{78}{8,9}\hfill
\opmul{89}{5,17}
%@Commentaire: La méthode pour obtenir le produit d'un entier par un nombre décimal est ici rappelée.