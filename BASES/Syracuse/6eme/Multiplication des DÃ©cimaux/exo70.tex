%@Auteur: D'après IREM Strasbourg\par
\begin{myenumerate}
  \item\label{6mulexo70q1} Recopie et complète :
    \begin{itemize}
    \item 37 fois 1 dixième égale \dotfill
    \item 45 fois 1 millième égale \dotfill
    \item 312 fois 1 centième égale \dotfill
    \item 19,3 fois 1 millième égale \dotfill
    \end{itemize}
  \item Traduis chacune des phrases ci-dessous par une multiplication.
  \item Pour chacune des phrases de la question \ref{6mulexo70q1}
, écris un schéma du type
\vspace{1cm}
\begin{center}
  \rnode{A}{37}\kern6cm\rnode{B}{\raisebox{6pt}{\hbox{}}}
\ncbar[angleA=90,nodesepA=3mm,nodesepB=3mm]{->}{A}{B}
\ncput*{$\times\dots$}
\ncline[nodesep=3mm]{->}{A}{B}
\ncput*{\small multiplié par 1 dixième}
\ncbar[angleA=-90,nodesepA=3mm,nodesepB=3mm]{->}{A}{B}
\ncput*{$\div$\ldots}
\end{center}
\vspace{5mm}
\item Recopie et complète pour chaque phrase de la question
  \ref{6mulexo70q1}
, la synthèse suivante :
\begin{center}
\psshadowbox{
  \begin{tabular}{lc}
    37 fois 1 dixième&\hbox to5cm{\dotfill}\\
    \\
    $37\times\ldots$&\hbox to5cm{\dotfill}\\
    \\
    $37\div\ldots$&\hbox to5cm{\dotfill}\\
    \\
    $37\times\dfrac{\ldots}{\ldots}$&\hbox to5cm{\dotfill}\\
  \end{tabular}
}
\end{center}
\end{myenumerate}
