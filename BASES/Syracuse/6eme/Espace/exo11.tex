%@metapost:perspective.mp
%@Titre: Perspective cavalière.
%@Dif:2
\compo{1}{perspective}{1}{Ce dessin représente un cube d'arête 6~cm
dessiné en {\bf perspective cavalière}.
\par La {\bf perspective cavalière} est une méthode mathématique pour
dessiner des solides.
\begin{quote}
\shadowbox{{\bf Règle \no1} : Les arêtes cachées sont dessinées en
pointillés.}
\end{quote}
Voici d'autres solides reproduis en utilisant la perspective cavalière.
}
\par
\includegraphics[scale=0.7]{perspective.2}\hfill\includegraphics[scale=0.7]{perspective.3}\hfill\includegraphics[scale=0.5]{perspective.4}\hfill\includegraphics[scale=0.6]{perspective.5}
\par
\begin{myenumerate}
\item Complète chacune des phrases suivantes:
\par{\em Dans la réalité, les six faces du cube sont des
\ldots\ldots\ldots\ldots de côté \ldots\ldots.
\par Alors que, sur le dessin, seules les faces $AEFB$ et
\ldots\ldots\ldots sont des \ldots\ldots\ldots\ldots.}
\item Les segments proposés ont-ils la même longueur ? Réponds par
{\bf Vrai} ou {\bf Faux}.
\par
\begin{tabular}{|c|c|c|c|c|}
\hline
{\em Dans la réalité}&\strut$[BF]$&$[BC]$&$[DC]$&$[EB]$ \\
\hline
\strut$[AB]$&&&&\\
\hline
\strut$[FC]$&&&&\\
\hline
\strut$[AD]$&&&&\\
\hline
\end{tabular}
\hfill
\begin{tabular}{|c|c|c|c|c|}
\hline
{\em Sur le dessin}&$[BF]$&$[BC]$&$[DC]$&$[EB]$ \\
\hline
$[AB]$&&&&\\
\hline
$[FC]$&&&&\\
\hline
$[AD]$&&&&\\
\hline
\end{tabular}
\item Complète le tableau suivant par {\bf Vrai} ou {\bf Faux}.
\begin{center}
\begin{tabular}{|c|c|c|}
\cline{2-3}
\multicolumn{1}{c|}{}&{\em Dans la réalité}&{\em Sur le dessin}\\
\hline
Les droites $(AB)$ et $(BC)$ sont perpendiculaires&&\\
\hline
Les droites $(AB)$ et $(BF)$ sont perpendiculaires&&\\
\hline
Les droites $(HD)$ et $(DC)$ sont perpendiculaires&&\\
\hline
Les droites $(AD)$ et $(BC)$ sont parallèles&&\\
\hline
Les droites $(AB)$ et $(DC)$ sont parallèles&&\\
\hline
Les droites $(AB)$ et $(EF)$ sont parallèles&&\\
\hline
\end{tabular}
\end{center}
\par Que remarque-t-on ?\dotfill\par\dotfill
\begin{center}
\shadowbox{{\bf Règle \no2} : Les
droites\ldots\ldots\ldots\ldots\ldots dans la réalité sont
représentées par des droites \ldots\ldots\ldots\ldots\ldots.}
\end{center}
\end{myenumerate}
%@Commentaire: Activité d'introduction en géométrie dans l'espace. Travail sur la perspective cavalière.