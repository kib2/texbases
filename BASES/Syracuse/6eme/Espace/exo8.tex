%@metapost:geomespace.mp
%@Dif:2
Dans cet exercice, le cube de référence est celui-ci : \includegraphics[scale=0.5]{geomespace.0}.
\begin{myenumerate}
\item Dans chacune des figures suivantes, indique le nombre de cubes que l'on a enlevés et le nombre de cubes restants.
\[\begin{tabular}{|c|c|c|c|}
\cline{2-4}
\multicolumn{1}{c|}{}&\includegraphics[scale=0.5]{geomespace.1}&\includegraphics[scale=0.5]{geomespace.2}&\includegraphics[scale=0.5]{geomespace.3}\\
\hline
Solide&${\cal S}_1$&${\cal S}_2$&${\cal S}_3$\\
\hline
cubes manquants&&&\\
\hline
cubes restants&&&\\
\hline
\end{tabular}
\]
\item Dans chacune des figures suivantes, indique le nombre de cubes que l'on a enlevés et le nombre de cubes restants.
\[\begin{tabular}{|c|c|c|c|}
\cline{2-4}
\multicolumn{1}{c|}{}&\includegraphics[scale=0.5]{geomespace.4}&\includegraphics[scale=0.5]{geomespace.5}&\includegraphics[scale=0.5]{geomespace.6}\\
\hline
Solide&${\cal S}_4$&${\cal S}_5$&${\cal S}_6$\\
\hline
cubes manquants&&&\\
\hline
cubes restants&&&\\
\hline
\end{tabular}
\]
\end{myenumerate}
%@Commentaire: On introduit la notion de volume.