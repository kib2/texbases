%@metapost:31dm04.mp
%@Auteur: Régis Leclercq
%@Dif:2
Voici un parallélépipède rectangle $ABCDEFGH$ dessiné en perspective
cavalière. Les questions posées, sauf mention spéciale, concernent le
pavé droit réel.
\par\compo{1}{31dm04}{1}{
\begin{myenumerate}
\item Nomme deux arêtes vues.
\item Nomme deux arêtes cachées.
\item Les droites $(AB)$ et $(AF)$ sont-elles perpendiculaires?
\item Cite une droite perpendiculaire à la droite $(AC)$.
\item Les droites $(AB)$ et $(DE)$ sont orthogonales sans être
  perpendiculaires. Cite une autre droite orthogonale à la droite
  $(AB)$ qui ne lui soit pas perpendiculaire.
\item Sur le dessin, quelle est la nature du quadrilatère $ABGF$?
  Quelle est la nature de ce même quadrilatère dans la réalité?
\end{myenumerate}
}
%@Commentaire: Travail sur la perspective cavalière et la réalité. La définition d'{\em orthogonale} doit avoir été expliquée avant.