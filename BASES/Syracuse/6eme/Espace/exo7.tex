%@metapost:ombre.mp
%@Titre: Ombres
%@Auteur: APMEP
%@Dif:3
Sur la figure ci-dessous, on a éclairé de trois façons différentes la
même pièce :
\par
\Compo{5}{ombre.10}{0.75}{
\begin{itemize}
\item en éclairant par devant, on a obtenu l'ombre de derrière;
\item en éclairant par la droite, on a obtenu l'ombre de gauche;
\item en éclairant par le dessus, on a obtenu l'ombre de dessous.
\end{itemize}
On a dessiné les trois ombres que l'on a obtenues puis on les a
comparé aux huit modèles d'ombres (appelés $m$, $n$, $o$, $p$, $q$,
$r$, $s$ et $t$) ci-dessous.
\begin{center}
\includegraphics[scale=0.85]{ombre.11}
\end{center}}
Les ombres obtenues sont des modèles $m$ et $r$ : on complète alors le
tableau
\[\begin{tabular}{|c|c|c|c|c|c|c|c|c|c|}
\hline
Ombre de&Pièce 1&Pièce 2&Pièce 3&Pièce 4&Pièce 5&Pièce 6&Pièce 7&Pièce 8&Pièce 9\\
\hline
derrière&$m$&&&&&&&&\\
\hline
gauche&$r$&&&&&&&&\\
\hline
dessous&$m$&&&&&&&&\\
\hline
\end{tabular}
\]
Suivant les mêmes éclairages, quels sont les modèles d'ombres obtenus
pour les huit solides suivants ? On indiquera les réponses dans le
tableau ci-dessus.
\begin{center}
\begin{tabular}{cccc}
\begin{tabular}{c}
\includegraphics[scale=0.7]{ombre.2}\\
Pièce 2\\
\end{tabular}
&\begin{tabular}{c}
\includegraphics[scale=0.7]{ombre.3}\\
Pièce 3\\
\end{tabular}
&\begin{tabular}{c}
\includegraphics[scale=0.7]{ombre.4}\\
Pièce 4\\
\end{tabular}
&
\begin{tabular}{c}
\includegraphics[scale=0.7]{ombre.6}\\
Pièce 5\\
\end{tabular}\\
\begin{tabular}{c}
\includegraphics[scale=0.7]{ombre.8}\\
Pièce 6\\
\end{tabular}
&\begin{tabular}{c}
\includegraphics[scale=0.7]{ombre.7}\\
Pièce 7\\
\end{tabular}
&\begin{tabular}{c}
\includegraphics[scale=0.7]{ombre.9}\\
Pièce 8\\
\end{tabular}
&\begin{tabular}{c}
\includegraphics[scale=0.7]{ombre.5}\\
Pièce 9\\
\end{tabular}\\
\end{tabular}
\end{center}
%@Commentaire: On travaille sur la représentation mentale des objets ; on recherche certaines vues. Pour les élèves en difficultés, il peut être intéressant de produire les solides en bois.