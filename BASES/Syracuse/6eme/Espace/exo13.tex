%@metapost:perspective.mp
%@Dif:3
\compo{9}{perspective}{1}{Le dessin ci-contre représente un pavé droit
en perspective cavalière avec en réalité $AB=3$~cm, $BC=5$~cm et
$CG=6$~cm.}
\begin{myenumerate}
\item Combien de faces compte le pavé droit ?\dotfill Quelle est leur
nature ?\dotfill
\item Cite toutes les faces du pavé :\dotfill\par
\dotfill
\item Quelles sont les faces \og{}visibles\fg{} ?\dotfill
\item Quelles sont les faces \og{}cachées\fg{} ?\dotfill
\item Sur le quadrillage ci-dessous, dessine les autres faces en vraie
grandeur.
\[\includegraphics{perspective.10}\]
\item Quelles sont les arêtes dessinées en vraie grandeur ?\dotfill
\item L'angle $\widehat{BAE}$ est-il un angle droit sur le dessin
?\dotfill Et en réalité ?\dotfill
\item L'angle $\widehat{BCG}$ est-il un angle droit sur le dessin
?\dotfill Et en réalité ?\dotfill
\end{myenumerate}
%@Commentaire: On continue le travail sur la perspective cavalière.