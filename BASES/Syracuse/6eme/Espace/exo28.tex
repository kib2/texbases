%@metapost:6espaceexo28.mp
%@Auteur: d'après Matthieu Berret\par
\compo{1}{6espaceexo28}{1}{Le bac à fleurs représenté sur la figure
  ci-contre est réalisé en ciment et a une épaisseur de 10~cm.
\par Un jardinier souhaite le remplir entièrement de terreau qui se
vend par sac de 6~L. Ce jardinier ne possède que quatre sacs.\par
L'objectif de cet exercice est de déterminer si ces quatre sacs vont
suffire.
}
\begin{myenumerate}
  \item Calculer, en centimètres, les dimensions (longueur, largeur et
    hauteur) du parallélépipède rectangle représentant l'intérieur du
    bac (partie à remplir de terreau).
  \item Calculer, en centimètres cube puis en litres, le volume de ce
    parallélépipède rectangle.
  \item Déterminer alors si les quatre sacs seront suffisants.
\end{myenumerate}