%@P:exocorcp
%@metapost:6espaceexo33.mp
%@Auteur: D'après {\em Mathématiques en sixième}\par
Sur la figure ci-dessous, le tracé en perspective cavalière d'un cube
$ABCDEFGH$ a été commencé.
\begin{myenumerate}
  \item
    \begin{enumerate}
    \item Termine le tracé. {\em Attention aux noms des sommets} : 
      \begin{itemize}
      \item la face située en haut du cube s'appelle $ABCD$;
      \item la face située en bas du cube s'appelle $EFGH$;
      \item la face, verticale, située à gauche du cube s'appelle $ADHE$;
      \item la face, verticale, située à droite du cube s'appelle $BCGF$;
      \item la face, verticale, située au fond du cube s'appelle $DCGH$.
      \end{itemize}
    \item Comment s'appelle la 6\ieme\ face, celle dont on n'a pas parlé ?
    \end{enumerate}
  \item On appelle {\em centre d'une face} le point d'intersection des
    diagonales de cette face.
    \begin{enumerate}
    \item Place sur le dessin :
      \begin{itemize}
      \item $R$ le centre de la face $ABCD$;
      \item $W$ le centre de la face $EFGH$;
      \item $U$ le centre de la face $BCGF$;
      \item $V$ le centre de la face $DCGH$;
      \item $S$ le centre de la face $ADHE$;
      \item $T$ le centre de la dernière face.
      \end{itemize}
    \item Trace en traits pleins les segments $[RS]$, $[RT]$, $[RU]$,
      $[WS]$, $[WT]$, $[WU]$, $[ST]$, $[TU]$.
    \item Trace en traits pointillés les segments $[SV]$, $[VU]$,
      $[RV]$, $[WV]$.
    \end{enumerate}
  \item Colorie le triangle $RST$ en vert; le triangle $RTU$ en jaune;
    le triangle $STW$ en bleu; le triangle $TUW$ en rouge.
  \item \`A quoi ressemble ce qui vient d'apparaître ?
\end{myenumerate}
\[\includegraphics{6espaceexo33.1}\]
%@Correction:
\[\includegraphics{6espaceexo33.2}\]