%@Dif:4
Une boite à chaussures est un parallélépipède rectangle de 17~cm
 sur 32~cm et de hauteur 11~cm.
\begin{myenumerate}
\item Dessine une représentation en perspective cavalière de la boite
{\bf sans couvercle}.\\\underline{\em Attention aux conventions}
\item Calcule le volume de cette boite et donne la réponse en litre.
\item Dessine un patron à l'échelle $\dfrac15$ de cette boite sans
 couvercle. On donnera d'abord les nouvelles dimensions.
\item Quelle est l'aire de carton nécessaire pour construire la boite
 à chaussures (sans couvercle) ?
\end{myenumerate}
%@Commentaire: Exercice de géométrie dans l'espace (perspective cavalière, patron) avec conversion d'unités de volume et utilisation de la multiplication d'un entier par une fraction.\par Exercice assez difficile, réservé à un travail d'approfondissement.