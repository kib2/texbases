%@Dif:3
Convertis en litre :
\begin{center}
12~hL\kern1cm400~dL\kern1cm60~daL\kern1cm2~500~cL\kern1cm18~daL\par
\par
0,5~hL\kern1cm3,75~daL\kern1cm940~cL\kern1cm0,5~dL\kern1cm0,09~daL
\end{center}
\par
\begin{center}
350~cL$+$0,08~daL$+$415~dL$+$0,02~hL\kern1cm3~hL~400~dL~5~cL$-$18~daL~12~cL
\end{center}
\begin{center}
2,5~m$^3$\kern1cm15\,000~cm$^3$\kern1cm0,9~dm$^3$\kern1cm0,04~m$^3$\kern1cm620~cm$^3$
\end{center}
%@Commentaire: La dernière question est difficile : il ne faut pas oublier la définition du litre : 1~L$=$1~dm$^3$.