%@P:exocorcp
%@Dif:3
Trois enfants achètent un sachet de 30 bonbons au prix de 8~\textgreek{\euro}. Eve en veut 15, Luc 9 et Paul le reste.
\begin{myenumerate}
  \item Combien de bonbons Paul aura-t-il ?
  \item Calcule sous forme fractionnaire le prix d'un bonbon.
  \item Calcule le montant que devra verser chaque enfant pour l'achat du paquet de bonbons.
\end{myenumerate}
%@Correction:
\begin{myenumerate}
  \item Paul en a 6.
  \item\subitem{}\par\opdiv*{8}{30}{a}{b}\opround{a}{2}{a}\opdiv[maxdivstep=4]{8}{30}\hfill Le prix d'un bonbon est $\dfrac8{30}$~\textgreek{\euro} ou environ \opprint{a}~\textgreek{\euro}.
  \item 
    \begin{description}
      \item[Eve] va payer la moitié : 4~\textgreek{\euro};
      \item[Luc] va payer $9\times\dfrac8{30}=\dfrac{72}{30}=\dfrac{24}{10}=$2,40~\textgreek{\euro};
      \item[Paul] va payer $6\times\dfrac8{30}=\dfrac{48}{30}=\dfrac{16}{10}=$1,60~\textgreek{\euro}.
    \end{description}
\end{myenumerate}