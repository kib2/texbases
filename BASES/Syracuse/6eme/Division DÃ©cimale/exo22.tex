%@Auteur: François Meria\par
Recopier et compléter le tableau suivant. On effectuera les
divisions décimales du tableau jusqu'à trois chiffres après la
virgule.

\begin{center}
\begin{tabularx}{\textwidth}{*{7}{|>{\centering}X}|}
\cline{2-7} \multicolumn{1}{>{\centering}X|}{}
       & \multicolumn{ 2}{c|}{valeur approchée} & \multicolumn{ 2}{c|}{valeur approchée} & \multicolumn{ 2}{c|}{valeur approchée} \tabularnewline
\multicolumn{1}{>{\centering}X|}{}
       & \multicolumn{ 2}{c|}{à l'unité par} & \multicolumn{ 2}{c|}{au dixième par} & \multicolumn{ 2}{c|}{au centième par} \tabularnewline
       \cline{2-7}
\multicolumn{1}{>{\centering}X|}{}       &     défaut &      excès
&     défaut &      excès &     défaut &      excès
       \tabularnewline \hline
       $17 \div 3$ &            &            &            & & &
       \tabularnewline \hline
       $1,529 \div 8$ &            &            & & &            &
       \tabularnewline \hline
       $25,7 \div 18$ &            &            &            & &            &
       \tabularnewline \hline
       $147 \div 56$ &            &            &            &
       & &            \tabularnewline \hline
       $7,778 \div 3$ &            &            &            & & &
       \tabularnewline \hline
       $555 \div 11$ &            &            &            & &            &
       \tabularnewline \hline
\end{tabularx}
\end{center}