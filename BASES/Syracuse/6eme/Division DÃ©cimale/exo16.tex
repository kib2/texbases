\begin{myenumerate}
\item Magali possède 45 roses. Elle veut faire des bouquets de 6
roses. Combien de bouquets peut-elle faire ? Lui restera-t-il des
fleurs ?
\par\dispo{1}{\vbox to3cm{\hbox to 5cm{}}}{\dotfill\par\dotfill\par\dotfill\par\dotfill}
\item Magali et cinq amies sont allées au restaurant. Le repas a coûté
55,56\textgreek{\euro}. Toutes souhaitent payer la même chose.
  \begin{enumerate}
  \item Quelle opération poser ?\dotfill\par\dotfill
  \item Donne un encadrement à l'unité près du prix à
payer.\dotfill\par\dotfill
  \item On peut donc écrire :
\[55,56=\ldots\times6+\ldots\]
Que reste-t-il à partager ? Comment faire ?\dotfill\par\dotfill
\item Pose alors l'opération.\par\vbox to 2.5cm{}\par
\item Quelle est alors la part à payer par chacune des amies ?\dotfill\par\dotfill\par\dotfill
\item Vérifie ton résultat par une autre opération.\par
\vbox to2.5cm{}\par
  \end{enumerate}
\item Magali a 45 litres d'eau. Elle a utilisé toute l'eau pour
remplir complètement 6 bidons identiques. Quelle quantité d'eau
a-t-elle mis dans chaque bidon ?\par\dispo{1}{\vbox to3cm{\hbox to 5cm{}}}{\dotfill\par\dotfill\par\dotfill\par\dotfill}
\end{myenumerate}
