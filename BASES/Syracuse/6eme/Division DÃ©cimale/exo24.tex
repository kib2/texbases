%@Auteur: François Meria\par
Recopier et compléter le tableau suivant. On effectuera les
divisions décimales du tableau jusqu'à trois chiffres après la
virgule. \'Ecrire dans la colonne \texttt{valeur approchée
choisie} la valeur approchée la plus proche du quotient.
\begin{center}
\begin{tabularx}{\textwidth}{|>{\centering}X|>{\centering}X|>{\centering}X|>{\centering}X|>{\centering}X|>{\centering}X|>{\centering}X|>{\centering}X|}
\cline{2-8}

\multicolumn{1}{>{\centering}X|}{} &
\multicolumn{1}{>{\centering}X|}{}& \multicolumn{2}{c|}{valeur
approchée} & \multicolumn{1}{>{\centering}X|}{} & \multicolumn{
2}{c|}{valeur approchée} &
\multicolumn{1}{>{\centering}X|}{valeur} \tabularnewline

\multicolumn{1}{>{\centering}X|}{}    & troncature & \multicolumn{
2}{c|}{à l'unité par} & troncature & \multicolumn{ 2}{c|}{au
dixième par} &  approchée \tabularnewline

\cline{3-4} \cline{6-7}

\multicolumn{1}{>{\centering}X|}{} & à l'unité & défaut & excès &
au dixième &     défaut & excès & choisie \tabularnewline

\hline $137 \div 49$ & & & & & & & \tabularnewline \hline $82 \div
36$ & &            & & &            & & \tabularnewline \hline $1
\div 73$ & & & & & & & \tabularnewline \hline $55 \div 52$ & & & &
& & & \tabularnewline \hline $69 \div 76$ & & & & & & &
\tabularnewline \hline $67 \div 63$ & & & & & & & \tabularnewline
\hline $135 \div 76$ &            & & & & & & \tabularnewline
\hline
\end{tabularx}
\end{center}