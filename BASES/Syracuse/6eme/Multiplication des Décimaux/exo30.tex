%@P:exocorcp
%@Dif:3
Au rayon charcuterie, un client achète 0,300~kg de mousse de canard à
12,50~\textgreek{\euro} le kg ; 0,280~kg de pâté de campagne à
11~\textgreek{\euro} le kg ; 0,250~kg de rillettes à
13,56~\textgreek{\euro} le kg et 3 pâtés impériaux à 1,80~\textgreek{\euro} l'un. Avec 15,24~\textgreek{\euro}, aura-t-il assez d'argent pour payer ?
%@Correction:
\opmul*{0.300}{12.50}{a}\opmul*{0.280}{11}{b}\opmul*{0.250}{13.56}{c}\opmul*{3}{1.80}{d}
\opadd*{a}{b}{e}\opadd*{e}{c}{f}\opadd*{f}{d}{g}
\opmul{0.300}{12.50}\kern1cm\opmul{0.280}{11}\kern1cm\opmul{0.250}{13.56}\kern1cm\opmul{3}{1.80}
\par Ce qui représente un total de \opprint{g}~\textgreek{\euro}. Donc le client ne dispose pas d'assez d'argent.
%@Commentaire: Appropriation de la multiplication de deux nombres décimaux. Travail avec un facteur plus petit que 1.