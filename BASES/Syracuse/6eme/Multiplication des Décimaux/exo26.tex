%@P:exocorcp
%@Dif:2
\opset{voperator=bottom,decimalsepsymbol={,}}
Les multiplications ci-dessous sont correctes au niveau des calculs mais il manque la virgule dans le résultat final. Replace la correctement.
\\
\opmul[displayshiftintermediary=all,resultstyle.d=\troucp{}]{24,5}{12,7}\hfill
\opmul[displayshiftintermediary=all,resultstyle.d=\troucp{}]{0,25}{1,7}\hfill
\opmul[displayshiftintermediary=all,resultstyle.d=\troucp{}]{2,5}{3,14}\hfill
\opmul[displayshiftintermediary=all,resultstyle.d=\troucp{}]{45}{2,7}\hfill
\opmul[displayshiftintermediary=all,resultstyle.d=\troucp{}]{4,218}{1,7}
%@Correction:
\opmul{24,5}{12,7}\hfill
\opmul{2,5}{3,14}\hfill
\opmul{45}{2,7}\hfill
\opmul{4,218}{1,7}
%@Commentaire: La position de la virgule est importante, les élèves ayant tendance à ne compter que les chiffres du 1\ier\ facteur.