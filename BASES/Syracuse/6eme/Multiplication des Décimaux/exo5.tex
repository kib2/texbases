%@P:exocorcp
%@Dif:2
\begin{myenumerate}
\item Amélie a calculé $0,123\,45\times1\,000$. Quel est le
chiffre des unités du résultat ? Justifie la réponse.
\item Reprends la question précédente pour les calculs
$12,345\times1\,000$; $0,012\,345\times100$; $12,345\times10$.
\item Christian a posé le calcul $987,45\div1\,000$. Quel
est le chiffre des unités du résultat ? Justifie la réponse.
\item Reprends la question précédente avec les calculs $98\,745\div100$;
$9\,874,5\div10$.
\end{myenumerate}
%@Correction:
\begin{myenumerate}
  \item Le résultat est 123,45. Le chiffre des unités est 3.
  \item \opmul[style=text]{12,345}{1000}. Le chiffres des unités est 5.
\par\opmul[style=text]{0,012345}{100}. Le chiffres des unités est 1.
\par\opmul[style=text]{12,345}{10}. Le chiffres des unités est 3.
\item Le résultat est 0,98745. Le chiffre des unités est 0.
\item \opdiv[style=text]{98745}{100}. Le chiffre des unités est 7.
\par\opdiv[style=text]{9874,5}{10}. Le chiffre des unités est 7.
\end{myenumerate}
%@Commentaire: Multiplication par 10, 100, 1\,000 et sens d'un chiffre dans l'écriture d'un nombre. Donné en DS après le chapitre sur les nombres décimaux.