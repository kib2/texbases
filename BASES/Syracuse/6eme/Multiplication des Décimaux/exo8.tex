%@P:exocorcp
%@Dif:2
Un quincaillier reçoit trois colis :
\begin{itemize}
\item le premier contient 250 marteaux pesant chacun 680~g;
\item le deuxième contient 250 boîtes de 510 vis. Chaque vis pèse
2,5~g;
\item le troisième contient 25 boîtes de peinture de 3~kg chacune.
\end{itemize}
Quelle est la masse totale des colis déposés ?
%@Correction:
\begin{multicols}{3}
  \begin{center}
1\ier\ colis.\par
    \begin{tabular}{c|b{2cm}}
      Opération&\multicolumn{1}{c}{Solution}\\
      \hline
      \opmul*{250}{680}{a}\opmul{250}{680}&Le 1\ier\ colis pèse \opprint{a}~g soit \opdiv*{a}{1000}{a}{b} \opprint{a}~kg.\\
    \end{tabular}
  \end{center}
\par\columnbreak
  \begin{center}
2\ieme\ colis.\par
    \begin{tabular}{c|b{3cm}}
      Opération&\multicolumn{1}{c}{Solution}\\
      \hline
      \opmul*{250}{510}{c}\opmul{250}{510}&\\
\opmul*{c}{2,5}{b}\opmul{c}{2,5}&Il y avait \opprint{c} vis et la masse est \opprint{b}~g soit \opdiv*{b}{1000}{b}{d}\opprint{b}~kg.\\
    \end{tabular}
  \end{center}
\par\columnbreak
  \begin{center}
3\ieme\ colis.\par
    \begin{tabular}{c|b{2cm}}
      Opération&\multicolumn{1}{c}{Solution}\\
      \hline
      \opmul*{25}{3}{d}\opmul{25}{3}&Le 3\ieme\ colis pèse \opprint{d}~kg.\\
    \end{tabular}
  \end{center}
\end{multicols}
La masse totale est \opadd*{a}{b}{e}\opadd*{e}{d}{e}\opprint{e}~kg.
%@Commentaire: C'est de la proportionnalité. La multiplication est ici vue comme opérateur.
