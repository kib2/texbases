%@Auteur:Véronique Glaçon\par
\begin{myenumerate}
\item Construis un segment $[TQ]$ de longueur 9~cm.
\item Place le point $P$ sur le segment $[TQ]$ tel que $TP=5$~cm.
\item Construis la droite $(d)$ médiatrice du segment $[PQ]$. Cette
  droite coupe le segment $[PQ]$ en $O$.
\item Construis le cercle de diamètre $[TP]$. On note I le centre de ce cercle.
\item Place un point $S$ sur ce cercle tel que $PS=2,5$~cm.
\item Construis la droite perpendiculaire à la droite $(d)$ passant par $S$.
\item Construis la droite perpendiculaire à la droite $(IS)$ passant
  par $I$, elle coupe la droite $(d)$ en N.
\end{myenumerate}