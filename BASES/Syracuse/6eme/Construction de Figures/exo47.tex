%@Titre:Conchoïde de Nicomède
%@geogebra:6constfigexo47.ggb
%@Danger: Il faut être le plus soigneux possible.
Soit une droite $(d)$ et un point $P$ n'appartenant pas à cette droite
$(d)$.
\begin{myenumerate}
  \item\label{6constfigexo47q1} Prends un point $Q_1$ sur la droite $(d)$.
  \item Construis le point $M_1$ de la demi-droite $[PQ_1)$ tel que
    $Q_1M_1=4$~cm. Ce point $M_1$ est situé de l'autre côté de la
    droite $(d)$ que $P$.
  \item Recommence la question \ref{6constfigexo47q1} plusieurs fois
    pour obtenir les points $M_2$, $M_3$, $M_4$, \ldots
\end{myenumerate}
%@Infor: On obtient ici une conchoïde de droite. Nicomède l'a définie et étudiée vers 200 Av J.C.