%@Titre:Un escargot mathématique.
%@Auteur: Véronique Glaçon\par
\begin{myenumerate}
  \item Trace un triangle $OAB$ rectangle et isocèle en $A$ tel que
    $OA=3$~cm. Code la figure.
  \item Construis le triangle $OBC$ rectangle en $B$ à côté du
    triangle $OAB$ tel que $BC=3$~cm. Code la figure.
  \item Construis le triangle $OCD$ rectangle en $C$ à côté du
    triangle $OBC$ tel que $CD=3$~cm. Code la figure.
  \item Construis le triangle $ODE$ rectangle en $D$ à côté du
    triangle $OCD$ tel que $DE=3$~cm. Code la figure.
  \item Continue ainsi de suite, pour construire les triangles $OEF$,
    $OFG$, $OGH$ et $OHI$.
\end{myenumerate}
{\em Si ton dessin est bien fait}, le segment $[OI]$ doit mesurer 9~cm
{\em exactement}.