%@geogebra:6constfiguresexo23.ggb
%@Auteur:D'après APMEP.
\begin{myenumerate}
  \item Trace une droite $(d)$ et place un point $S$ n'appartenant pas à la droite $(d)$.
  \item Prends un point $M$ sur la droite $(d)$. Trace au crayon gris le segment $[SM]$, puis, à l'encre, la demi-droite $[Mx)$ qui est perpendiculaire à la droite $(SM)$ et qui se trouve du même côté que $S$ par rapport à la droite $(d)$.
  \item Recommence comme au 2/ avec de nombreux points de la droite $(d)$.
  \item Tu constates que tes demi-droites \og enveloppent\fg\ une courbe. Dessine cette courbe au crayon gris. Renseigne-toi sur son nom.
\end{myenumerate}