%@Auteur: Nathalie Lespinasse\par
Tracer un cercle de centre $O$ et de rayon 10~cm et deux diamètres perpendiculaires $[AB]$ et $[CD]$.
\par Construire les bissectrices des quatre angles obtenus. Leur intersection avec le cercle donne dans l'ordre les points $A$, $E$, $D$, $F$, $B$, $G$, $C$ et $H$. Les joindre deux à deux dans cet ordre. On obtient un octogone.
\par Dans le triangle $OAH$, placer les points $H'$ sur $[OH]$, $O'$ sur $[OA]$ et $A'$ sur $[AH]$ tels que $HH'=OO'=AA'=1$~cm. Joindre les points $H'$, $O'$, $A'$ pour former un triangle. Recommencer dans ce triangle $O'A'H'$ les mêmes constructions et ainsi de suite jusqu'à ce qu'on ne puisse plus construire de triangle.
\par Dans le triangle voisin $OAE$, faire les mêmes constructions mais en tournant dans l'autre sens.
\par Continuer ainsi dans chacun des triangles $OED$, $ODF$,\ldots en alternant toujours le sens de rotation.