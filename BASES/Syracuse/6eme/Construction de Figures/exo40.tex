%@Titre: L'heptagone régulier (construction approchée)
%@Auteur: d'après IREM Réunion.
\begin{myenumerate}
  \item Construis un cercle de centre $O$ et de rayon 5~cm.
  \item
    \begin{enumerate}
      \item Construis un diamètre $[II']$ de ce cercle.
      \item Construis un diamètre $[JJ']$ de ce cercle tels que les droites $(II')$ et $(JJ')$ soient perpendiculaires.
    \end{enumerate}
  \item Soit $K$ le milieu du segment $[OI]$.
    \begin{enumerate}
    \item Trace la droite $(J'K)$ : elle recoupe le cercle en $L$.
    \item Trace la parallèle à la droite $(J'K)$ passant par $O$ : elle coupe le cercle en $M$ ($M$ se situant du même côté que $L$).
    \end{enumerate}
  \item La perpendiculaire à la droite $(OI)$ passant par $M$ coupe la droite $(OI)$ en $M'$. La perpendiculaire à la droite $(OI)$ passant par $L$ coupe la droite $(OI)$ en $L'$.
  \item On appelle $A'$ le milieu du segment $[L'M']$. La perpendiculaire à la droite $(OI)$ passant par $A'$ coupe le cercle en $A$.
  \item Reporte sept fois la longueur $IA$ sur le cercle.
\end{myenumerate}