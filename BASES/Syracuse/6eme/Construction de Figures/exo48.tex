%@Titre:Ellipse point par point
%@geogebra:6constfigexo48.ggb
%@Danger: Il faut être le plus soigneux possible.
Sur une feuille blanche, réalise le programme de construction suivant :
\begin{myenumerate}
  \item Place deux points $O$ et $P$ distants de 60~mm au centre de la
    feuille.
  \item Trace un segment $[AB]$ de longueur 10~cm au centre de la feuille.
  \item Place un point $C$ sur le segment $[AB]$.
  \item Le cercle de centre $O$ et de rayon $AC$ coupe le cercle de
    centre $P$ et de rayon $BC$ en deux points. {\em Si ce n'est pas
      le cas, change la position du point $C$}. Marque ces deux points
    en rouge. Efface les deux cercles et le point C.
  \item Change la position du point $C$ sur le segment $[AB]$, et refais la même construction que dans la question précédente.
\end{myenumerate}
Marque beaucoup de points rouges en modifiant la position du point $C$
sur le segment $[AB]$. Ces points rouges sont sur une ligne appelée
{\em ellipse}.