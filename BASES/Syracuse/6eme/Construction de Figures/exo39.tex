%@Auteur: François Meria\par
\texttt{Toutes les constructions doivent se faire au COMPAS et à la règle
(sans utiliser les graduations sauf pour l'étape 1).}

\shadowbox{
\begin{minipage}[c]{\textwidth}
\textbf{\'Etape 1.}\\
Construire un cercle $\mathcal{C}$ de centre $O$ et de diamètre
horizontal $[DD']$ tel que $DD'=16~$cm.\\

\textbf{\'Etape 2.}\\
Construire la médiatrice du segment $[DD']$. Nommer $A$ le point
du cercle au Nord, c'est-à-dire sur le cercle, sur la médiatrice
de $[DD']$ en haut du cercle.\\

\textbf{\'Etape 3.}\\
Construire la médiatrice du segment $[OD']$ et nommer $B$ le
milieu du $[OD']$.\\
Tracer le segment $[AB]$ en pointillés.\\

\textbf{\'Etape 4.}\\
Tracer le cercle de centre $B$ passant par $A$ et nommer $E$ le
point d'intersection du segment $[DO]$ et de ce cercle.\\

\textbf{\'Etape 5.}\\
Tracer le cercle de centre $A$ et de rayon $[AE]$. Il coupe le
cercle $\mathcal{C}$ de départ en $A_1$ et $A_4$.\\

\textbf{\'Etape 6.}\\
Le cercle de centre $A_1$ passant par $A$ recoupe le cercle ${\cal
C}$ en $A_2$.\\
Le cercle de centre $A_4$ passant par $A$ recoupe le cercle ${\cal
C}$ en $A_3$.\\

\textbf{\'Etape 7.}\\
Tracer le polygone $AA_2A_4A_1A_3$ puis effacer les traits de
construction. Enfin colorier le \textit{pentagramme} ainsi obtenu.
\end{minipage}
}