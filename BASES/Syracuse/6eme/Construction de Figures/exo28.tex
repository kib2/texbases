%@Auteur: Virginie Morel
\begin{myenumerate}
  \item Place deux points $A$ et $B$ distant de 3~cm.
  \item Trace le cercle de centre $A$ et de rayon 3~cm. On l'appelle $\cal C$.
  \item Trace le cercle de centre $B$ et de rayon 3~cm. On l'appelle ${\cal C}'$.
  \item Ces deux cercles se coupent en $E$ et $F$. Place ces deux points.
  \item La droite $(AE)$ recoupe le cercle $\cal C$ en $G$. Place $G$.
  \item La droite $(EB)$ recoupe le cercle ${\cal C}'$ en $H$. Place $H$.
  \item Trace en couleur l'arc de cercle de centre $E$ et de rayon $EG$ compris entre $H$ et $G$.
  \item La droite $(FA)$ recoupe le cercle $\cal C$ en $I$. Place $I$.
  \item La droite $(FB)$ recoupe le cercle ${\cal C}'$ en $J$. Place $J$.
  \item Trace en couleur l'arc de cercle de centre $F$ et de rayon $FI$ compris entre $I$ et $J$.
  \item Repasse en couleur certains arcs de cercle pour que la figure obtenue soit un ovale.
\end{myenumerate}