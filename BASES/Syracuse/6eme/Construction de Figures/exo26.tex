%@metapost:cons605exo03.mp
%@Auteur: d'après IREM Strasbourg
\par\compo{1}{cons605exo03}{1}{Trace trois droites $(d_1)$, $(d_2)$ et
$(d_3)$ qui se coupent respectivement en $A$, $B$ et $C$ comme
l'indique la figure ci-contre.\par Trace ensuite sur la droite $(d_1)$
un point $M$ qui n'est pas sur le segment $[AC]$ et tel que $AM=1$~cm.
\par Trace l'arc de cercle de centre $A$ qui relie $M$ à $P$.
\par Trace l'arc de cercle de centre $B$ qui relie $P$ à $Q$.
\par Trace l'arc de cercle de centre $C$ qui relie $Q$ à $R$.
\par Trace l'arc de cercle de centre $A$ qui relie $R$ à $S$.
\par Trace l'arc de cercle de centre $B$ qui relie $S$ à $T$.
\par{\em Si la figure est bien faite, on peut tracer l'arc de cercle
de centre $C$ qui relie $T$ à $M$.}
}