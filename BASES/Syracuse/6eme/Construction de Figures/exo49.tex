%@metapost:6constructionfiguresexo49.mp
%@Titre: Animaux compassés.
%@Auteur:Véronique Glaçon\par
Sur la feuille suivante :
\begin{myenumerate}	
	\item Trace le cercle $\mathscr{C}_A$ de centre A passant par M.
\vspace{0.2cm}	
	\item Trace le cercle $\mathscr{C}_B$ de centre B et de rayon $1,8$~cm.
\vspace{0.2cm}	
	\item Trace l'arc de cercle $\mathscr{C}_E$ de centre E et de rayon $2$~cm à l'intérieur du cercle $\mathscr{C}_A$.
\vspace{0.2cm}	
	\item Trace l'arc de cercle $\mathscr{C}_F$ de centre F et de rayon $2$~cm à l'intérieur du cercle $\mathscr{C}_B$.
\vspace{0.2cm}
	\item Trace le cercle de centre C et de rayon $0,9$~cm. 
\vspace{0.2cm}	
	\item Trace le cercle de centre C et de rayon $1,4$~cm
\vspace{0.2cm}	
	\item Trace le cercle de centre D et de rayon $0,9$~cm. 
\vspace{0.2cm}	
	\item Trace le cercle de centre D et de rayon $1,4$~cm.
\vspace{0.2cm}
	\item Sans tracer le segment [CD], place I le milieu de ce segment.
\vspace{0.2cm}
	\item Trace l'arc de cercle de centre I et de rayon $6,3$~cm qui va de la partie supérieure du cercle $\mathscr{C}_A$ (à gauche) à la partie supérieure du cercle $\mathscr{C}_B$ (à droite) sans aller à l'intérieur de ces cercles
\vspace{0.2cm}
	\item Trace l'arc de cercle de centre G, de rayon $8$~cm qui démarre du bord inférieur du cercle $\mathscr{C}_B$.
\vspace{0.2cm}	
	\item Trace l'arc de cercle de centre H, de rayon $8$~cm qui démarre du bord inférieur du cercle $\mathscr{C}_A$.
\vspace{0.2cm}	
	\item Ces deux arcs de cercle se coupent au point J, il est inutile de prolonger au delà.  
\vspace{0.2cm}	
  \item Trace vers le haut l'arc de cercle de centre J et de rayon $1,5$~cm, limité par les deux arcs précédents.
\vspace{0.2cm}
  \item Trace une partie du cercle de centre M passant par N, qui ne doit pas passer sur ce qui a déjà été dessiné.
\vspace{0.2cm}
  \item Trace au arc de cercle de centre N passant par P ( en y démarrant d'ailleurs ) dans le sens inverse des aiguilles d'une montre et qui ne rentre pas dans les figures déjà tracées.
\vspace{0.2cm}  
   \item Trace le demi-cercle vers le bas de diamètre [PQ].
\vspace{0.2cm}
   \item Trace l'arc de cercle de centre R passant par Q ( en y démarrant d'ailleurs ) dans le sens inverse des aiguilles d'une montre et qui ne rentre pas dans les figures déjà tracées.
\vspace{0.2cm}
   \item Quel animal as tu dessiné? \dotfill
        \\ A l'aide de ton compas, trace ses moustaches.
\end{myenumerate}
\newpage
\begin{figure}[b] 
\begin{center}
\includegraphics{6constructionfiguresexo49.1}
\end{center}
\end{figure}