%@Auteur: Thomas Rey
%@Dif:3
La construction décrite ci-dessous est à faire sur une feuille blanche
non quadrillée.
\begin{myenumerate}
\item Construis un triangle $ABC$ tel que $\widehat{ABC}=105$\degres;
$AB~=~7,3$~cm; $BC~=~9,6$~cm.
\item Trace la parallèle à la droite $(BC)$ passant par $A$ et la
parallèle à la droite $(AB)$ passant par $C$. Ces deux droites se
coupent en un point que l'on appelle $D$.
\item \`A l'extérieur du quadrilatère $ABCD$, place le point $F$ tel
que $\widehat{CBF}=105$\degres\ et $BF=7,3$~cm.
\item Trace la parallèle à la droite $(BC)$ passant par $F$ et la
parallèle à la droite $(FB)$ passant par $C$. Ces deux droites se
coupent en un point que l'on appelle $E$.
\item Trace la parallèle à la droite $(AB)$ passant par $F$ et la
parallèle à la droite $(FB)$ passant par $A$. Ces deux droites se
coupent en un point que l'on appelle $G$.
\item
  \begin{enumerate}
  \item Colorie l'intérieur du quadrilatère $ABCD$ en jaune.
  \item Colorie l'intérieur du quadrilatère $BCEF$ en rouge.
  \item Colorie l'intérieur du quadrilatère $ABFG$ en vert.
  \end{enumerate}
\end{myenumerate} Cette figure est la représentation en perspective
\emph{axonométrique} d'un cube.