%@metapost:6constfigexo44.mp
\compo{1}{6constfigexo44}{1}{On donne le texte de construction suivant :
\begin{itemize}
\item Trace un triangle $ABC$, isocèle en $C$, tel que $AB=3$~cm et
$BC=4$~cm.
\item Trace la médiatrice du segment $[BC]$. Elle coupe le segment
  $[AC]$ en $F$ et le segment $[BC]$ en $E$. Place les points $E$ et
  $F$.
\item Place, à l'extérieur du triangle $ABC$, le point $G$ tel que le
triangle $CFG$ soit équilatéral.
\item La parallèle à la droite $(EF)$ passant par $C$ coupe le segment
  $[FG]$ en $H$.
\item Place, à l'extérieur du triangle $FGC$, le point $I$ tel que le
  triangle $FHI$ soit équilatéral.
\end{itemize}
La figure a déjà été commencée ci-contre. Termine-la.
}