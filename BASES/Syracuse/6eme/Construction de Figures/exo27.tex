%@P:exocorcp
%@geogebra:6constfiguresexo27.ggb
%@metapost:cons605exo04.mp
%@Auteur:d'après IREM Strasbourg
\begin{myenumerate}
  \item Trace un cercle de rayon 2,5~cm et marque un point $O$ sur ce cercle.
  \item Place un point $M_1$ sur ce cercle. Trace la droite $(OM_1)$. Trace sur cette droite, en rouge, les points $N_1$ et $P_1$ tels que $M_1N_1=M_1P_1=5$~cm.
\item Place un point $M_2$ sur ce cercle. Trace la droite $(OM_2)$. Trace sur cette droite, en rouge, les points $N_2$ et $P_2$ tels que $M_2N_2=M_2P_2=5$~cm.
\item Place un point $M_3$ sur ce cercle. Trace la droite $(OM_3)$. Trace sur cette droite, en rouge, les points $N_3$ et $P_3$ tels que $M_3N_3=M_3P_3=5$~cm.
\item Recommence avec d'autres points $M_4$, $M_5$,\ldots beaucoup de points sur le cercle.
\end{myenumerate}
Si tu as construit de nombreux points $M$, tu dois voir apparaître
une courbe appelée {\em cardioïde}.
%@Correction:
\[\includegraphics{cons605exo04.1}\]
