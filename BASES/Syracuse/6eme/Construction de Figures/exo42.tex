%@Titre: Construction de tangentes par la méthode d'Euclide.
Construis un cercle $\mathscr{C}$ de centre $O$ et de rayon 4~cm et
place un point $A$ en dehors de ce cercle.
\par Soit $H$ le point d'intersection du segment $[OA]$ et de
$\mathscr{C}$.
\par Le cercle de centre $O$ et de rayon $OA$ coupe la perpendiculaire
à la droite $(OA)$ passant par $H$ en $E$ et $F$.
\par On appelle $C$ et $D$ les points d'intersection respectifs de
$\mathscr{C}$ avec les segments $[OE]$ et $[OF]$.
\par{\em Que peut-on dire des droites $(AC)$ et $(OE)$ ? des droites
  $(AD)$ et $(OF)$ ?}
