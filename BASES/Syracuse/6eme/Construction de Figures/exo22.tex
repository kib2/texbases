%@geogebra:6constfiguresexo22.ggb
%@Auteur: D'après APMEP.
\begin{myenumerate}
  \item Trace un cercle aussi grand que possible sur ta feuille.
  \item Prends un point $S$ à l'extérieur du disque - pas trop loin.
  \item Prends un point $M$ sur le cercle. Au crayon gris, trace le segment $[SM]$ puis, à l'encre, la droite perpendiculaire en $M$ à la droite $(SM)$.
  \item Recommence comme au 3/ avec de nombreux points sur le cercle.
  \item Tu constates que tes segments \og enveloppent\fg\ une courbe. Dessine cette courbe au crayon gris. Renseigne-toi sur son nom.
\end{myenumerate}