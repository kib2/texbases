Un trésor se trouve enfoui sur une île sur laquelle se trouve un arbre
et une source distants de 400~m. On ne peut pas aller de l'arbre au
trésor en ligne droite à cause d'un rocher. Un parchemin révèle que,
pour trouver le trésor, il faut :
\[
\psshadowbox{
\begin{minipage}{300pt}
\begin{itemize}
\item{\em en partant de l'arbre, parcourir 300~\mbox{m} en direction de la
source ;}
\item{\em tourner sur la droite de 60\degres ;}
\item{\em avancer en ligne droite de 500~\mbox{m} ;}
\item{\em tourner sur la droite de 100\degres ;}
\item{\em avancer en ligne droite de 700~\mbox{m} et creuses.}
\end{itemize}
\end{minipage}
}
\]
\begin{myenumerate}
\item Place deux points $A$ (arbre) et $S$ (source) distants de
4~cm. \`A quelle longueur réelle correspond 1~cm sur le dessin ?
\item Place le point $T$ où se trouve le trésor. Quelle est la
distance de l'arbre au trésor ?
\item Mesure l'angle $\widehat{SAT}$. Qu'aurait-on écrit s'il n'y
avait pas eu le rocher ?
\end{myenumerate}