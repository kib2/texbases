%@P:exocorcp
%@Titre: Des flèches.
%@Auteur: D'après Didier BEGLIOMINI -- Mathématiques archéologiques.\par
\begin{myenumerate}
\item Sur la même figure,
  \begin{enumerate}
  \item Trace un triangle $ABC$ tel que $AB=2$~cm; $AC=5$~cm et
$BC=4$~cm.
  \item Construis le triangle $BCD$ rectangle en $B$ avec $BD=2,5$~cm,
$D$ n'étant pas du même côté que $A$ par rapport au segment $[BC]$.
  \end{enumerate}
\item {\em On fera une nouvelle figure.}\par Sur la même figure,
  \begin{enumerate}
  \item Trace un triangle $ABC$ isocèle de sommet $A$ tel que
$AB=2,5$~cm et $BC=4$~cm.
  \item Construis le triangle $BCD$ équilatéral tel que $D$ et $A$ ne
soient pas du même côté du segment $[BC]$.
  \item Soit $O$ le point d'intersection des segments $[AD]$ et
$[BC]$, et $I$ le point du segment $[BC]$ tel que $BI=1$~cm. Construis
le triangle $OIJ$ rectangle en $O$ avec $IJ=3$~cm de façon que $J$
soit du même côté que $A$ par rapport au segment $[BC]$.
  \item Construis le triangle $IJK$ isocèle de sommet $J$ avec
$IK=2$~cm, le point $K$ se trouvant du même côté que le point $C$ par
rapport au segment $[IJ]$.
  \end{enumerate}
\end{myenumerate}
%@Correction:
\includegraphics{6consexo24c.1}\hfill\includegraphics{6consexo24c.2}