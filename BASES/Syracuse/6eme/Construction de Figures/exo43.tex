%@Auteur: D'après APMEP.\par
\raisebox{6pt}{\hbox{\dbend}}$\left\{
    \begin{tabular}{l}
      $\star${\em On utilisera une feuille non quadrillée au format
        {\em 21$\times$29,7}.}\\
      $\star${\em On sera extrêmement précis et soigneux.}
    \end{tabular}
    \right.
$
\begin{myenumerate}
  \item Trace un cercle aussi grand que possible sur ta feuille.
  \item Prends un point $S$ à l'intérieur du cercle - pas trop près du
    centre.
  \item Prends un point $M_1$ sur le cercle. Au crayon gris, trace
    {\em en pointillés} le segment $[SM_1]$ puis la droite $(d_1)$
    perpendiculaire à la droite $(SM_1)$ em $M_1$ et qui recoupe le
    cercle. Repasse {\em en trait continu} la partie de la droite $(d_1)$ qui
    est intérieure au cercle.
  \item Recommence comme au 3/ avec de nombreux points sur le
    cercle. {\em On en prendra au minimum 20.}
  \item Tu constates que tes segments \og enveloppent\fg\ une
    courbe. Dessine cette courbe au crayon gris. Renseigne-toi sur son
    nom.
\end{myenumerate}
%@Commentaire: Variante de l'exercice \verb+exo21+ destinée à une classe plus faible.