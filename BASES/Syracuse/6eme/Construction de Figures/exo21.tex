%@geogebra:6constfiguresexo21.ggb
%@Auteur: D'après APMEP.
\begin{myenumerate}
  \item Trace un cercle aussi grand que possible sur ta feuille.
  \item Prends un point $S$ à l'intérieur du disque - pas trop près du
    centre.
  \item Prends un point $M$ sur le cercle. Au crayon gris, trace le
    segment $[SM]$ puis la demi-droite $[Mx)$, d'origine $M$,
    perpendiculaire à la droite $(SM)$ et qui recoupe le
    cercle. Repasse à l'encre la partie de la demi-droite $[Mx)$ qui
    est intérieure au disque.
  \item Recommence comme au 3/ avec de nombreux points sur le cercle.
  \item Tu constates que tes segments \og enveloppent\fg\ une
    courbe. Dessine cette courbe au crayon gris. Renseigne-toi sur son
    nom.
\end{myenumerate}