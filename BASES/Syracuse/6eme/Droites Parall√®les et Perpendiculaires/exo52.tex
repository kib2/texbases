%@metapost:6parallelesexo51.mp
%@Auteur: d'après Céline Leroy\par
\compo{4}{6parallelesexo51}{1}{\begin{itemize} 
\item Trace la perpendiculaire à la droite $(d)$ passant par $A$. Elle
  coupe la droite $(d)$ en $M$.
\item Trace la parallèle à la droite $(d)$ passant par $B$. Elle coupe
  la droite $(AM)$ en $N$.
\end{itemize}
Que peut-on dire des droites $(AM)$ et $(BN)$ ? Justifie la réponse.}