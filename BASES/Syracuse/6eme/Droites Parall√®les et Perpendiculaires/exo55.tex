%@Titre:L'anguinéa
%@Auteur:\url{http://www.animath.fr/UE/missenard/courbes.html}
%@geogebra:6parallelesexo55.ggb
%@Danger:Réalise ces courbes sur grande feuille, avec soin et précision.
\begin{myenumerate}
  \item Trace un cercle $\mathscr C$ de diamètre $[OA]$ et une
    parallèle $(d)$ à la droite $(OA)$ qui coupe le cercle $\mathscr C$.
  \item Une droite quelconque passant par $O$ recoupe le cercle
    $\mathscr C$ en un point $P$ et coupe la droite $(d)$ en un point $N$.
  \item La parallèle à la droite $(d)$ passant par $P$ et la
    perpendiculaire à la droite $(d)$ passant par $N$ se coupent en un
    point $M$.
  \item Recommence l'étape précédente en prenant une autre droite
    passant par $O$ : tu peux ainsi construire de nombreux points $M$.
\end{myenumerate}
Tu verras apparaître la courbe qui est l'ensemble de ces points
$M$. Cette courbe a pour nom {\em anguinéa}.
