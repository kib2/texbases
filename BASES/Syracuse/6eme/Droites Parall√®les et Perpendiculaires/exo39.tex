%@Dif:2
\begin{myenumerate}
\item Trace une droite $(d_1)$ et place un point $A$ qui n'est pas sur
  la droite $(d_1)$.
\item Trace une droite $(d_2)$ qui est perpendiculaire à la droite
  $(d_1)$.
\item Trace la droite $(d_3)$ qui est parallèle à la droite $(d_1)$ et
  qui passe par le point $A$.
\end{myenumerate}
%@Commentaire: Application directe des techniques de tracés de droites parallèles et perpendiculaires.