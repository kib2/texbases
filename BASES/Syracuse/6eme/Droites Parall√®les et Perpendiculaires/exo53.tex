%@metapost:6parallelesexo53.mp
%@Auteur: D'après {\em Aide et soutien Mathématiques CM2/6\ieme} - SCEREN\par
On considère la situation suivante :
\begin{quote}
  On dispose d'une droite $(d)$ et d'un point $A$. On souhaite tracer
  la parallèle à la droite $(d)$ passant par $A$. Mais un pavé est
  placé de la façon suivante :
\par
\begin{tabular}{cc}
\includegraphics[scale=0.75]{6parallelesexo53.1}&\includegraphics[scale=0.75]{6parallelesexo53.2}\\
Vue {\em normale}&Vue de dessus.
\end{tabular}
\end{quote}
Comment construire cette parallèle à la droite $(d)$ passant par $A$ ?
Justifie la méthode envisagée. {\em Pour cela, on pourra reproduire la
  vue de dessus et faire les explications sur cette figure.}