%@Auteur: D'après IREM Strasbourg
\begin{myenumerate}
  \item
    \begin{enumerate}
    \item Trace un segment $[AB]$ de 10~cm de longueur.
    \item Trace une droite $(d_1)$ passant par $A$.
    \item Trace la droite $(\Delta_1)$ passant par le point $B$ et
      perpendiculaire à la droite $(d_1)$.
    \item Appelle $C_1$ le point d'intersection des droites $(d_1)$ et
      $(\Delta_1)$.
    \item Trace une droite $(d_2)$ passant par $A$.
    \item Trace la droite $(\Delta_2)$ passant par le point $B$ et
      perpendiculaire à la droite $(d_2)$.
    \item Appelle $C_2$ le point d'intersection des droites $(d_2)$ et
      $(\Delta_2)$.
    \end{enumerate}
  \item Trace le cercle de diamètre $[AB]$. Que remarque-t-on ?
\end{myenumerate}
%@Commentaire: Variante de l'exercice \verb+exo23+. Donné en DS.