%@Titre:Une cardioïde point par point
%@Auteur:\url{www.col-camus-soufflenheim.ac-strasbourg.fr/}\par
\begin{myenumerate}
  \item Dessine au centre de ta feuille un cercle de centre $O$ et de rayon 8~cm. Place un point $A$ sur ce cercle.
  \item \label{6parallelesexo57q2} Place un point $M$ sur le cercle. Trace le segment $[OM]$.
  \item Trace d'un trait fin la perpendiculaire au segment $[OM]$ et
    passant par $M$. Cette droite s'appelle $(d)$.
  \item Trace d'un trait fin la perpendiculaire à $(d)$ et passant par $A$. Cette droite s'appelle $(d')$.
  \item Marque une petite croix verte à l'intersection des droites $(d)$ et $(d')$.
  \item Recommence à partir de l'étape \ref{6parallelesexo57q2} avec de nombreux points $M$.
\end{myenumerate}
L'ensemble de ces croix vertes forme une courbe appelée {\em cardioïde}.
