%@P:exocorcp
%@metapost:604dm08.mp
%@Dif:3
\begin{myenumerate}
\item Construis en vraie grandeur le triangle dessiné ci-dessous à
  main levée, sachant que $TI=8$~cm et $RI=7$~cm.
\[\includegraphics{604dm08.2}\]
\item Que peut-on dire du triangle $RTI$ ? {\em Explique pourquoi.}
\item On appelle $(d_1)$, la droite passant par $I$ et perpendiculaire
  à la droite $(RI)$.\\On appelle $(d_2)$, la droite passant par $R$
  et perpendiculaire à la droite $(TI)$.\\Les droites $(d_1)$ et
  $(d_2)$ se coupent au point $E$.
\item Démontre que les droites $(EI)$ et $(TR)$ sont parallèles.
\item On appelle $(d_3)$, la droite passant par $I$ et parallèle à la
  droite $(RE)$. Les droites $(d_3)$ et $(TR)$ se coupent en $K$.
  \\On appelle $(d_4)$, la droite passant par $T$ et parallèle à la
  droite $(RE)$.
\item Démontre que les droites $(d_4)$ et $(IK)$ sont parallèles.
\end{myenumerate}
%@Commentaire: Reprise de l'exercice \verb+exo10+. Double utilisation des propriétés de démonstration sur les droites parallèles et perpendiculaires.
%@Correction:
\begin{myenumerate}
  \setcounter{enumi}{1}
\item Comme les droites $(TR)$ et $(RI)$ sont perpendiculaires alors
  le triangle $RTI$ est rectangle en $R$.
  \setcounter{enumi}{3}
\item Je sais que les droites $(EI)$ et $(RI)$ sont perpendiculaires
  et je sais aussi que les droites $(TR)$ et $(RI)$ sont
  perpendiculaires. Alors je peux conclure que les droites $(EI)$ et
  $(TR)$ sont parallèles.
  \setcounter{enumi}{5}
\item Je sais que les droites $(IK)$ et $(RE)$ sont parallèles
  et je sais aussi que les droites $(d_4)$ et $(RE)$ sont
  parallèles. Alors je peux conclure que les droites $(d_4)$ et
  $(IK)$ sont parallèles.
\end{myenumerate}