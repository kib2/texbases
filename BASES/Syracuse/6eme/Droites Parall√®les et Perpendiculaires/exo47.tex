%@Titre:Graduer une droite sans utiliser les graduations de la règle.
Trace une droite et marque deux points $A$ et $B$ sur cette droite. Les
points $A$ et $B$ devront être assez éloignés.
\begin{quote}
  \begin{center}
    \psshadowbox{
      \begin{minipage}{1.0\linewidth}
        {\em Objectif} : Trouver une longueur pour graduer cette
        droite sachant que $A$ sera l'origine et l'abscisse de $B$
        sera 7.
      \end{minipage}
      }
  \end{center}
\end{quote}
Pour cela :
\begin{itemize}
\item Trace une droite $(d)$ passant par le point $A$.
\item Choisis un rayon {\em quelconque} avec ton compas et reporte, à
  partir de $A$, sept fois ce rayon sur la droite $(d)$. Les points
  ainsi obtenus s'appellent $A_1$, $A_2$, $A_3$, $A_4$, $A_5$, $A_6$
  et $A_7$.
\item Trace la droite $(A_7B)$.
\item Trace la droite $(d_1)$, parallèle à la droite $(A_7B)$ et
    passant par $A_1$. Elle coupe la droite $(AB)$ en $C$.
\item Trace la droite $(d_2)$, parallèle à la droite $(A_7B)$ et
    passant par $A_2$.
  \begin{quote}
    \psframebox{
      \begin{minipage}{1.0\linewidth}
        \begin{myenumerate}
          \item Que peux-tu dire des droites $(d_1)$ et $(d_2)$ ?
            Quelle propriété as-tu utilisée ?
            \par\dotfill\par\dotfill\par\dotfill\par\dotfill\par\dotfill
        \end{myenumerate}
      \end{minipage}
      }
  \end{quote}
\item Trace la droite $(d_3)$, parallèle à la droite $(d_1)$ et
  passant par $A_3$.
  \begin{quote}
    \psframebox{
      \begin{minipage}{1.0\linewidth}
        \begin{myenumerate}
          \setcounter{enumi}{1}
          \item Que peux-tu dire des droites $(A_7B)$ et $(d_3)$ ?
            Quelle propriété as-tu utilisée ?
            \par\dotfill\par\dotfill\par\dotfill\par\dotfill\par\dotfill
        \end{myenumerate}
      \end{minipage}
      }
  \end{quote}
\item Trace la droite $(d_4)$, parallèle à la droite $(d_3)$ et
  passant par $A_4$.
\item Trace la droite $(d_5)$, parallèle à la droite $(d_3)$ et
  passant par $A_5$.
  \begin{quote}
    \psframebox{
      \begin{minipage}{1.0\linewidth}
        \begin{myenumerate}
          \setcounter{enumi}{2}
          \item Que peux-tu dire des droites $(A_7B)$ et $(d_3)$ ?
            Quelle propriété as-tu utilisée ?
            \par\dotfill\par\dotfill\par\dotfill\par\dotfill\par\dotfill
        \end{myenumerate}
      \end{minipage}
      }
  \end{quote}
\item Trace la droite $(d_6)$, parallèle à la droite $(A_7B)$ et
  passant par $A_6$.
\end{itemize}
{\em Si ta construction est correcte et très précise, la longueur $AC$
  permet de graduer la droite $(AB)$ tel que $A$ soit l'origine et
  l'abscisse de $B$ soit 7.}
