%@Titre:La versiera
%@Auteur:\url{http://www.animath.fr/UE/missenard/courbes.html}
%@geogebra:6parallelesexo56.ggb
%@Danger:Réalise ces courbes sur grande feuille, avec soin et précision.
\begin{myenumerate}
  \item Trace un cercle de diamètre $[OA]$ et la droite $(d)$
    perpendiculaire à la droite $(OA)$ passant par $A$.
  \item\label{6parallelesexo56q2} Une droite quelconque passant par $O$ 
recoupe le cercle en
    un point $N$ et coupe la droite $(d)$ en un point $P$.
  \item La parallèle à la droite $(OA)$ passant par $P$ et la
    parallèle à la droite $(d)$ passant par $N$ se coupent en un point $M$.
  \item Recommence à partir de l'étape \ref{6parallelesexo56q2}
    , en prenant une autre droite passant par $O$, pour construire
    un nouveau point $M$.
\end{myenumerate}
Recommence cela de nombreuses fois pour voir apparaître la courbe qui
est l'ensemble de ces points $M$. Cette courbe a pour nom {\em
  versiera}.
