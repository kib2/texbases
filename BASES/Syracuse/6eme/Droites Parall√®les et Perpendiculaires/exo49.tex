%@Dif:4
\begin{myenumerate}
  \item Trace un triangle $ABC$ tel que $BC=10$~cm; $AB=7$~cm et $AC=9$~cm.
  \item Sur le segment $[BC]$, place le point $I$ tel que $BI=2$~cm.
  \item
    \begin{enumerate}
    \item Trace la perpendiculaire à la droite $(AB)$ passant par
      $I$. Elle coupe la droite $(AB)$ en $J$.
    \item Trace la perpendiculaire à la droite $(BC)$ passant par
      $I$. Elle coupe la droite $(AB)$ en $K$.
    \item Trace la perpendiculaire à la droite $(AB)$ passant par
      $K$. Elle coupe la droite $(BC)$ en $L$.
    \item Trace la perpendiculaire à la droite $(BC)$ passant par
      $L$. Elle coupe la droite $(AC)$ en $M$.
    \item Trace la perpendiculaire à la droite $(AC)$ passant par
      $K$. Elle coupe la droite $(AC)$ en $N$.
    \end{enumerate}
  \item
    \begin{enumerate}
    \item Trace la parallèle à la droite $(AB)$ passant par $I$. Elle
      coupe la droite $(AC)$ en $O$.
    \item Trace la parallèle à la droite $(AB)$ passant par $L$. Elle
      coupe la droite $(AC)$ en $P$.
    \end{enumerate}
\end{myenumerate}
\ding{33}\dotfill
\begin{myenumerate}
  \item Que peut-on dire des droites $(LM)$ et $(IK)$ ? Explique pourquoi.
  \item Que peut-on dire des droites $(LP)$ et $(IO)$ ? Explique pourquoi.
  \item Que peut-on dire des droites $(IJ)$ et $(LK)$ ? Explique pourquoi.
\end{myenumerate}
%@Commentaire: Exercice destiné à un approfondissement. Donné lors de la correction d'un devoir pour les élèves ayant très bien réussi. L'exercice est distribué en deux parties : la construction puis, une fois celle-ci effectuée, les démonstrations.