%@Auteur: D'après IREM Strasbourg
\begin{myenumerate}
  \item Trace un segment $[AB]$ de 10~cm de longueur.
  \item Trace une droite $(d_1)$ passant par $A$.
  \item Trace la droite $(\Delta_1)$ passant par le point $B$ et perpendiculaire à la droite $(d_1)$.
  \item Appelle $C_1$ le point d'intersection des droites $(d_1)$ et $(\Delta_1)$.
  \item Trace une droite $(d_2)$ passant par $A$.
  \item Trace la droite $(\Delta_2)$ passant par le point $B$ et perpendiculaire à la droite $(d_2)$.
  \item Appelle $C_2$ le point d'intersection des droites $(d_2)$ et $(\Delta_2)$.
  \item Recommence plusieurs fois les étapes 2 à 5.
\end{myenumerate}
{\em Si la figure est bien faite, les points $C$ sont sur un même cercle. Trace ce cercle, précise son centre et son rayon.}