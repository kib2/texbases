%@P:exocorcp
%@geogebra:6symetrieexo33.ggb
%@metapost:6symetrieexo33.mp
%@Dif:2
Sur le cercle ci-dessous, on a placé quatre points $A$, $B$, $C$, $M$.
\\Construis, {\em sur la figure ci-dessous} :
\begin{itemize}
\item le point $M_1$ symétrique de $M$ par rapport à la droite $(AB)$;
\item le point $M_2$ symétrique de $M$ par rapport à la droite $(BC)$;
\item le point $M_3$ symétrique de $M$ par rapport à la droite $(CA)$.
\end{itemize}
Que remarque-t-on pour les points $M_1,M_2,M_3$ ?
\[\includegraphics{6symetrieexo33.1}\]
%@Correction:
\[\includegraphics{6symetrieexo33c.1}\]
Les points $M_1$, $M_2$ et $M_3$ sont alignés. (Ils forment une droite appelée {\em droite de Simpson} associée au point $M$.)