%@Auteur: François Meria\par
On donne la figure suivante :
\begin{center}
 \psset{unit=1cm}
    \pspicture(18,12)
    \psline(0,5)(18,9)
    \pstGeonode[PointSymbol=+](4,8){S}
    \pstGeonode[PointName=none,PointSymbol=none](4,5){K}
    \pstCircleOA{S}{K}
    \pstGeonode[PointSymbol=+,PosAngle=90](11,10){A}
    \pstGeonode[PointSymbol=none,PosAngle=45](13,8){B}
    \pstLineAB[nodesepB=-4]{A}{B}
    \put(17,9.5){$(d)$}
    \endpspicture
\end{center}
Construire sur cette feuille les figures symétriques du cercle de
centre $S$ et de la demi-droite $[AB)$ par rapport à la droite
$(d)$. Pour les constructions, on utilisera le compas et on
laissera apparaître les traits de construction.