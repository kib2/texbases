%@P:exocorcp
\begin{center}
  \begin{VF}{Nombres décimaux}
    La fraction décimale $\dfrac1{10}$ se lit {\em un dixième}.&\vr&\fa\\
    Dans le nombre 1\,238; le chiffre 8 est le chiffre des
    unités.&\vr&\fa\\
    Dans le nombre 1,238; le chiffre 8 est le chiffre des unités.&\fa&\vr\\
    On peut écrire $2,56=2+\dfrac{56}{10}$.&\fa&\vr\\
    On peut écrire $3,14=\dfrac{314}{1\,000}$.&\fa&\vr\\
  \end{VF}
\end{center}
%@Correction:
\begin{center}
  \begin{VFcor}{Nombres décimaux}
    La fraction décimale $\dfrac1{10}$ se lit {\em un dixième}.&\vr&\fa\\
    Dans le nombre 1\,238; le chiffre 8 est le chiffre des
    unités.&\vr&\fa\\
    Dans le nombre 1,238; le chiffre 8 est le chiffre des unités.&\fa&\vr\\
    On peut écrire $2,56=2+\dfrac{56}{10}$.&\fa&\vr\\
    On peut écrire $3,14=\dfrac{314}{1\,000}$.&\fa&\vr\\
  \end{VFcor}
\end{center}