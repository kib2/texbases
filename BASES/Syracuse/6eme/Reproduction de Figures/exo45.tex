%@Auteur: François Meria\par
\begin{multicols}{2}
    \begin{center}
    \psset{unit=0.4cm}
        \pspicture(15,15)
            \pstGeonode[PointSymbol=none,PosAngle={235,-45,45,135,0,90}](0,0){D}(15,0){C}(15,15){B}(0,15){A}(15,7.5){J}(7.5,15){I}
            \pspolygon(A)(B)(C)(D)
            \pstInterLL[PointSymbol=none,PosAngle=-118]{A}{C}{B}{D}{O}
            \psline(A)(C)
            \psline(B)(D)
            \pstGeonode[PointSymbol=none,PointName=none](7.5,0){I_1}(0,7.5){J_1}
            \psline(J)(J_1)
            \psline(I)(I_1)
            \pstBissectBAC[linestyle=none,PointSymbol=none,PointName=none]{D}{A}{O}{M_1}%
            \pstInterLL[PointName=none,PointSymbol=none]{A}{M_1}{J}{O}{T_1}%
            \psline(A)(T_1)
            \pstBissectBAC[linestyle=none,PointSymbol=none,PointName=none]{O}{A}{B}{M_2}%
            \pstInterLL[PointName=none,PointSymbol=none]{A}{M_2}{I}{O}{T_2}%
            \psline(A)(T_2)
            \psline(T_1)(T_2)
            \pstMarkAngle[MarkAngleRadius=1.2]{T_1}{A}{O}{}
            \pstMarkAngle[MarkAngleRadius=1.4]{T_1}{A}{O}{}
            \pstMarkAngle[MarkAngleRadius=1.4]{O}{A}{T_2}{}
            \pstMarkAngle[MarkAngleRadius=1.6]{O}{A}{T_2}{}

            \pstBissectBAC[linestyle=none,PointSymbol=none,PointName=none]{B}{O}{I}{N_1}%
            \pstInterLL[PointName=none,PointSymbol=none]{O}{N_1}{I}{B}{P_1}%
            \psline(O)(P_1)
            \pstBissectBAC[linestyle=none,PointSymbol=none,PointName=none]{J}{O}{B}{N_2}%
            \pstInterLL[PointName=none,PointSymbol=none]{O}{N_2}{J}{B}{P_2}%
            \psline(O)(P_2)
            \psline(P_1)(P_2)
            \pstMarkAngle[MarkAngleRadius=1.2]{B}{O}{P_1}{}
            \pstMarkAngle[MarkAngleRadius=1.4]{B}{O}{P_1}{}
            \pstMarkAngle[MarkAngleRadius=1.4]{P_2}{O}{B}{}
            \pstMarkAngle[MarkAngleRadius=1.6]{P_2}{O}{B}{}
%fin de la première partie de la construction
            \put(6.2,-1){Figure 1}
        \endpspicture
\end{center}
\par\columnbreak\par
\begin{enumerate}[1.]
  \item
    \begin{enumerate}[(a)]
    \item Tracer un carré $ABCD$ de $15$~cm de côté et ses quatre axes
de symétrie. Appeler $O$ leur point d'intersection.
    \item Placer le point $I$, milieu du segment $[AB]$ et le point
$J$, milieu du segment $[BC]$.
    \item Construire les bissectrices des angles $\widehat{OAD}$,
$\widehat{OAB}$, $\widehat{IOB}$ et $\widehat{JOB}$.
     \item Compléter la construction pour obtenir la figure 1.
     \end{enumerate}
    \item
        \begin{enumerate}[(a)]
        \item Compléter la figure 1 par symétrie par rapport aux deux
diagonales du carré $ABCD$.
        \item Colorier la figure 2 à l'aide de deux couleurs que l'on
alternera.
        \end{enumerate}
\end{enumerate}
\end{multicols}
\begin{center}
    \psset{unit=1cm}
        \pspicture(15,15)
            \pstGeonode[PointSymbol=none,PointName=none,PosAngle={235,-45,45,135,0,90}](0,0){D}(15,0){C}(15,15){B}(0,15){A}(15,7.5){J}(7.5,15){I}
            \pspolygon(A)(B)(C)(D)
            \pstInterLL[PointSymbol=none,PointName=none,PosAngle=50]{A}{C}{B}{D}{O}
            \psline(A)(C)
            \psline(B)(D)
            \pstGeonode[PointSymbol=none,PointName=none](7.5,0){I_1}(0,7.5){J_1}
            \psline(J)(J_1)
            \psline(I)(I_1)
            \pstBissectBAC[linestyle=none,PointSymbol=none,PointName=none]{D}{A}{O}{M_1}%
            \pstInterLL[PointName=none,PointSymbol=none]{A}{M_1}{J}{O}{T_1}%
            \psline(A)(T_1)
            \pstBissectBAC[linestyle=none,PointSymbol=none,PointName=none]{O}{A}{B}{M_2}%
            \pstInterLL[PointName=none,PointSymbol=none]{A}{M_2}{I}{O}{T_2}%
            \psline(A)(T_2)
            \psline(T_1)(T_2)

            \pstBissectBAC[linestyle=none,PointSymbol=none,PointName=none]{B}{O}{I}{N_1}%
            \pstInterLL[PointName=none,PointSymbol=none]{O}{N_1}{I}{B}{P_1}%
            \psline(O)(P_1)
            \pstBissectBAC[linestyle=none,PointSymbol=none,PointName=none]{J}{O}{B}{N_2}%
            \pstInterLL[PointName=none,PointSymbol=none]{O}{N_2}{J}{B}{P_2}%
            \psline(O)(P_2)
            \psline(P_1)(P_2)

%fin de la première partie de la construction

            \pstOrtSym[PointName=none,PointSymbol=none]{B}{D}{T_1}{Q_1}
            \pstOrtSym[PointName=none,PointSymbol=none]{B}{D}{T_2}{Q_2}
            \pspolygon(Q_1)(Q_2)(C)

            \pstOrtSym[PointName=none,PointSymbol=none]{A}{C}{P_1}{R_1}
            \pstOrtSym[PointName=none,PointSymbol=none]{A}{C}{P_2}{R_2}
            \pspolygon(R_1)(R_2)(O)

%coloriage
            \pstInterLL[PointName=none,PointSymbol=none]{T_1}{T_2}{O}{A}{S}
            \pspolygon[fillstyle=solid,fillcolor=lightgray](A)(T_2)(S)
            \pspolygon[fillstyle=solid,fillcolor=lightgray](O)(T_1)(S)
            \pspolygon[fillstyle=solid,fillcolor=lightgray](A)(J_1)(T_1)

            \pstInterLL[PointName=none,PointSymbol=none]{P_1}{P_2}{O}{B}{S_1}
            \pspolygon[fillstyle=solid,fillcolor=lightgray](O)(P_1)(I)
            \pspolygon[fillstyle=solid,fillcolor=lightgray](O)(S_1)(P_2)
            \pspolygon[fillstyle=solid,fillcolor=lightgray](P_1)(S_1)(B)

            \pstInterLL[PointName=none,PointSymbol=none]{R_1}{R_2}{O}{D}{U}
            \pspolygon[fillstyle=solid,fillcolor=lightgray](O)(R_1)(U)
            \pspolygon[fillstyle=solid,fillcolor=lightgray](D)(U)(R_2)
            \pspolygon[fillstyle=solid,fillcolor=lightgray](R_2)(O)(I_1)

            \pstInterLL[PointName=none,PointSymbol=none]{Q_1}{Q_2}{O}{C}{U_1}
            \pspolygon[fillstyle=solid,fillcolor=lightgray](Q_1)(U_1)(C)
            \pspolygon[fillstyle=solid,fillcolor=lightgray](O)(U_1)(Q_2)
            \pspolygon[fillstyle=solid,fillcolor=lightgray](Q_2)(J)(C)

            \put(6.7,-0.7){Figure 2}
        \endpspicture
\end{center}