%@metapost:603constructions.mp
%@Titre: Hélice.
%@Dif:2
\begin{itemize} 
\item[$\square$] Trace deux droites perpendiculaires $(d_1)$ et
$(d_2)$ sécantes en $O$. Marque un point $A$ sur la droite $(d_1)$.
\item[$\square$] Reporte la longueur $OA$ à partir de $A$ pour obtenir
le point $B$ sur $(d_1)$, puis à partir de $B$ pour obtenir le point
$C$ et enfin à partir de $C$ pour obtenir le point $D$.
\item[$\square$] Reporte la distance $OA$ à partir de $O$ : sur la
droite $(d_2)$ en $E$, puis sur la droite $(d_1)$ en $F$ puis sur la
droite $(d_2)$ en $G$.
\item[$\square$] Reporte la distance $OB$ à partir de $O$ : sur la
droite $(d_2)$ en $H$, puis sur la droite $(d_1)$ en $I$ puis sur la
droite $(d_2)$ en $J$.
\item[$\square$] Reporte la distance $OC$ à partir de $O$ : sur la
droite $(d_2)$ en $K$, puis sur la droite $(d_1)$ en $L$ puis sur la
droite $(d_2)$ en $M$.
\item[$\square$] Reporte la distance $OD$ à partir de $O$ : sur la
droite $(d_2)$ en $P$, puis sur la droite $(d_1)$ en $Q$ puis sur la
droite $(d_2)$ en $R$.
\item[$\square$] Trace le demi-cercle de centre $A$ et de rayon $OA$
au dessus de la droite $(d_1)$. Trace le demi-cercle de centre $C$ et
de rayon $OA$ au dessous de la droite $(d_1)$. Trace le demi-cercle de
centre $B$ et de rayon $OB$ au dessus de la droite $(d_1)$.
\item[$\square$] Trace le demi-cercle de centre $E$ et de rayon $OE$ à
droite de la droite $(d_2)$. Trace le demi-cercle de centre $K$ et de
rayon $OE$ à gauche de la droite $(d_2)$. Trace le demi-cercle de
centre $H$ et de rayon $OH$ à droite de la droite $(d_2)$.
\item[$\square$] Trace le demi-cercle de centre $F$ et de rayon $OF$
au dessous de la droite $(d_1)$. Trace le demi-cercle de centre $L$ et
de rayon $OF$ au dessus de la droite $(d_1)$. Trace le demi-cercle de
centre $I$ et de rayon $OI$ au dessous de la droite $(d_1)$.
\item[$\square$] Trace le demi-cercle de centre $G$ et de rayon $OG$ à
gauche de la droite $(d_2)$. Trace le demi-cercle de centre $M$ et de
rayon $OG$ à droite de la droite $(d_2)$. Trace le demi-cercle de
centre $J$ et de rayon $OJ$ à gauche de la droite $(d_2)$.
\item[$\square$] Efface les traits de construction et les noms des
différents points. Colorie la figure.
\end{itemize}
\[\includegraphics[scale=0.25]{603constructions.10}\]
%@Commentaire: \og{}Belle figure\fg{} à construire.