%@Auteur: François Meria\par
\begin{center}
\psset{unit=0.75cm}
    \pspicture(-9,-10.4)(9,7)
        \pstGeonode[PointSymbol=none,PosAngle={0},PointName=none](0,0){O}
        \pstGeonode[PointSymbol=none,PosAngle={0},PointName=none](0,-2){N}
        \pstGeonode[PointSymbol=none,PosAngle={0},PointName=none](0,0.3){O_1}
        \pstGeonode[PointSymbol=none,PosAngle={0},PointName=none](0,0.6){O_2}
        \pstGeonode[PointSymbol=none,PosAngle={0},PointName=none](0,0.9){O_3}
        \pstGeonode[PointSymbol=none,PosAngle={0},PointName=none](0,1.2){O_4}
        \pstGeonode[PointSymbol=none,PosAngle={0},PointName=none](0,1.5){O_5}
        \pstGeonode[PointSymbol=none,PosAngle={0},PointName=none](0,1.8){O_6}
        \pstGeonode[PointSymbol=none,PosAngle={0},PointName=none](0,2.1){O_7}
        \pstGeonode[PointSymbol=none,PosAngle={0},PointName=none](0,2.4){O_8}
        \pstGeonode[PointSymbol=none,PosAngle={90},PointName=none](0,6.8){P}
        \pstGeonode[PointSymbol=none,PosAngle={0},PointName=none](0,2.1){Q}
        \pstGeonode[PointSymbol=none,PosAngle={0},PointName=none](0,1.8){K_1}
        \pstGeonode[PointSymbol=none,PosAngle={0},PointName=none](0,1.5){K_2}
        \pstGeonode[PointSymbol=none,PosAngle={1},PointName=none](0,1.2){K_3}
        \pstGeonode[PointSymbol=none,PosAngle={2},PointName=none](0,0.9){K_4}
        \pstGeonode[PointSymbol=none,PosAngle={3},PointName=none](0,0.6){K_5}
        \pstGeonode[PointSymbol=none,PosAngle={4},PointName=none](0,0.3){K_6}
        \pstGeonode[PointSymbol=none,PosAngle={5},PointName=none](0,0){K_7}
        \pstGeonode[PointSymbol=none,PosAngle={6},PointName=none](0,-0.3){K_8}
        \pstGeonode[PointSymbol=none,PosAngle={7},PointName=none](0,-0.6){K_9}
        \pstGeonode[PointSymbol=none,PosAngle={8},PointName=none](0,-0.9){L_1}
        \pstGeonode[PointSymbol=none,PosAngle={9},PointName=none](0,-1.2){L_2}
        \pstGeonode[PointSymbol=none,PosAngle={10},PointName=none](0,-1.5){L_3}
        \pstGeonode[PointSymbol=none,PosAngle={11},PointName=none](0,-1.8){L_4}
        \pstGeonode[PointSymbol=none,PosAngle={12},PointName=none](0,-2.1){L_5}
        \pscircle[fillstyle=solid,fillcolor=black](L_4){8.6}
        \pscircle[fillstyle=solid,fillcolor=white](L_3){8.3}
        \pscircle[fillstyle=solid,fillcolor=black](L_2){8}
        \pscircle[fillstyle=solid,fillcolor=white](L_1){7.7}
        \pscircle[fillstyle=solid,fillcolor=black](K_9){7.4}
        \pscircle[fillstyle=solid,fillcolor=white](K_8){7.1}
        \pscircle[fillstyle=solid,fillcolor=black](K_7){6.8}
        \pscircle[fillstyle=solid,fillcolor=white](K_6){6.5}
        \pscircle[fillstyle=solid,fillcolor=black](K_5){6.2}
        \pscircle[fillstyle=solid,fillcolor=white](K_4){5.9}
        \pscircle[fillstyle=solid,fillcolor=black](K_3){5.6}
        \pscircle[fillstyle=solid,fillcolor=white](K_2){5.3}
        \pscircle[fillstyle=solid,fillcolor=black](K_1){5}
        \pscircle[fillstyle=solid,fillcolor=white](Q){4.7}
        \pscircle[fillstyle=solid,fillcolor=black](O_8){4.4}
        \pscircle[fillstyle=solid,fillcolor=white](O_7){4.1}
        \pscircle[fillstyle=solid,fillcolor=black](O_6){3.8}
        \pscircle[fillstyle=solid,fillcolor=white](O_5){3.5}
        \pscircle[fillstyle=solid,fillcolor=black](O_4){3.2}
        \pscircle[fillstyle=solid,fillcolor=white](O_3){2.9}
        \pscircle[fillstyle=solid,fillcolor=black](O_2){2.6}
        \pscircle[fillstyle=solid,fillcolor=white](O_1){2.3}
        \pscircle[fillstyle=solid,fillcolor=black](O){2}
    \endpspicture
\end{center}

\textbf{Programme de construction }

{\texttt{Toutes les constructions doivent se faire au COMPAS
et à la règle (sans utiliser les graduations sauf pour reporter les longueurs).}}

\shadowbox{
\begin{minipage}[c]{\textwidth}
\begin{multicols}{2}\setlength{\columnseprule}{0.5pt}
\textbf{\'Etape 1.}\\
Construire une droite $(d)$ verticale au centre de la feuille, et placer un point $O$ sur cette droite.\\

\textbf{\'Etape 2.}\\
Tracer le cercle ${\cal C}$ de centre $O$ et de rayon $2$~cm. Le point au sud de $O$ à l'intersection
de ${\cal C}$ et $(d)$ est noté $N$.\\

\textbf{\'Etape 3.}\\
Placer les points $O_1$, $O_2$, \ldots, $O_8$ au dessus de $O$ sur la droite $(d)$, espacés de $3$~mm.
Tracer les cercles ayant ces points comme centre et passant par $N$. Le dernier cercle de centre $O_8$ recoupe
 la droite $(d)$ au point $P$.\\

\textbf{\'Etape 4.}\\
Placer un point $O'$ sur $(d)$ entre $O$ et $P$ tel que $NO'=4,1$~cm. Tracer le cercle de centre $O'$ passant par $P$.\\

\textbf{\'Etape 5.}\\
En descendant de $3$~mm les centres toujours au-dessous de $O'$, tracer ainsi $14$ cercles  toujours passant par $P$.\\

\textbf{\'Etape 6.}\\
Colorier la figure obtenue en alternant deux couleurs.\\
\end{multicols}
\setlength{\columnseprule}{0pt}
\end{minipage}
}