%@metapost:603constructions.mp
%@Titre: \'Etoile à six branches incurvées.
%@Dif:2
\begin{itemize}
\item[$\square$] Trace deux droites perpendiculaires $(d)$ et $(d_1)$. Elles se coupent en $O$.
\item[$\square$] Sur la droite $(d)$, place deux points $A$ et $G$ tel que $AO=7$~cm et $GO=7$~cm.
\item[$\square$] Sur la droite $(d_1)$, place deux points $J$ et $D$ tel que $JO=7$~cm et $DO=7$~cm.
\item[$\square$] Reporte, sur le cercle à partir de $A$, six fois le rayon : on obtient les points $C,E,I,K$. Vérifie que l'on passe en $G$ et $A$.
\item[$\square$] Reporte, sur le cercle à partir de $D$, six fois le rayon : on obtient les points $F,H,L,B$. Vérifie que l'on passe en $J$ et $D$.
\item[$\square$] Trace les triangles $AEI$ et $CGK$.
\item[$\square$] Trace les arcs de cercles de centre $A$, $C$, $E$,\ldots de rayon 7~cm qui vont de $K$, $A$, $C$,\ldots à $O$.
\item[$\square$] Trace les arcs de cercles de centre $B$, $D$, $F$,\ldots de rayon $AB$ qui vont de $A$, $C$, $E$,\ldots aux côtés des deux triangles. On obtient les points $M$, $N$, $P$,\ldots
\item[$\square$] Trace les arcs de cercles de centres $F$, $H$, $J$,\ldots de rayon $FC$ qui vont de $B$ à $M$, $E$ à $N$,\ldots
\item[$\square$] Trace les arcs de cercles de centres $M$, $N$, $P$,\ldots de rayon $MO$ qui vont de $P$, $M$, $N$,\ldots à $O$.
\item[$\square$] Efface les noms des différents points et droites puis quelques traits de construction\ldots
\item[$\square$] \ldots puis colorie la figure.
\[\includegraphics[scale=0.6]{603constructions.5}\]
\end{itemize}
%@Commentaire: \og{}Belle figure\fg{} à construire.