%@metapost:isoechiquier.mp
%@Dif:2
\begin{myenumerate}
\item Construis un triangle $ABC$ isocèle en $A$ tel que $BC=13$~cm et $AB=AC=16$~cm.
\item Place sur le segment $[AB]$, à partir du point $A$, des points tous les 16 millimètres.
\item Joins tous les points obtenus au point $C$.
\item Place sur le segment $[AC]$, à partir du point $A$, des points tous les 8 millimètres.
\item Joins tous les points obtenus au point $B$.
\item Colorie.
\end{myenumerate}
Voilà, en réduction, ce que l'on peut obtenir.
\[\includegraphics{isoechiquier.1}\]
%@Commentaire: La construction n'est pas difficile mais elle demande beaucoup de précision.