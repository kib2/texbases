%@metapost:603constructions.mp
%@Titre: Les trois lunes.
%@Dif:2
\begin{cursive}
\begin{itemize}
\item[$\square$] Trace un cercle de centre $O$. Marque un point $A$
sur le cercle, puis reporte cinq fois le rayon à partir de $A$, dans
le sens des aiguilles d'une montre, pour marquer les points $B$, $C$,
$D$, $E$ et $F$.
\item[$\square$] Trace les segments $[AC]$ et $[OB]$ sécants en
$G$. Trace les segments $[EC]$ et $[OD]$ sécants en $H$. Trace les
segments $[EA]$ et $[OF]$ sécants en $I$.
\item[$\square$] Trace l'arc de cercle de centre $B$, de rayon $BO$,
de $C$ à $A$, à l'intérieur du cercle.
\item[$\square$] Trace l'arc de cercle de centre $D$, de rayon $DO$,
de $E$ à $C$, à l'intérieur du cercle.
\item[$\square$] Trace l'arc de cercle de centre $F$, de rayon $FO$,
de $A$ à $E$, à l'intérieur du cercle.
\item[$\square$] Trace l'arc de cercle de centre $G$, de rayon $GC$,
de $C$ à $A$, à l'intérieur du cercle.
\item[$\square$] Trace l'arc de cercle de centre $H$, de rayon $HE$,
de $E$ à $C$, à l'intérieur du cercle.
\item[$\square$] Trace l'arc de cercle de centre $I$, de rayon $IA$,
de $A$ à $E$, à l'intérieur du cercle.
\item[$\square$] Efface les noms des différents points puis quelques
traits de construction\ldots
\item[$\square$] \ldots et colorie les trois lunes.
\end{itemize}
\end{cursive}
\[\includegraphics[scale=0.85]{603constructions.8}\]
%@Commentaire: \og{}Belle figure\fg{} à construire.