%@metapost:603constructions.mp
%@Titre: Le triangle celtique.
%@Dif:2
\begin{cursive}
\begin{itemize}
\item[$\square$] Trace un cercle $\cal C$ de centre $O$ et de rayon
  7~cm.
\item[$\square$] Trace un diamètre $[AD]$ de ce cercle.
\item[$\square$] Le cercle de centre $A$ et de rayon 7~cm coupe le
  cercle $\cal C$ en $B$ et $F$.
\item[$\square$] Le cercle de centre $D$ et de rayon 7~cm coupe le
  cercle $\cal C$ en $C$ et $E$.
\item[$\square$] Trace le cercle de centre $O$ et de  rayon 6~cm. Il
  coupe les diamètres en $A_1$, $B_1$, $C_1$,\ldots
\item[$\square$] Trace le cercle de centre $O$ et de  rayon 5~cm. Il
  coupe les diamètres en $A_2$, $B_2$, $C_2$,\ldots
\item[$\square$] Trace l'arc de cercle de centre $D_2$ allant de $C$ à
  $E$.
\item[$\square$] Trace l'arc de cercle de centre $D_2$ allant de $C_1$
  à $E_1$.
\item[$\square$] Recommence les deux étapes précédentes avec les
  points $F_2$ et $B_2$ pour centre.
\item[$\square$] Efface les noms des différents points et droites puis
  quelques traits de construction\ldots
\item[$\square$] \ldots puis colorie la figure.
\end{itemize}
\end{cursive}
\[\includegraphics[scale=0.75]{603constructions.2}\]
%@Commentaire: \og{}Belle figure\fg{} à construire.