%@Auteur: François Meria\par
\vspace*{1cm}\par
\begin{multicols}{2}
On veut construire la figure ci-contre. Pour cela voici le
programme de construction. Construire la figure correspondant au
programme de construction suivant. Toutes les longueurs données
sont exprimées en cm. \textit{On laissera apparaître tous les
traits de construction}.

\columnbreak

\begin{center}
\psset{unit=0.5cm} \pspicture(2,5)(12,10)
    \pstGeonode[PointSymbol=none,PosAngle={180,0}](2,7){R}(8,7){S}
    \pstGeonode[PointName=none,PointSymbol=none](3,7){Z}(5,7){Q}(7,7){W}(9,7){N}(11,7){M}(13,7){P}
    \pstInterCC[PointName=none,PointSymbol=none]{S}{R}{R}{Z}{A}{B}
    \pstInterCC[PointName=none,PointSymbol=none]{S}{R}{R}{Q}{C}{D}
    \pstInterCC[PointName=none,PointSymbol=none]{S}{R}{R}{W}{E}{F}
    \pstInterCC[PointName=none,PointSymbol=none]{S}{R}{R}{N}{G}{H}
    \pstInterCC[PointName=none,PointSymbol=none]{S}{R}{R}{M}{I}{J}
    \pstInterCC[PointName=none,PointSymbol=none]{S}{R}{R}{P}{K}{L}

    \pstLineAB{R}{S} \pstLineAB{A}{R} \pstLineAB{A}{S}
    \pstLineAB{B}{R} \pstLineAB{B}{S} \pstLineAB{C}{R}
    \pstLineAB{C}{S} \pstLineAB{D}{R} \pstLineAB{D}{S}
    \pstLineAB{E}{R} \pstLineAB{E}{S} \pstLineAB{F}{R}
    \pstLineAB{F}{S} \pstLineAB{G}{R} \pstLineAB{G}{S}
    \pstLineAB{H}{R} \pstLineAB{H}{S} \pstLineAB{I}{R}
    \pstLineAB{I}{S} \pstLineAB{J}{R} \pstLineAB{J}{S}
    \pstLineAB{K}{R} \pstLineAB{K}{S} \pstLineAB{L}{R}
    \pstLineAB{L}{S}
\endpspicture
\end{center}
\end{multicols}

\begin{enumerate}[(a)]
    \item Tracer un segment $[RS]$ tel que $RS=6$.
    \item Tracer le cercle $\mathcal{C}$ de centre $S$ et de rayon 6.
    \item Tracer le cercle $\mathcal{C}_1$ de centre $R$ et de rayon 1.\\
    Placer les points $A$ et $B$ qui sont les points d'intersection de $\mathcal{C}_1$ avec $\mathcal{C}$.
    \item Tracer le cercle $\mathcal{C}_2$ de centre $R$ et de rayon 3.\\
    Placer les points $C$ et $D$ qui sont les points d'intersection de $\mathcal{C}_2$ avec $\mathcal{C}$.
    \item Tracer le cercle $\mathcal{C}_3$ de centre $R$ et de rayon 5.\\
    Placer les points $E$ et $F$ qui sont les points d'intersection de $\mathcal{C}_3$ avec $\mathcal{C}$.
    \item Tracer le cercle $\mathcal{C}_4$ de centre $R$ et de rayon 7.\\
    Placer les points $G$ et $H$ qui sont les points d'intersection de $\mathcal{C}_4$ avec $\mathcal{C}$.
    \item Tracer le cercle $\mathcal{C}_5$ de centre $R$ et de rayon 9.\\
    Placer les points $I$ et $J$ qui sont les points d'intersection de $\mathcal{C}_5$ avec $\mathcal{C}$.
    \item Tracer le cercle $\mathcal{C}_6$ de centre $R$ et de rayon 11.\\
    Placer les points $K$ et $L$ qui sont les points d'intersection de $\mathcal{C}_6$ avec $\mathcal{C}$.
    \item Tracer en rouge tous les triangles ayant pour sommets $R$, $S$ et l'un des points construits précédemment.
    Coder en vert les longueurs égales sur cette figure.
\end{enumerate}

\vskip 0.2cm

Que peut-on dire de la droite $(RS)$ pour cette figure géométrique
? On ne demande pas de justification.