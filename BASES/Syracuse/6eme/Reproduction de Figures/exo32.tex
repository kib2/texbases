%@metapost:etoilecinqbranches.mp
%@Titre: Construction d'une étoile à cinq branches.
%@Dif:2
\begin{myenumerate}
  \item {\em Sur une feuille blanche}, construis un cercle $\cal C$ de centre $O$ et de 10~cm de diamètre.
  \item Trace un diamètre $[AB]$.
  \item Trace une demi-droite d'origine $O$ perpendiculaire à la droite $(AB)$. Cette demi-droite coupe le cercle $\cal C$ en un point $K$. Place le point $K$.
  \item Place le point $I$, milieu du segment $[OB]$.
  \item Trace le cercle de diamètre $[OB]$. Ce cercle coupe le segment $[KI]$ en un point $J$.
  \item 
    \begin{enumerate}
    \item Trace le cercle de centre $K$ et de rayon $KJ$. Ce cercle coupe le cercle $\cal C$ en $D$ et $E$.
    \item Trace le cercle de centre $D$ passant par $E$. Ce cercle recoupe le cercle $\cal C$ en $F$.
    \item Trace le cercle de centre $E$ passant par $D$. Ce cercle recoupe le cercle $\cal C$ en $H$.
    \end{enumerate}
  \item Place le point $G$, diamétralement opposé au point $K$ sur le cercle $\cal C$.
  \item Trace les segments $[DG]$, $[GE]$, $[EF]$, $[FH]$ et $[HD]$ et colorie l'étoile obtenue.
\end{myenumerate}
\[\includegraphics{etoilecinqbranches.1}\]