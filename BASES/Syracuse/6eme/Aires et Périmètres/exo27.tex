%@metapost:remediationaireperimetre.mp
%@Dif:2
On considère un rectangle $\cal R$ de longueur $L$ et de largeur $\ell$. On note $\cal P$ le périmètre du rectangle $\cal R$ et $\cal A$ son aire.
\par\compo{2}{remediationaireperimetre}{1}{
Complète alors le tableau suivant:
\[\begin{tabular}{|c|c|c|c|}
\hline
Longueur $L$&Largeur $\ell$&Périmètre $\cal P$&Aire $\cal A$\\
\hline
7~cm&5~cm&&\\
\hline
8~cm&15~mm&&\\
\hline
10~cm&&30~cm&\\
\hline
&5~m&&30~m$^2$\\
\hline
&6~hm&&39~hm$^2$\\
\hline
\end{tabular}
\]
}
%@Commentaire: Travail sur le rectangle en remédiation.