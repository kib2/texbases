%@metapost:Evaluation1996exo21.mp
%@Auteur: \'Evaluation 1996
%@Dif:2
Observe les figures ci-dessous et coche toutes les phrases qui sont
justes.
\begin{myenumerate}
  \item\subitem{}\par
\begin{center}
\begin{tabular}{cc}
Figure 1&Figure 2\\
\includegraphics{Evaluation1996exo21.1}&\includegraphics{Evaluation1996exo21.2}\\
\end{tabular}
\end{center}
L'aire de la figure 1 est la plus grande\hfill$\square$
\par Les deux figures ont la même aire\hfill$\square$
\par L'aire de la figure 2 est la plus grande\hfill$\square$
\par\vspace{2mm}\par
Le périmètre de la figure 1 est le plus grand\hfill$\square$
\par Les deux figures ont le même périmètre\hfill$\square$
\par Le périmètre de la figure 2 est le plus grand\hfill$\square$
\item \subitem{}\par
\begin{center}
\begin{tabular}{cc}
Figure 3&Figure 4\\
\includegraphics{Evaluation1996exo21.3}&\includegraphics{Evaluation1996exo21.4}\\
\end{tabular}
\end{center}
L'aire de la figure 3 est la plus grande\hfill$\square$
\par Les deux figures ont la même aire\hfill$\square$
\par L'aire de la figure 4 est la plus grande\hfill$\square$
\par\vspace{2mm}\par
Le périmètre de la figure 3 est le plus grand\hfill$\square$
\par Les deux figures ont le même périmètre\hfill$\square$
\par Le périmètre de la figure 4 est le plus grand\hfill$\square$
\end{myenumerate}
%@Commentaire: Permet de poursuivre le travail sur la distinction des notions de périmètre et d'aire.