%@Auteur: François Meria\par
\begin{myenumerate}
    \item
    \begin{multicols}{2}
    \begin{center}
    \pspicture(5,4.5)
        \pstGeonode[PointSymbol=+,PosAngle={225,-45,-45}](0.5,0.5){A}(4.5,0.5){B}(2.5,0.5){D}
        \pcline{<->}(0.5,0)(2.5,0) \lput*{:U}{4~cm}
        \pstSegmentMark[SegmentSymbol=pstslashh]{A}{D}
        \pstSegmentMark[SegmentSymbol=pstslashh]{D}{B}
        \pstRotation[RotAngle=70,PointSymbol=+,PosAngle=160]{A}{D}{C}
        \pstSegmentMark[SegmentSymbol=pstslashh]{A}{C}
        \pstRotation[RotAngle=70,PointSymbol=+,PosAngle=90]{A}{B}{E}
        \pstSegmentMark[SegmentSymbol=pstslashh]{C}{E}
        \pstLineAB[linestyle=dashed]{C}{D}
        \pstLineAB[linestyle=dashed]{B}{E}
        \pstMarkAngle[arrows=->,MarkAngleRadius=0.7]{D}{A}{C}{}
        \put(1.2,1){$70^{\degres}$}
\endpspicture
    \end{center}
\begin{enumerate}[(a)]
        \item Construire la figure ci-contre en vraies grandeurs.
        \item Placer sur la figure construite le point $I$, milieu
        du segment $[CD]$, puis coder la figure.
        \item Placer sur la figure construite le point $J$, milieu
        du segment $[BE]$, puis coder la figure.
        \item Tracer la droite $(IJ)$.
    \end{enumerate}
\end{multicols}
    \item
        \begin{enumerate}
            \item Que peut-on dire des longueurs des segments $[AC]$
            et $[AD]$ ? Justifier.
            \item Que peut-on dire des longueurs $CI$ et $ID$ ? Justifier.
            \item En déduire que la droite $(AI)$ est la médiatrice du segment $[CD]$.
        \end{enumerate}
    \item En utilisant la question précédente, que peut-on dire de
    la droite $(AJ)$ par rapport au segment $[BE]$ ?
    \item Sur la figure précédente, coder les angles
    $\widehat{BAJ}$ et $\widehat{EAJ}$, puis mesurer ces angles.
\end{myenumerate}