\begin{myenumerate}
\item
\begin{enumerate}
\item Construis un triangle $ABC$ tel que $BC=7$~cm;
$\widehat{ABC}=60$\degres\ et $\widehat{BCA}=50$\degres.
\item Mesure l'angle $\widehat{BAC}$.
\end{enumerate}
\item
\begin{enumerate}
\item Construis la droite $(d)$, perpendiculaire à la droite $(BC)$ et
passant par $A$. La droite $(d)$ coupe la droite $(BC)$ en $H$.
\item Mesure les angles $\widehat{BAH}$ et $\widehat{HAC}$.
\end{enumerate}
\item
\begin{enumerate}
\item Construis la droite $(d_1)$, parallèle à la droite $(AB)$ et
passant par $H$.
\item Construis la droite $(d_2)$, parallèle à la droite $(AB)$ et
passant par $C$.
\item Prouve que les droites $(d_1)$ et $(d_2)$ sont parallèles.
\end{enumerate}
\item
\begin{enumerate}
\item Construis la droite $(d_3)$, perpendiculaire à la droite $(d_1)$
et passant par $H$. Elle coupe la droite $(d_2)$ en $I$.
\item Prouve que les droites $(d_2)$ et $(d_3)$ sont perpendiculaires.
\end{enumerate}
\item
\begin{enumerate}
\item Soit $J$ un point de la droite $(d_2)$ tel que $IJ=HI$.
\\Construis la droite $(d_4)$, perpendiculaire à la droite $(d_1)$
et passant par $J$. Elle coupe la droite $(d_1)$ en $K$.
\item Prouve que les droites $(d_3)$ et $(d_4)$ sont parallèles.
\end{enumerate}
\end{myenumerate}