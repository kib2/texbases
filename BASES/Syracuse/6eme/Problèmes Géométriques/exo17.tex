Soit $[AB]$ un segment de 4~cm et $(\cal C)$ est le cercle de
 centre $A$ et de rayon $AB$.\par La droite $(d_1)$ est la médiatrice
 du segment $[AB]$.  Elle coupe le segment $[AB]$ en $C$ et le cercle
 en $(\cal C)$ en $F$.
\begin{myenumerate}
\item Fais une figure que l'on complétera au fur et à mesure de
l'exercice.
\item Pourquoi le point $C$ est le milieu du segment $[AB]$ ?
\item Calcule la longueur $FB$.
\item La droite $(d_2)$ est la parallèle à la droite $(d_1)$ passant
par $I$, milieu du segment $[CB]$. La droite $(d_2)$ coupe le
demi-cercle de diamètre $[AB]$ en $G$.
\begin{enumerate}
\item Prouve que les droites $(d_2)$ et $(AB)$ sont perpendiculaires.
\item Déduis-en que la droite $(d_2)$ est la médiatrice du segment
$[CB]$.
\end{enumerate}
\item Calcule la longueur $BG$.
\end{myenumerate}