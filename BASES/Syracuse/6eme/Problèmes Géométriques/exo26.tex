%@metapost:6pbgeoexo26.mp
%@Auteur:d'après \url{maths-college.blogspot.com/}\par
On donne la figure ci-dessous {\em qui n'est pas en vraie
  grandeur}. Elle est constituée à partir de triangles équilatéraux
{\em uniquement}.
\[\includegraphics{6pbgeoexo26.1}\]
\begin{myenumerate}
  \item Reproduis, à l'aide des instruments, la figure en vraie grandeur.
  \item \'Ecris un programme de construction de cette figure.
  \item Calcule le périmètre de cette surface.
  \item En prenant le triangle \ding{172} comme unité d'aire,
    détermine l'aire de cette surface. {\em Tu justifieras clairement
      ton calcul}.
\end{myenumerate}