%@P:exocorcp
\begin{myenumerate}
  \item Trace un segment $[EF]$ tel que $EF=8$~cm.
  \item Trace le cercle ${\mathscr C}_1$ de centre $E$ et de rayon 2,5~cm.
  \item Trace le cercle ${\mathscr C}_2$ {\em de diamètre} $[EF]$; appelle $I$ son centre.
  \item Les cercles ${\mathscr C}_1$ et ${\mathscr C}_2$ se coupent aux points $A$ et $B$. Place ces deux points.
  \item
    \begin{enumerate}
    \item Quelle est la particularité du triangle $AEB$ ? Explique
      pourquoi.
    \item Est-ce que le quadrilatère $AEBI$ est un losange ? Explique
      {\em clairement} ta réponse.
    \item Quel est le périmètre du quadrilatère $AEBI$ ? Explique ton
      résultat par un calcul {\em détaillé}.
    \end{enumerate}
  \item
    \begin{enumerate}
      \item Construis un point $D$ sur le cercle ${\mathscr C}_2$ tel que $FD=4$~cm.
      \item Quelle est la particularité du triangle $IFD$ ? Explique
        pourquoi.
      \item Calcule le périmètre du triangle $IFD$.
    \end{enumerate}    
\end{myenumerate}
%@Commentaire: C'est une reprise de l'exercice \verb+exo67+ du dossier {\sl éléments de géométrie}. Adapté pour être donné dans un DS où il y a un mélange de plusieurs notions.
%@Correction:
\begin{myenumerate}
  \setcounter{enumi}{4}
  \item
    \begin{enumerate}
    \item Comme $[EA]$ et $[EB]$ sont des rayons du cercle ${\mathscr
        C}_1$ alors les longueurs $EA$ et $EB$ sont égales. Donc le
      triangle $AEB$ est isocèle en $E$.
    \item Les longueurs $AE$ et $AI$ ne sont pas égales. Donc le
      quadrilatère $AEBI$ ne peut pas être un losange.
    \item ${\mathscr P}=AE+EB+BI+IA=2,5+2,5+4+4=13$~cm.
    \end{enumerate}
  \item
    \begin{enumerate}
      \setcounter{enumii}{1}
    \item Comme $[IF]$ et $[ID]$ sont des rayons du cercle ${\mathscr
        C}_2$ alors $IF=ID=4$~cm.\\ Comme les longueurs $IF$, $ID$ et
      $FD$ sont égales alors le triangle $IFD$ est équilatéral.
    \item ${\mathscr P}=IF+FD+DI=4+4+4=12$~cm.
    \end{enumerate}
\end{myenumerate}