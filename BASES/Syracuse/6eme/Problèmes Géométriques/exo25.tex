%@P:exocorcp
%@metapost:6pbgeoexo25.mp
\compo{1}{6pbgeoexo25}{1}{Voici, ci-contre, une figure à main levée.
  \begin{myenumerate}
    \item \`A l'aide des codages sur la figure, indique la nature
      {\em précise} des triangles :
      \begin{description}
      \item[$ABC$] : je sais qu'il possède \dotfill\par donc c'est un
        triangle \dotfill
      \item[$ACD$] : c'est un triangle \dotfill
      \item[$DCE$] : je sais qu'il possède \dotfill\par donc c'est un
        triangle \dotfill
      \end{description}
  \end{myenumerate}
}
\begin{myenumerate}
  \setcounter{enumi}{1}
\item Reproduis ci-dessous, {\em à l'aide des instruments}, la figure.
\vspace{9cm}
\item Calcule le périmètre du polygone $ADECB$.
\end{myenumerate}
%@Correction:
  \begin{myenumerate}
    \item \begin{description}
      \item[$ABC$] : je sais qu'il possède {\cursive trois côtés de même
          longueur} donc c'est un
        triangle {\cursive équilatéral}.
      \item[$ACD$] : c'est un triangle {\cursive quelconque}.
      \item[$DCE$] : je sais qu'il possède {\cursive deux côtés de
          même longueur} donc c'est un triangle {\cursive isocèle}.
      \end{description}
      \setcounter{enumi}{2}
    \item \[\Eqalign{
        \mathscr P&=AD+DE+EC+CB+BA\cr
        \mathscr P&=8~\mbox{cm}+5~\mbox{cm}+5~\mbox{cm}+4~\mbox{cm}+4~\mbox{cm}\cr
        \mathscr P&=26~\mbox{cm}\cr
        }\]
  \end{myenumerate}