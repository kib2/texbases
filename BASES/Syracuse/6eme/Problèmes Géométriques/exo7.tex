\begin{myenumerate}
\item Construis un rectangle $ABCD$ tel que $AB=8$~cm et $AD=4$~cm.
\item Quel est le périmètre du rectangle $ABCD$ et quelle est son aire ?
\item
\begin{enumerate}
\item Soit $I$ le milieu du segment $[AD]$. \`A l'extérieur du rectangle
$ABCD$, construis le demi-cercle de centre $I$ et de rayon 2~cm.\\\`A
l'extérieur du rectangle $ABCD$, construis le demi-cercle de diamètre
$[AB]$. L'ensemble de la figure constitue une surface appelée ${\cal
S}_1$.
\item Calcule le périmètre de cette surface ${\cal S}_1$.
\end{enumerate}
\item
\begin{enumerate}
\item Construis la droite $(d)$, parallèle à la droite $(AB)$ passant
par $I$.
\item Que peux-tu dire des droites $(d)$ et $(AD)$ ? Justifie la
réponse.
\end{enumerate}
\item
\begin{enumerate}
\item Soit $K$ le milieu du segment $[OB]$. La parallèle à la droite
$(BC)$ passant par $K$ coupe la droite $(d)$ en $P$.
\item Prouve que les droites $(KP)$ et $(AB)$ sont perpendiculaires.
\end{enumerate}
\item
\begin{enumerate}
\item Quelle est la nature du quadrilatère $AKPI$ ? Justifie la
réponse.
\item Quelle est l'aire de la surface $AKPI$ ?
\end{enumerate}
\end{myenumerate}