%@Dif:1
\begin{myenumerate}
\item Dessine, sur du papier millimétré, un repère d'origine $O$ avec
pour unité de longueur le centimètre.
\item Place les points suivants
$A(-1;4)$; $B(-3;1)$; $C(0;-4)$; $D(2;-4)$; $E(5;1);$
\par$\,F(3;4)$; $G(-1;2)$; $H(0;2)$; $I(0;1)$; $J(-1;1)$; $K(2;2);
\,L(3;2)$; $M(3;1);$\par$\,N(2;1)$; $P(1;1)$; $Q(1;0)$; $R(-1;-2);
\,S(0;-1)$; $T(2;-1)$; $U(3;-2);$
\par$\,V(0;-3)$; $W(2;-3)$; $X(0;-2)$; $Y(2;-2)$
\item Relie les points $A$, $B$, $C$, $D$, $E$, $F$, $A$ puis $I$,
$J$, $G$, $H$, $I$ puis $K$, $L$, $M$, $N$, $K$ puis $P$, $Q$ puis
$R$, $S$, $T$, $U$ puis $R$, $V$, $S$ puis $T$, $W$, $V$ puis $X$,
$Y$.
\end{myenumerate}
%@Commentaire: Utilisation des relatifs dans un repère du plan (placement de points).