%@Dif:2
\begin{myenumerate}
\item Construis un repère du plan : unité 1~cm. On pourra utiliser
du papier millimétré.
\item Dans ce repère, place les points $A(3;-3)$; $B(-2;4)$; $C(-9;-3)$.
\par Trace le triangle $ABC$.
\item Trace les droites $(d_1)$, $(d_2)$, $(d_3)$ perpendiculaires
 respectivement aux droites $(AB)$, $(BC)$, $(CA)$ en $C$, $A$ et $B$.
\par Quelles sont les coordonnées des points d'intersection des
droites $(d_1)$, $(d_2)$, $(d_3)$ avec les côtés du triangle ?
\item On appelle $H$ le point d'intersection des trois droites $(d_1)$,
$(d_2)$ et $(d_3)$.
\par Quelles sont les coordonnées du point $H$ ?
\end{myenumerate}
%@Commentaire: Utilisation des relatifs dans un repère du plan (placement de points, lecture graphique de coordonnées). Association géométrie plane - repère.