%@metapost:604quadri.mp
%@Dif:3
Les figures ci-dessous ont été réalisées à main levée ; chaque dessin montre les quatre côtés d'un quadrilatère, et, pour certains d'entre eux, les diagonales. Certaines informations ont été portées sur les dessins sous forme codée.
\par
\dispo{1}{
\begin{tabular}{|c|c|}
\hline
\multicolumn{1}{|l|}{\pscirclebox{A}}&\multicolumn{1}{l|}{\pscirclebox{B}}\\
\includegraphics{604quadri.5}&\includegraphics{604quadri.6}\\
\hline
\multicolumn{1}{|l|}{\pscirclebox{C}}&\multicolumn{1}{l|}{\pscirclebox{D}}\\
\includegraphics{604quadri.7}&\includegraphics{604quadri.8}\\
\hline
\multicolumn{1}{|l|}{\pscirclebox{E}}&\multicolumn{1}{l|}{\pscirclebox{F}}\\
\includegraphics{604quadri.9}&\includegraphics{604quadri.10}\\
\hline
\end{tabular}}{
\begin{myenumerate}
\item Associe à chaque dessin le texte qui le décrit.
\begin{description}
\item[Phrase 1] : Les diagonales du quadrilatère se coupent en formant un angle droit.
\item[Phrase 2] : Les côtés opposés du quadrilatère sont parallèles, deux des côtés consécutifs\footnote{Qui se suivent} sont perpendiculaires et ont même longueur
\item[Phrase 3] : Deux des angles opposés du quadrilatère sont des angles droits.
\item[Phrase 4] : Deux des côtés consécutifs du quadrilatère ont la même longueur et les diagonales sont perpendiculaires.
\item[Phrase 5] : Deux côtés consécutifs du quadrilatère ont même longueur.
\item[Phrase 6] : Deux des côtés consécutifs du quadrilatère ont même longueur, ses diagonales ont même longueur et se coupent en leur milieu.
\end{description}
\item Pour chaque figure, trace, aux instruments, un quadrilatère qui respecte les informations.
\end{myenumerate}
}
%@Commentaire: Travail sur la liaison figure-texte. Intérêt de la figure à main levée. Le vocabulaire est ici un peu plus difficile à cause du mot {\em consécutif}.