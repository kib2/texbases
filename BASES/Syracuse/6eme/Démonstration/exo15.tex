%@metapost:Idcquad.mp
%@Auteur: Thierry Gauvin\par
Réponds aux questions, il peut y avoir
{\textit{\underline{plusieurs}}} bonnes réponses : entoure celles que
tu choisis.
\par{\textbf{\underline{Barème :}}} tu {\textit{\underline{gagnes}}} 1
point par bonne réponse; tu {\textit{\underline{perds}}} 1 point par
réponse fausse; aucune réponse vaut 0.

{\small
\begin{tabular}{|m{4.5cm}|m{2.5cm}|m{2.5cm}|m{2.5cm}|m{2.5cm}|}
\hline
\quad \'Enoncés & \quad Réponse A & \quad Réponse B & \quad Réponse C & \quad Réponse D \\
\hline
Les cotés d'un losange \dots & sont perpendiculaires & sont parallèles deux à deux & ont même milieu & sont tous égaux \\
\hline
Les diagonales d'un carré \dots & sont perpendiculaires & sont parallèles & ont même milieu & sont de même longueur \\
\hline
Les cotés d'un rectangle \dots & sont perpendiculaires & sont parallèles deux à deux & ont même milieu & sont tous égaux \\
\hline
Les diagonales d'un cerf-volant \dots & sont perpendiculaires & sont parallèles & ont même milieu & sont de même longueur \\
\hline
Les cotés d'un parallélogramme \dots & sont perpendiculaires & sont parallèles deux à deux & ont même milieu & sont tous égaux \\
\hline
Les diagonales d'un rectangle \dots & sont perpendiculaires & sont parallèles & ont même milieu & sont de même longueur \\
\hline
Un cerf-volant dont les diagonales se coupent en leur milieu est : & un carré & un rectangle & un losange & un parallélogramme \\
\hline
Un losange qui a un angle droit est : & un carré & un rectangle & un cerf-volant & un parallélogramme \\
\hline
Un rectangle dont les diagonales sont perpendiculaires est : & un parallélogramme & un cerf-volant& un losange & un carré \\
\hline
Quelle est la phrase exacte ? & un rectangle est un carré & un carré est un losange & un cerf-volant est un losange & un trapèze est un parallélogramme \\
\hline
$\includegraphics[width=4.5cm]{Idcquad.1}$ & (CR) est la médiatrice de [EF] & (EF) est la médiatrice de [CR] & CEFR est un cerf-volant & CFRE est un cerf-volant \\
\hline
$\includegraphics[width=4.5cm]{Idcquad.2}$ & VPAC est un rectangle & CDPR est un losange & CDP est un triangle équilatéral & VPDC est un cerf-volant \\
\hline
D'après le codage de la figure, IJKL est $\includegraphics[width=4cm]{Idcquad.3}$ & un losange & un rectangle & un cerf-volant & un carré \\
\hline
\end{tabular}
}
