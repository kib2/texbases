%@Auteur: Nathalie Lespinasse\par
{\em Le travail rendu devra être fait correctement : écriture et présentation soignées.}
\begin{description}
\item[Qu'est ce qu'un polygone ?]\hfill\par
  \begin{myenumerate}
    \item Trouvez dans le dictionnaire la définition du mot {\em polygone} et recopiez-la.
    \item Le mot polygone vient-il du grec ou du latin ?
    \item Le mot polygone est formé de deux petits mots simples, lesquels ? Cherchez leur signification.
  \end{myenumerate}
\item[Polygones particuliers]\hfill\par
  \begin{myenumerate}
    \item Trouvez dans le dictionnaire les définitions des mots suivants. Donnez aussi leur étymologie (c'est-à-dire les mots grecs ou latins qui les forment) :
      \begin{center}
        \begin{tabular}{cccc}
          hexamètre&décasyllabe&octopode&pentathlon\\
          dodécaphonique&ennéade&heptacorde\\
        \end{tabular}
      \end{center}
\item Relisez très attentivement les définitions, puis recopiez et complétez les pointillés par le nombre qui convient :
\begin{center}
        \begin{tabular}{cccc}
          hexamètre&décasyllabe&octopode&pentathlon\\
          \ldots&\ldots&\ldots&\ldots\\
          dodécaphonique&ennéade&heptacorde\\
          \ldots&\ldots&\ldots\\
        \end{tabular}
      \end{center}
\item Trouvez le nom d'un polygone 
  \begin{multicols}{3}
    \begin{enumerate}[a--]
    \item à cinq côtés
    \item à six côtés
    \item à sept côtés
    \item à huit côtés
    \item à neuf côtés
    \item à dix côtés
    \item à douze côtés
    \item à trois côtés
    \item à quatre côtés.
    \end{enumerate}
  \end{multicols}
  \end{myenumerate}
\end{description}