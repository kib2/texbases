%@Auteur:d'après {\bf Lecture \& Maths} -- SCEREN
\par Coche la case de l'information qui manque pour résoudre ces problèmes.
\begin{enumerate}[A/]
\item La voiture de M. Georges consomme 6 litres de gazole aux 100~km. Dimanche dernier, avec sa famille, il s'est rendu en promenade au château de Versailles. \\Quelle quantité de gazole la voiture a-t-elle consommée pour ce déplacement ?
{\em
\begin{itemize}
\item[$\square$] La vitesse moyenne de la voiture.
\item[$\square$] Le prix du litre d'essence
\item[$\square$] La longueur du trajet.
\end{itemize}
}
\item Pour l'achat d'un magnétoscope, l'école dispose d'une somme de 
200~\textgreek{\euro}. Elle organise une tombola qui rapporte 
300~\textgreek{\euro} et gagne un concours de dessin doté de 
60~\textgreek{\euro}.\\Aura-t-elle assez d'argent pour acheter ce 
magnétoscope ?
{\em
\begin{itemize}
\item[$\square$] Le prix du magnétoscope.
\item[$\square$] Le nombre de dessins envoyés.
\item[$\square$] Le nombre de billets vendus pour la tombola.
\end{itemize}
}
\item Une caisse de raisins pèse 10~kg lorsqu'elle est bien remplie. Un 
marchand de fruits reçoit 5 caisses. Il commande aussi 5 caisses de 
bananes.\\Quelle est la masse de raisin reçu ?
{\em
\begin{itemize}
\item[$\square$] Le prix d'une caisse vide.
\item[$\square$] La masse d'une caisse vide.
\item[$\square$] Le prix du kilogramme de raisin.
\end{itemize}
}
\end{enumerate}
