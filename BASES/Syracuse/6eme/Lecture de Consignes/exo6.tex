%@Auteur: d'après {\em Consignes et démarches en mathématiques}
\par{\em Voici l'énoncé de l'exercice de Lucette :}
\begin{quote}
  Expliquer ce que signifient les écritures $(AB)$, $[AB)$, $[AB]$ et $AB$.
\end{quote}
{\em Aide-la en complétant les phrases suivantes avec les mots : } un (des) crochet(s), la demi-droite, la droite, la longueur, un nombre, un (des) parenthèse(s) {\em et} le segment.
\begin{enumerate}[(1)]
\item L'écriture $(AB)$, entre \ldots, désigne \ldots passant par $A$ et $B$.
\item L'écriture $[AB)$ a \ldots à gauche de $A$ et \ldots à droite de $B$ : il s'agit donc de \ldots d'origine $A$ passant par $B$.
\item L'écriture $[AB]$, entre \ldots, désigne \ldots d'extrémités $A$ et $B$.
\item L'écriture $AB$, sans aucun symbole, désigne \ldots du segment $[AB]$ ; $AB$ est donc \ldots.
\end{enumerate}