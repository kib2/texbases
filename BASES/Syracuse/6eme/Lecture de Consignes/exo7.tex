%@Titre: Trois programmes pour une même figure.
%@Auteur: d'après {\em Consignes et démarches en mathématiques}.
\begin{description}
\item[Programme 1]\subitem\newline
\begin{itemize}
\item Trace deux droites $(d_1)$ et $(d_2)$ parallèles distantes de
4~cm.
\item Place un point $A$ sur la droite $(d_1)$.
\item Trace la perpendiculaire à la droite $(d_1)$ passant par $A$.
\item Cette droite coupe la droite $(d_2)$ en $B$.
\item Place un point $D$ sur la droite $(d_1)$ à 4~cm de $A$.
\item Trace la perpendiculaire à la droite $(d_1)$ passant par $D$.
\item Cette droite coupe la droite $(d_2)$ en $C$.
\item Construis le milieu $I$ du segment $[BC]$ et le milieu $J$ du
segment $[AD]$.
\item Trace en couleur le polygone $ABCD$ et les segments $[ID]$ et
$[CJ]$.
\end{itemize}
\item[Programme 2]\hfill\newline
  \begin{itemize}
  \item Construis un carré $ABCD$ de côté 4~cm.
  \item Construis le milieu $I$ du segment $[BC]$.
  \item Trace la parallèle à la droite $(AB)$ passant par $I$.
  \item Elle coupe le segment $[AD]$ en $J$.
  \item Trace en couleur le carré $ABCD$ et les segments $[ID]$ et
$[CJ]$.
  \end{itemize}
\item[Programme 3]\hfill\newline
  \begin{itemize}
  \item Place un point $C$.
  \item Construis un point $D$ tel que $CD=4$~cm.
  \item Trace un triangle $CDI$ rectangle en $C$ tel que $CI=2$~cm.
  \item Trace un triangle $CDJ$ rectangle en $D$ tel que $DJ=2$~cm et
les droites $(IJ)$ et $(CD)$ soient parallèles.
  \item Construis le point $B$ tel que $I$ soit le milieu du segment
$[BC]$.
  \item Trace la parallèle $(d_1)$ à la droite $(CD)$ passant par $B$.
  \item Trace la perpendiculaire $(d_2)$ à la droite $(CD)$ passant
par $D$.
  \item Les droites $(d_1)$ et $(d_2)$ se coupent en $A$.
  \item Trace en couleur le polygone $ABCD$ et les segments $[ID]$ et
$[CJ]$.
  \end{itemize}
\end{description}