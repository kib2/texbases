%@Auteur:d'après {\bf Lecture \& Maths} -- SCEREN
\par
Entoure le numéro de l'information qui manque pour résoudre ces problèmes.
\begin{enumerate}[A/]
\item Guillaume, avec un jerrycan de 20 litres d'eau de source, remplit le plus grand nombre possible de bouteilles. Combien de bouteilles remplit-il ?
{\em Quelle information manque-t-il dans cet énoncé ?
\begin{enumerate}[1/]
\item La provenance de l'eau de source.
\item Le nombre de litres de limonade.
\item La capacité d'une bouteille.
\end{enumerate}
}
  \item Un épicier revient du marché avec de belles oranges. \`A la fin de la journée, il a vendu 36~kg d'oranges.\\Combien d'argent a-t-il retiré de cette vente ?
{\em Quelle information manque-t-il dans cet énoncé ?
\begin{enumerate}[1/]
\item Le prix de vente du kilogramme d'oranges.
\item La taille des oranges.
\item Le nombre d'oranges par kilogramme.
\end{enumerate}
}
\item Un chauffeur de taxi fait le plein d'essence de son véhicule. Le réservoir contient 40 litres. Combien de kilomètres pourra-t-il parcourir avec le réservoir plein ?
{\em Quelle information manque-t-il dans cet énoncé ?
\begin{enumerate}[1/]
\item La vitesse du véhicule.
\item La consommation du véhicule.
\item Le prix d'un litre d'essence.
\end{enumerate}
}
\end{enumerate}
