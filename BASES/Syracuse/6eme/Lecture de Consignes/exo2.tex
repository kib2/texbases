%@Dif:3
On donne ci-dessous la solution {\em exacte} d'un problème :
\begin{center}
\psshadowbox{\begin{minipage}{\linewidth/3}
$(100\times15)=1\,500$
\\$50\times32=1\,600$
\\$1\,500+1\,600=3\,100$
\\$3\,100-2\,500=600$
\\Il reste en caisse 600 euros.
\end{minipage}
}
\end{center}
Retrouve le texte de ce problème à partir des expressions ci-dessous :
\begin{itemize}
\item à 15 euros
\item et 50 repas
\item Le restaurateur met
\item Quelle part de la recette lui reste-t-il en caisse ?
\item 100 repas
\item à 32 euros.
\item Un restaurateur sert
\item 2\,500 euros dans son coffre.
\end{itemize}
