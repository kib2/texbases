\begin{myenumerate}
\item Construis un triangle $EFG$ tel que $FG=10$~cm; $GE=7$~cm et
$\widehat{EGF}=60$\degres.
\item Trace, en vert, les bissectrices des angles
$\widehat{EFG}$, $\widehat{FGE}$, $\widehat{GEF}$. On laissera
apparaître les traits de construction.
\item Que remarque-t-on ? On appellera $I$ le point particulier.
\item La droite $(d)$, perpendiculaire à la droite $(EF)$ et passant
par $I$, coupe la droite $(EF)$ en $P$.\\La droite $(d_1)$,
perpendiculaire à la droite $(FG)$ et passant par $I$, coupe la droite
$(FG)$ en $R$.\\ La droite $(d_2)$, perpendiculaire à la droite $(GE)$
et passant par $I$, coupe la droite $(GE)$ en $S$.\par Construis le
cercle de centre $I$ et de rayon $IP$. Que remarque-t-on ?
\end{myenumerate}