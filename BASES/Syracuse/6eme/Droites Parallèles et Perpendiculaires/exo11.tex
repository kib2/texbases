%@geogebra:6parallelesexo11.ggb
%@Dif:2
Soit $ABC$ un triangle quelconque.
\begin{myenumerate}
\item Trace la droite $(d_1)$ perpendiculaire à la droite $(AB)$
  passant par $A$.
\\Trace la droite $(d_2)$ perpendiculaire à la droite $(AB)$ passant
par $B$.
\item Trace la droite $(d_3)$ perpendiculaire à la droite $(AC)$
  passant par $A$.
\\Trace la droite $(d_4)$ perpendiculaire à la droite $(AC)$ passant
par $C$.
\item Trace la droite $(d_5)$ perpendiculaire à la droite $(BC)$
  passant par $B$.
\\Trace la droite $(d_6)$ perpendiculaire à la droite $(BC)$ passant
par $C$.
\item On appelle $I$ le point d'intersection des droites $(d_2)$ et
  $(d_4)$. On appelle $J$ le point d'intersection des droites $(d_1)$
  et $(d_6)$. On appelle $K$ le point d'intersection des droites
  $(d_3)$ et $(d_5)$.
\\Trace en rouge les segments $[AI]$, $[BJ]$ et $[CK]$.
\\Fais trois remarques concernant ces segments.
\end{myenumerate}
%@Commentaire: Permet de travailler la construction de droites perpendiculaires avec la syntaxe habituelle.