%@P:exocorcp
%@metapost:604dm08.mp
%@Dif:3
\compo{2}{604dm08}{1}{
\begin{myenumerate}
\item
  \begin{enumerate}
  \item Construis en vraie grandeur le triangle dessiné ci-contre à
  main levée, sachant que $TI=8$~cm et $RI=7$~cm.
\item \'Ecris un programme de construction pour ce triangle.
  \end{enumerate}
\item Que peut-on dire du triangle $RTI$ ? {\em Explique pourquoi.}
\end{myenumerate}
}
\begin{myenumerate}
  \setcounter{enumi}{2}
\item
  \begin{enumerate}
  \item Trace $(d_1)$, la droite passant par $I$ et perpendiculaire à
    la droite $(RI)$. Trace $(d_2)$, la droite passant par $R$ et
    perpendiculaire à la droite $(TI)$. Les droites $(d_1)$ et $(d_2)$
    se coupent au point $E$.
\item Trace le cercle de diamètre $[ER]$. Que remarque-t-on ?
  \end{enumerate}
\item
  \begin{enumerate}
  \item Trace la médiatrice du segment $[RI]$. Elle coupe le segment
    $[TI]$ en un point $O$.
  \item Trace le cercle de centre $O$ et de rayon $OI$. Que
    remarque-t-on ?
  \end{enumerate}
\end{myenumerate}
%@Commentaire: Reprise de l'exercice \verb+exo10+ sans la partie démonstration.
%@Correction:
\begin{myenumerate}
\item
  \begin{enumerate}
    \setcounter{enumii}{1}
  \item Comme les droites $(TR)$ et $(RI)$ sont perpendiculaires alors
  le triangle $RTI$ est rectangle en $R$.
  \end{enumerate}
\end{myenumerate}