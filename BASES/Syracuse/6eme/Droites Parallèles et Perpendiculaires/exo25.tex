%@P:exocorcp
\begin{myenumerate}
\item
\begin{enumerate}
\item Sur une feuille non quadrillée, trace deux droites perpendiculaires $(d_1)$ et $(d_2)$.\\
On appelle $A$ leur point d'intersection.
\item Place un point $B$ sur la droite $(d_1)$ et un point $D$ sur la droite $(d_2)$ tels que $AB=7$~cm  et $AD=3,5$~cm.
\item Trace la perpendiculaire à la droite $(d_1)$ passant par $B$, puis la perpendiculaire à la droite $(d_2)$ passant par $D$.\\Ces deux droites se coupent en $C$.
\item Quelle est la nature du quadrilatère $ABCD$ ? Justifie.
\end{enumerate}
\item
\begin{enumerate}
\item Reprends la question 1 et refaire une figure avec $AB=AD=4,5$~cm.
\item Quelle est dans ce deuxième cas de figure la nature du quadrilatère $ABCD$ ? Justifie.
\end{enumerate}
\end{myenumerate}
%@Correction:
\begin{myenumerate}
  \item
    \begin{enumerate}
      \setcounter{enumii}{3}
    \item Je sais que le quadrilatère $ABCD$ a trois angles droits. Alors je peux conclure que le quadrilatère $ABCD$ est un rectangle.
    \end{enumerate}
  \item
    \begin{enumerate}
      \setcounter{enumii}{2}
    \item D'après la question 1.d., je sais déjà que le quadrilatère $ABCD$ est un rectangle. Je sais aussi qu'il a 4 côtés de même longueur. Alors je peux conclure que $ABCD$ est un rectangle et en même temps un losange : le quadrilatère $ABCD$ est en fait un carré.
    \end{enumerate}
\end{myenumerate}