%@Titre:Une rosace à quatre branches.
%@geogebra:6parallelesexo58.ggb
%@Auteur: IREM Strasbourg\par
\begin{myenumerate}
\item Trace deux droites $(d)$ et $(\Delta)$ perpendiculaires, se
  coupant en un point $O$.
\item Prends un point $A_1$ sur la droite $(d)$.
\item Place sur la droite $(\Delta)$ un point $B_1$ tel que
  $A_1B_1=9$~cm.
\item Trace la perpendiculaire en $O$ à la droite $(A_1B_1)$.
\item Marque en rouge le point d'intersection de cette droite
  avec la droite $(A_1B_1)$.
\item Prends un point $A_2$ sur la droite $(d)$.
\item Place sur la droite $(\Delta)$ un point $B_2$ tel que $A_2B_2=9$~cm.
\item Trace la perpendiculaire en $O$ à la droite $(A_2B_2)$.
\item Marque en rouge le point d'intersection de cette droite
  avec la droite $(A_2B_2)$.
\end{myenumerate}
Continue avec un point $A_3$, un point $A_4$,\ldots beaucoup de points sur la droite $(d)$.