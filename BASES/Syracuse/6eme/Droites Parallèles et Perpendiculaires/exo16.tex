%@metapost:paraperp605exo01.mp
%@Auteur: d'après IREM Strasbourg
\begin{center}
  \begin{tabular}{|c|c|}
\hline
\includegraphics{paraperp605exo01.1}&\includegraphics{paraperp605exo01.2}\\
Figure 1&Figure 2\\
\hline
\includegraphics{paraperp605exo01.3}&\includegraphics{paraperp605exo01.4}\\
Figure 3&Figure 4\\
\hline
  \end{tabular}
\end{center}
Tous les tracés ci-dessous doivent être exécutés {\em à main levée}.
\begin{myenumerate}
\item Sur la figure 1, trace une droite parallèle à la droite $(d_1)$
et une droite parallèle à la droite $(d_2)$.
\item Sur la figure 2, trace une droite perpendiculaire à la droite
$(d_3)$ et une droite perpendiculaire à la droite $(d_4)$.
\item Sur la figure 3, trace la droite parallèle à la droite $(d_1)$
passant par $A$ et la droite parallèle à la droite $(d_2)$ passant par
$B$.
\item Sur la figure 4, trace la droite perpendiculaire à la droite
$(d_3)$ passant par $E$ et la droite perpendiculaire à la droite
$(d_4)$ passant par $F$.
\end{myenumerate}