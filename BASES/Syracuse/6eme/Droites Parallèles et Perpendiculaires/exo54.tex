%@Titre:La trisectrice de Mac-Laurin
%@Auteur:\url{http://www.animath.fr/UE/missenard/courbes.html}
%@geogebra:6parallelesexo54.ggb
%@Danger:Réalise ces courbes sur grande feuille, avec soin et précision.

\begin{myenumerate}
  \item Trace un cercle $\mathscr C$ de centre $O$, de rayon 10~cm.%R
  \item Soit $A$ un point de $\mathscr C$. $B$ est le point de la
    droite $(OA)$ extérieur au cercle $\mathscr  C$ et tel que
    $AB=5$~cm.\\%R/2.
    La droite $(d)$ est la droite qui est perpendiculaire à la droite
    $(OA)$ en $B$.
  \item Place sur le cercle $\mathscr C$ un point $P$ quelconque; la
    droite $(AP)$ coupe la droite $(d)$ en $Q$. $M$ est le milieu du
    segment $[PQ]$.
  \item Recommence l'étape précédente avec de nombreux autres points
    $P$ de $\mathscr C$.
\end{myenumerate}
Les points $M$ décrivent une courbe nommée {\em trisectrice de Mac-Laurin}.
