%@Dif:2
\begin{myenumerate}
\item Trace une droite $(d)$ et place trois points $A$, $B$ et $C$ sur
cette droite.
\item
\begin{enumerate}
\item Trace la perpendiculaire à la droite $(d)$ passant par le point
  $A$. Cette droite s'appelle $(d_1)$.
\item Trace la perpendiculaire à la droite $(d)$ passant par le point
  $B$. Cette droite s'appelle $(d_2)$.
\end{enumerate}
\end{myenumerate}
%@Commentaire: Exercice donné en DS avant d'avoir fait les propriétés. Permet d'insister sur le vocabulaire et les techniques de tracés.


