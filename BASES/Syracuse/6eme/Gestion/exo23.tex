%@P:exocorcp
%@metapost:gestion605exo002.mp
%@Auteur:d'après Galion-Thèmes Collège
%@Dif:2
\par\compo{1}{gestion605exo002}{1}{Voici le \og climogramme\fg{} d'une
ville d'Afrique du Nord. Il a été établi de la manière suivante : pour
le mois de janvier par exemple, la température moyenne a été de
8\degres\ et la hauteur des précipitations a été de 90 millimètres de
pluie : on marque le point Ja (comme Janvier) sur le graphique.

On procède de même pour chaque mois de l'année : F, M, A, Ma, Ju, Jl,
Ao, S, O, N, D. Le climogramme comporte ainsi 12 points.
}
\begin{myenumerate}
  \item Pour cette ville, quelle est la température moyenne au mois de
Juin ? au mois de Mai ?
  \item En quel mois la température est-elle la plus élevée ? la plus
basse ?
  \item Quel est le mois le plus sec ? le plus humide ?
  \item Construis un tableau qui regroupe toutes les informations du
climogramme.
    \begin{center}
      \begin{tabular}{|l|c|ccc}
\hline
Mois&\ldots&\ldots\\
\hline
Température (\degres C)&\ldots&\ldots\\
\hline
Hauteur de pluie (en mm)&\ldots&\ldots\\
\hline
      \end{tabular}
    \end{center}
\end{myenumerate}
%@Correction:
\begin{myenumerate}
  \item La température moyenne est de 20\degres C au mois de Juin et de
    16\degres C au mois de Mai.
  \item La température est la plus élevée en Août. Elle est la plus
    basse en Janvier.
  \item Le mois le plus sec est Juillet. Le mois le plus humide est Décembre.
  \item\hfill\newline
    \begin{center}
      \begin{tabular}{|l|c|c|c|c|c|c|c|c|c|c|c|c|}
\hline
Mois&Ja.&Fe.&Ma.&Av.&Ma.&Ju.&Jl.&Ao.&Se.&Oc.&No.&De\\
\hline
Température (\degres C)&8&11&12&13&16&20&22&24&21&20&15&11\\
\hline
Hauteur de pluie (en mm)&90&95&75&60&22&10&5&15&50&110&130&150\\
\hline
      \end{tabular}
    \end{center}
\end{myenumerate}