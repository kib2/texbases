%@Auteur: d'après Galion-Thèmes Collège
%@Dif:3
La forme du climogramme en un point du globe donne des indications sur le climat de la région : d'où son nom.

Voici des données pour diverses régions du globe terrestre. Dessine, sur du papier millimétré, les climogrammes correspondants. Compare-les ensuite et déduis-en des remarques judicieuses sur les climats.
\begin{description}
\item[Climat tempéré] PARIS
{\small
  \begin{center}
    \begin{tabular}{|c|c|c|c|c|c|c|c|c|c|c|c|c|c|}
\cline{2-13}
\multicolumn{1}{c|}{}&Ja&Fe&Ma&Av&Ma&Ju&Jl&Ao&Se&Oc&No&De\\
\hline
Température (\degres{}C)&3,5&4,3&7,5&11&14,5&17,8&19,5&19,7&16,5&11,8&7,3&4,3\\
\hline
Pluie (mm)&56&45&35&42&57&54&59&64&55&50&51&60\\
\hline
    \end{tabular}
  \end{center}
}
\item[Climat équatorial] BRAZZAVILLE
{\small
  \begin{center}
    \begin{tabular}{|c|c|c|c|c|c|c|c|c|c|c|c|c|c|}
\cline{2-13}
\multicolumn{1}{c|}{}&Ja&Fe&Ma&Av&Ma&Ju&Jl&Ao&Se&Oc&No&De\\
\hline
Température (\degres{}C)&25,8&26,2&26,3&25,4&25,7&23,1&21,3&22,9&24,7&25,8&25,8&25,6\\
\hline
Pluie (mm)&143&174&146&237&132&17&2&3&33&156&181&169\\
\hline
    \end{tabular}
  \end{center}
}
\item[Climat tropical] MEXICO
{\small
  \begin{center}
    \begin{tabular}{|c|c|c|c|c|c|c|c|c|c|c|c|c|c|}
\cline{2-13}
\multicolumn{1}{c|}{}&Ja&Fe&Ma&Av&Ma&Ju&Jl&Ao&Se&Oc&No&De\\
\hline
Température (\degres{}C)&27&27&25,9&24,1&21,7&20,2&19&19,8&21,6&23&24,8&26,7\\
\hline
Pluie (mm)&71&95&42&32&20&10&4&7&13&12&26&50\\
\hline
    \end{tabular}
  \end{center}
}
\item[Climat aride] ALEXANDRIE
{\small
  \begin{center}
    \begin{tabular}{|c|c|c|c|c|c|c|c|c|c|c|c|c|c|}
\cline{2-13}
\multicolumn{1}{c|}{}&Ja&Fe&Ma&Av&Ma&Ju&Jl&Ao&Se&Oc&No&De\\
\hline
Température (\degres{}C)&14,6&15&16,8&18,8&21,6&24,1&25,8&25,5&25,6&24,2&20,7&16,5\\
\hline
Pluie (mm)&41&24&5&2&3&0&0&0&0&4&28&42\\
\hline
    \end{tabular}
  \end{center}
}
\end{description}