%@Dif:2
Pour être homologué, un ballon de rugby doit répondre à certaines
normes parmi lesquelles :
\begin{quote}
\begin{itemize}
\item[$\bullet$] sa longueur doit être comprise entre 2,8~dm et
  3~dm.
\item[$\bullet$] sa masse doit être comprise entre 0,4~kg et
  0,44~kg.
\end{itemize}
\end{quote}
Si l'une des deux conditions n'est pas satisfaite, le ballon n'est pas
homologué.
\\Parmi les quatre ballons décrits ci-dessous, indique s'ils peuvent
être homologués et pourquoi.
\begin{center}
\begin{tabular}{|c|c|c|}
\cline{2-3}
\multicolumn{1}{c|}{}&{\bf Longueur}&{\bf Masse}\\
\hline
{\bf Ballon A}&2,75~dm&0,43~kg\\
\hline
{\bf Ballon B}&2,806~dm&0,438~kg\\
\hline
{\bf Ballon C}&2,895~dm&0,453~kg\\
\hline
{\bf Ballon D}&2,901~dm&0,46~kg\\
\hline
\end{tabular}
\end{center}