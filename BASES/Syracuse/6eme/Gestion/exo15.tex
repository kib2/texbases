%@Fichier: multirow.
%@Dif:3
\begin{myenumerate}
  \item Observe le tableau.
    \begin{center}
      \begin{tabular}{|c|c|c|c|c|c|c|}
\cline{3-7}
\multicolumn{2}{c|}{}&6\ieme&5\ieme&4\ieme&3\ieme&Total\\
\hline
\multirow{2}{2.5cm}{\centerline{Garçons}}&Externes&40&33&37&49&159\\
\cline{2-7}
&Demi-pensionnaires&36&40&34&32&142\\
\hline
\multirow{2}{2.5cm}{\centerline{Filles}}&Externes&41&37&46&45&169\\
\cline{2-7}
&Demi-pensionnaires&32&37&28&32&129\\
\hline
\multicolumn{2}{|r|}{\bf Total}&149&147&145&158&599\\
\hline
      \end{tabular}
    \end{center}
\item \`A l'aide du tableau, réponds aux questions suivantes :
  \begin{enumerate}
  \item Quel est l'effectif des filles externes en 6\ieme\ ?
  \item Quel est l'effectif des garçons demi-pensionnaires en 3\ieme\ ?
  \item Quel est l'effectif des filles demi-pensionnaires en 6\ieme\ ?
  \item Quel est l'effectif des garçons externes ?
  \item Quel est l'effectif des élèves en 5\ieme\ ?
  \item Quel est l'effectif des élèves ?
  \item Quel est l'effectif des externes en 4\ieme\ ?
  \item Quel est l'effectif des externes ?
  \item Quel est l'effectif des filles ?
  \end{enumerate}
\end{myenumerate}