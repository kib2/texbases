%@Dif:3
Complète le tableau suivant dans lequel la première ligne indique différents volumes d'un corps et les lignes suivantes, la masse correspondante. Pour chacune des lignes à remplir, est indiqué, dans la marge, le nom du corps avec sa {\em masse volumique}.
\renewcommand{\arraystretch}{1.5}
\begin{center}
\begin{tabular}{l|c|c|c|c|c|c|c|}
\cline{2-8}
&{\bf Volume}&1~cm$^3$&1~dm$^3$&1~m$^3$&145~cm$^3$&4~dm$^3$&0,5~m$^3$\\
\cline{2-8}
Bois : 0,7~g par cm$^3$&{\bf Masse}&0,7~g&&&&&\\
\cline{2-8}
Fer : 7,8~kg par dm$^3$&{\bf Masse}&&7,8~kg&&&&\\
\cline{2-8}
Huile : 0,9~kg par dm$^3$&{\bf Masse}&&0,9~kg&&&&\\
\cline{2-8}
Mercure : 13,6~g par cm$^3$&{\bf Masse}&13,6~g&&&&&\\
\cline{2-8}
\end{tabular}
\end{center}
\renewcommand{\arraystretch}{1}
%@Commentaire: On travaille la proportionnalité ainsi que les multiplications par 10, 100, 1\,000. Les conversions sont aussi présentes.