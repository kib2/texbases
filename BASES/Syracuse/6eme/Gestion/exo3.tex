%@Fichier: tabularx.
%@metapost:603dmtousdm04.mp
%@Dif:3
Le graphique ci-dessous montre l'évolution de la population rurale
(personnes qui habitent la campagne) et de la population urbaine
(personnes qui habitent en ville) en France de 1900 à 1980.
\par
\compo{1}{603dmtousdm04}{1}{
\begin{myenumerate}
\item De 1900 à 1930, la population rurale a-t-elle augmenté ou
diminué ? de 1950 à 1980 ?
\item Reprends la question précédente avec la population urbaine.
\item Sur la graduation horizontale, combien d'années sont
représentées par 1~cm ? Combien d'années sont représentées par
1~mm ?\\Sur la graduation verticale, combien d'habitants sont
représentés par 1~cm ? Combien d'habitants sont représentés par
1~mm ?
\item Construis et complète le tableau ci-dessous puis range ensuite
les valeurs obtenues par ordre décroissant.
\[\begin{tabularx}{7cm}{|c|X|X|}
\hline
Année&Population rurale&Population urbaine\\
\hline
1900&&\\
\hline
1930&&\\
\vdots&&\\
\end{tabularx}
\]
\end{myenumerate}
}