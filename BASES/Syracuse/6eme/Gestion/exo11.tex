%@Dif:4
Complète le tableau suivant dans lequel la première ligne indique
différentes masses d'un corps et les lignes suivants, le volume
correspondant. Pour chacune des lignes à remplir, est indiqué, dans la
marge, le nom du corps avec sa {\em masse volumique}.
\renewcommand{\arraystretch}{1.5}
\[
\begin{tabular}{l|c|c|c|c|c|}
\cline{2-6}
&{\bf Masse}&10,5~g&1,03~kg&2,5~kg&100~g\\
\cline{2-6}
Argent : 10,5~g par cm$^3$&{\bf Volume}&1~cm$^3$&&&\\
\cline{2-6}
Lait : 1,03~kg par dm$^3$&{\bf Volume}&&1~dm$^3$&&\\
\cline{2-6}
Verre : 2,5~kg par dm$^3$&{\bf Volume}&&&2,5~dm$^3$&\\
\cline{2-6}
\end{tabular}
\]
\renewcommand{\arraystretch}{1}
%@Commentaire: On travaille la proportionnalité ainsi que les multiplications par 10, 100, 1\,000. Les conversions sont aussi présentes. Les calculs sont moins faciles que dans l'exercice \verb+exo10+.