%@metapost:6gestionexo28.mp
%@Auteur: d'après CIMT, Universty of Exeter\par
\dispo{1}{
\begin{tabular}{|lccr|}
\hline
Athènes&1&0&Milan\\
Lille&2&2&Anderlecht\\
Bucarest&1&1&Kiev\\
Madrid&2&2&Lyon\\
Celtic&1&0&Manchester\\
Benfica&3&1&Copenhague\\
Moscou&0&2&Porto\\
Arsenal&3&1&Hambourg\\
Breme&1&0&Chelsea\\
Sofia&0&2&Barcelone\\
\hline
\end{tabular}
}{Le tableau ci-contre indique les résultats de certains matches de Ligue des
  Champions joués les 21 et 22 Novembre 2006.\par
L'équipe citée la première est l'équipe {\em à domicile} et l'autre
équipe est celle {\em à l'extérieur}.\par Les chiffres du milieu
indiquent le score, c'est-à-dire le nombre de buts marqués par chaque
équipe.
}
\begin{myenumerate}
  \item
    \begin{enumerate}
    \item  Combien de matches se sont terminés par une victoire de
    l'équipe à domicile ?
  \item  Combien de matches se sont terminés par une victoire de
    l'équipe à l'extérieur ?
  \item  Combien de matches se sont terminés par un match nul ?
    \end{enumerate}
  \item
    \begin{enumerate}
    \item Quel est le nombre de buts marqués par les équipes à
      domicile ?
    \item Quel est le nombre de buts marqués par les équipes à
      l'extérieur ?
    \end{enumerate}
  \item
    \begin{enumerate}
    \item Combien d'équipes ont marqué 0 but ? 1 but ? 2 buts ? 3 buts ?
    \item\hfill\newline\compo{1}{6gestionexo28}{1}{Construis un {\em
          diagramme en bâtons} comme celui ci-contre. Quel est le
        nombre de buts inscrit le plus souvent ?}
    \end{enumerate}
\end{myenumerate}