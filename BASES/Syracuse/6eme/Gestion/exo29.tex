%@metapost:6gestionexo29.mp
%@Auteur: D'après  CIMT, University of Exeter\par
\dispo{1}{
  \begin{tabular}{|c|c|c|}
    \hline
    Année&Hommes&Femmes\\
    \hline
    1948&7,8&5,7\\
\hline
1952&7,6&6,2\\
\hline
1956&7,8&6,4\\
\hline
1960&8,1&6,8\\
\hline
1964&8,1&6,8\\
\hline
1968&8,9&6,8\\
\hline
1972&8,2&6,8\\
\hline
1976&8,4&6,7\\
\hline
1980&8,5&7,1\\
\hline
1984&8,5&7,0\\
\hline
1988&8,7&7,4\\
\hline
1992&8,7&7,1\\
\hline
1996&8,5&7,1\\
\hline
2000&8,6&7,0\\
\hline
  \end{tabular}
}{Une des épreuves les plus populaires des Jeux Olympiques
  est le {\em saut en longueur}.\par Le tableau ci-contre indique la
  longueur du saut du vainqueur de 1948 à 2000.
\par
Sur le graphique ci-dessous, on a déjà représenté le premier résultat
pour les hommes. Complète-le.{\em On
  utilisera des couleurs différentes pour les hommes et pour les femmes.}
}
\[\includegraphics{6gestionexo29.1}\]