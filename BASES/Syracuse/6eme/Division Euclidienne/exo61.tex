%@Dif:3
On donne le critère de divisibilité par 13 :
\begin{center}
\psshadowbox{\begin{minipage}{0.75\linewidth}
\begin{itemize}
\item On barre le chiffre des unités et on ajoute son quadruple au
 nombre restant. 
\item Le nouveau nombre obtenu est-il un multiple de 13 ? 
\item Si oui, alors le nombre initial l'est aussi. 
\item Si non, alors le nombre initial ne l'est pas non plus. 
\item Si on ne sait pas conclure, on recommence avec ce nombre ce que
 l'on a fait précédemment. 
\end{itemize}
\end{minipage}
}
\end{center}
{\em Est-ce que 451\,487 et 370\,395 sont divisibles par 13 ?} 
\begin{center}
\opadd[voperator=bottom,operandstyle.1.1=\gray,operandstyle.2.1=\white,resultstyle.2=\rnode{A}{},resultstyle.1=\white]{451487}{280}
\kern1cm
\opadd[voperator=bottom,operandstyle.1.5=\rnode{B}{},operandstyle.1.1=\gray,operandstyle.2.1=\white,resultstyle.2=\rnode{C}{},resultstyle.1=\white]{45176}{240}
\kern1cm
\opadd[voperator=bottom,operandstyle.1.4=\rnode{D}{},operandstyle.1.1=\gray,operandstyle.2.1=\white,resultstyle.2=\rnode{E}{},resultstyle.1=\white]{4541}{40}
\kern1cm
\opadd[voperator=bottom,operandstyle.1.3=\rnode{F}{},operandstyle.1.1=\gray,operandstyle.2.1=\white,resultstyle.1=\white]{458}{320}
\end{center}
\ncarc[nodesepA=3mm]{->}{A}{B}
\ncarc[nodesepA=3mm]{->}{C}{D}
\ncarc[nodesepA=3mm]{->}{E}{F}
Comme 77 n'est pas un multiple de 13 alors 451\,487 non plus.
\begin{center}
\opadd[voperator=bottom,operandstyle.1.1=\gray,operandstyle.2.1=\white,resultstyle.2=\rnode{A}{},resultstyle.1=\white]{370395}{200}
\kern1cm
\opadd[voperator=bottom,operandstyle.1.5=\rnode{B}{},operandstyle.1.1=\gray,operandstyle.2.1=\white,resultstyle.2=\rnode{C}{},resultstyle.1=\white]{37059}{360}
\kern1cm
\opadd[voperator=bottom,operandstyle.1.4=\rnode{D}{},operandstyle.1.1=\gray,operandstyle.2.1=\white,resultstyle.2=\rnode{E}{},resultstyle.1=\white]{3741}{40}
\kern1cm
\opadd[voperator=bottom,operandstyle.1.3=\rnode{F}{},operandstyle.1.1=\gray,operandstyle.2.1=\white,resultstyle.1=\white]{377}{280}
\end{center}
\ncarc[nodesepA=3mm]{->}{A}{B}
\ncarc[nodesepA=3mm]{->}{C}{D}
\ncarc[nodesepA=3mm]{->}{E}{F}
Comme 65 est un multiple de 13 ($65=13\times5$) alors 370\,395 est un
multiple de 13. 
\par\vspace{5mm}\par\ding{46}{\em \`A ton tour} : les nombres suivants
sont-ils des multiples de 13 ? 
\begin{multicols}{4}
47\,073\par137\,851\par685\,489\par68\,164\par130\,120\par22\,502\par34\,293\par322\,496
\end{multicols}