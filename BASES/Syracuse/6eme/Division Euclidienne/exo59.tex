%@P:exocorcp
%@Dif:3
On donne le critère de divisibilité par 7 :
\begin{center}
  \psshadowbox{
    \begin{minipage}{0.5\linewidth}
      Un nombre est divisible par 7 si le nombre formé en supprimant
le dernier chiffre et en soustrayant deux fois la valeur de ce chiffre
est divisible par 7.
    \end{minipage}
    }
\end{center}
{\em Est-ce que 406 est divisible par 7 ?}\\$40-2\times6=28$. Comme 28
est divisible par 7 alors 406 aussi ($406=58\times7$).
\par\ding{46} {\em \`A ton tour} : Est-ce que 175; 486; 658; 1\,351;
2\,547 sont divisibles par 7 ?
%@Correction:
\begin{description}
\item[175] $17-2\times5=17-10=7$. Oui, 175 est divisible par 7.
\item[486] $48-2\times6=48-12=36$. Non, 486 n'est pas divisible par 7.
\item[658] $65-2\times8=65-16=49$. Oui, 658 est divisible par 7.
\item[1\,351] $135-2\times1=135-2=133$. On recommence
$13-2\times3=13-6=7$. Oui, 1\,351 est divisible par 7.
\item[2\,547] $254-2\times7=254-14=240$. On recommence
$24-2\times0=24-0=24$. Non, 2\,547 n'est pas divisible par 7.
\end{description}