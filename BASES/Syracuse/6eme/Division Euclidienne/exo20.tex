%@P:exocorcp
%@Auteur: Régis Leclercq
%@Dif:2
Recopie et complète, comme dans l'exemple. Attention, le reste doit
toujours être plus petit que le diviseur !
\par\underline{\bf{exemple}}: $57=\underbrace{6}_{diviseur}\times
\underbrace{9}_{quotient}+\underbrace{3}_{reste}$. Autrement dit,
$57\div6=9$ (reste 3).
\par
\begin{multicols}{2}
\begin{myenumerate}
\item $48=(9\times \dots)+\dots$. Autrement dit \ldots\ldots
\item $37=(5\times \dots)+\dots$. Autrement dit \ldots\ldots
\item $29=(4\times \dots)+\dots$. Autrement dit \ldots\ldots
\item $86=(9\times \dots)+\dots$. Autrement dit \ldots\ldots
\item $62=(8\times \dots)+\dots$. Autrement dit \ldots\ldots
\item $47=(6\times \dots)+\dots$. Autrement dit \ldots\ldots
\end{myenumerate}
\end{multicols}
%@Correction:
\begin{multicols}{2}
\begin{myenumerate}
\item \opidiv*{48}{9}{q}{r}\opidiv[style=text]{48}{9}. Autrement dit $48\div9=\opprint{q}$ (reste \opprint{r}).
\item \opidiv*{37}{5}{q}{r}\opidiv[style=text]{37}{5}. Autrement dit $37\div5=\opprint{q}$ (reste \opprint{r}).
\item \opidiv*{29}{4}{q}{r}\opidiv[style=text]{29}{4}. Autrement dit $29\div4=\opprint{q}$ (reste \opprint{r}).
\item \opidiv*{86}{9}{q}{r}\opidiv[style=text]{86}{9}. Autrement dit $86\div9=\opprint{q}$ (reste \opprint{r}).
\item \opidiv*{62}{8}{q}{r}\opidiv[style=text]{62}{8}. Autrement dit $62\div8=\opprint{q}$ (reste \opprint{r}).
\item \opidiv*{47}{6}{q}{r}\opidiv[style=text]{47}{6}. Autrement dit $47\div6=\opprint{q}$ (reste \opprint{r}).
\end{myenumerate}
\end{multicols}