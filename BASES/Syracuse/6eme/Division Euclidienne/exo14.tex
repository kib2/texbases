%@P:exocorcp
%@Dif:2
\begin{myenumerate}
\item Recopie et effectue les divisions euclidiennes suivantes :\\
\opidiv[resultstyle=\white,remainderstyle=\white,
intermediarystyle=\white]{63}{8}\hfill
\opidiv[resultstyle=\white,remainderstyle=\white,
intermediarystyle=\white]{125}{7}\hfill
\opidiv[resultstyle=\white,remainderstyle=\white,
intermediarystyle=\white]{245}{6}\hfill
\opidiv[resultstyle=\white,remainderstyle=\white,
intermediarystyle=\white]{351}{9}
\item Pour chacune d'elles, recopie et complète la phrase :
\par
\begin{cursive}
Dans la division euclidienne de\ldots\ldots par\ldots\ldots, le
dividende est\ldots\ldots, le diviseur est\ldots\ldots, le quotient
est\ldots\ldots et le reste est\ldots\ldots
\end{cursive}
\item Traduis chaque division par une égalité.
\end{myenumerate}
%@Correction:
\begin{myenumerate}
  \item\subitem{}\par
\opidiv{63}{8}\hfill\opidiv{125}{7}\hfill\opidiv{245}{6}\hfill\opidiv{351}{9}
\opidiv*{63}{8}{q1}{r1}\opidiv*{125}{7}{q2}{r2}\opidiv*{245}{6}{q3}{r3}\opidiv*{351}{9}{q4}{r4}
\item\subitem{}\par
\begin{cursive}
Dans la division euclidienne de 63 par 8, le dividende est 63, le diviseur est 8, le quotient est \opprint{q1} et le reste est \opprint{r1}.
\end{cursive}
\par
\begin{cursive}
Dans la division euclidienne de 125 par 7, le
dividende est 125, le diviseur est 7, le quotient
est \opprint{q2} et le reste est \opprint{r2}.
\end{cursive}
\par
\begin{cursive}
Dans la division euclidienne de 245 par 6, le
dividende est 245, le diviseur est 6, le quotient
est \opprint{q3} et le reste est \opprint{r3}.
\end{cursive}
\par
\begin{cursive}
Dans la division euclidienne de 351 par 9, le
dividende est 351, le diviseur est 9, le quotient
est \opprint{q4} et le reste est \opprint{r4}.
\end{cursive}
\item \opidiv[style=text]{63}{8}\hfill\opidiv[style=text]{125}{7}\hfill\opidiv[style=text]{245}{6}\hfill\opidiv[style=text]{351}{9}
\end{myenumerate}
%@Commentaire: Divisions euclidiennes simples. La description de la division sous deux formes (phrase et égalité) est importante pour l'acquisition du concept de la division euclidienne.