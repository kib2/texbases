%@P:exocorcp
%@Auteur: Régis Leclercq
%@Dif:2
\begin{myenumerate}
\item \begin{itemize}
\item[$\bullet$] \'Ecrire la liste des dix premiers multiples de 8.
\item[$\bullet$] Recopier et compléter: \og\ La différence entre deux multiples de 8 consécutifs est égale à \ldots\ldots \fg
\end{itemize}
\item \begin{itemize}
\item[$\bullet$] Voici des nombres: 32, 80, 1\,632, 176 et 112.\\Vérifier que ces nombres peuvent s'écrire comme $16\times\dots$
\item[$\bullet$] Que peut-on en déduire?
\end{itemize}
\item  \begin{itemize}
\item[$\bullet$] Citer les dix premiers multiples de 2 puis les dix premiers multiples de 3.
\item[$\bullet$] Citer parmi ces deux listes ceux qui sont à la fois multiples de 2 et de 3.
\end{itemize}
\end{myenumerate}
%@Correction:
\begin{myenumerate}
  \item 8; 16; 24; 32; 40; 48; 56; 64; 72; 80.\par \og\ La différence entre deux multiples de 8 consécutifs est égale à 8\fg.
  \item \opidiv[style=text]{32}{16}\kern1cm\opidiv[style=text]{80}{16}\kern1cm\opidiv[style=text]{1632}{16}\kern1cm\opidiv[style=text]{176}{16}\kern1cm\opidiv[style=text]{112}{16}\par Ce sont tous des multiples de 16.
  \item 2; 4; 6; 8; 10; 12; 14; 16; 18; 20.\par 3; 6; 9; 12; 15; 18; 21; 24; 27; 30.\par6; 12; 18; 24; 30 sont à la fois multiples de 2 et de 3.
\end{myenumerate}