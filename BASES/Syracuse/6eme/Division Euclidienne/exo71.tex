%@Titre:Déterminer le jour précis d'une date du \sc{XXI}\ieme\ siècle.
Pour cela, on applique la méthode suivante :
\columnseprule0.4pt
\begin{multicols}{2}
\begin{itemize}
\item on calcule $a$ qui est égal à la somme de 100 et du nombre
  constitué par les deux derniers chiffres de l'année;
\item on calcule $b$ qui est le quotient de la division euclidienne de
  $a$ par 4;
\item $c$ est le numéro du jour cherché;
\item $d$ est le numéro du mois, donné par le tableau suivant :\par
\vspace{2mm}
\begin{center}
  \begin{tabular}{|l|c|}
    \hline
    \multicolumn{1}{|c|}{Mois}&\no\\
    \hline
    Janvier; Octobre&0\\
    \hline
    Mai&1\\
    \hline
    Août&2\\
    \hline
    Février, Mars, Novembre&3\\
    \hline
    Juin&4\\
    \hline
    Septembre, Décembre&5\\
    \hline
    Avril, Juillet&6\\
    \hline
  \end{tabular}
\end{center}
\vspace{2mm}
\item on calcule $e=a+b+c+d$;
\item enfin on calcule $f$ qui est le reste de la division euclidienne de $e$ par 7.
\item Si $f=0$ alors c'est un dimanche; si $f=1$, c'est un lundi; si
  $f=2$, c'est un mardi;\ldots
\end{itemize}
\par\columnbreak\par
Par exemple, cherchons le jour du 14 Juillet 2050 :
\begin{itemize}
\item[$\star$] $a=100+50=150$;
\item[$\star$] \opidiv[style=text]{150}{4} donc $b=37$;
  \begin{center}
    \opidiv{150}{4}
  \end{center}
\item[$\star$] $c=14$;
\item[$\star$] $d=6$;
\item[$\star$] $e=150+37+14+6=207$.
\item[$\star$] Comme \opidiv[style=text]{207}{7} alors $f=4$.
  \begin{center}
    \opidiv{207}{7}
  \end{center}
\end{itemize}
Le 14 Juillet 2050 sera {\em un jeudi} !
\end{multicols}
\columnseprule0pt
\begin{myenumerate}
  \item Quel sera le jour du solstice d'été 2030, c'est-à-dire le 21
    Juin 2030 ?
  \item Quel sera le jour où tu auras 50 ans ?
\end{myenumerate}