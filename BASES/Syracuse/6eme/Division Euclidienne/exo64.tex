%@P:exocorcp
%@Auteur: Thierry Gauvin\par
\underline {Complète les phrases ci-dessous :}
\begin{myenumerate}
 \item Un nombre est divisible par 2 si \dotfill
 \item Un nombre est divisible par 5 si \dotfill
 \item Un nombre est divisible par 3 si \dotfill
 \item Un nombre est divisible par 9 si \dotfill
\end{myenumerate}
%@Correction:
\begin{myenumerate}
 \item Un nombre est divisible par 2 si le chiffre des unités est 0, 2, 4, 6 ou 8.
 \item Un nombre est divisible par 5 si le chiffre des unités est 0 ou 5.
 \item Un nombre est divisible par 3 si la somme des ses chiffres est dans la table de 3.
 \item Un nombre est divisible par 9 si la somme des ses chiffres est dans la table de 9.
\end{myenumerate}