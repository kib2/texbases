%@P:exocorcp
%@Dif:2
\begin{myenumerate}
  \item Quel est le plus grand reste possible lorsqu'on divise par 9 ? par 83 ? par 175 ?
  \item Dans une division euclidienne, le diviseur est 17, le reste 14. De combien faut-il diminuer ou augmenter le dividende pour obtenir un quotient exact ?
\end{myenumerate}
%@Correction:
\begin{myenumerate}
  \item Si on divise par 9 alors le plus grand reste possible est 8. Si c'est par 83 alors le plus grand reste est 82. Si c'est par 175 alors le plus grand reste est 174.
  \item Il faut augmenter le dividende de 3 ou le diminuer de 14.
\end{myenumerate}
%@Commentaire: On travaille sur le reste de la division euclidienne. Importance de la condition sur le reste.