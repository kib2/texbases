%@P:exocorcp
%@Dif:3
\begin{myenumerate}
\item Recopie et complète les divisions euclidiennes suivantes :
\\\opidiv[resultstyle=\white,remainderstyle=\white,
intermediarystyle=\white]{7965}{86}\hfill
\opidiv[resultstyle=\white,remainderstyle=\white,
intermediarystyle=\white]{7045}{103}\hfill
\opidiv[resultstyle=\white,remainderstyle=\white,
intermediarystyle=\white]{4000}{120}
\item Pour chacune d'elles, recopie et complète la phrase :
\par
\begin{cursive}
Dans la division euclidienne de\ldots\ldots par\ldots\ldots, le
dividende est\ldots\ldots, le diviseur est\ldots\ldots, le quotient
est\ldots\ldots et le reste est\ldots\ldots
\end{cursive}
\item Traduis chaque division par une égalité.
\end{myenumerate}
%@Correction:
\begin{myenumerate}
  \item\subitem{}\par
\opidiv{7965}{86}\hfill\opidiv{7045}{103}\hfill\opidiv{4000}{120}
\opidiv*{7965}{86}{q1}{r1}\opidiv*{7045}{103}{q2}{r2}\opidiv*{4000}{120}{q3}{r3}
\item \subitem{}\par
\begin{cursive}
Dans la division euclidienne de 7\,965 par 86, le
dividende est 7\,965, le diviseur est 86, le quotient
est \opprint{q1} et le reste est \opprint{r1}.
\end{cursive}
\par
\begin{cursive}
Dans la division euclidienne de 7\,045 par 103, le
dividende est 7\,045, le diviseur est 103, le quotient
est \opprint{q2} et le reste est \opprint{r2}.
\end{cursive}
\par
\begin{cursive}
Dans la division euclidienne de 4\,000 par 120, le
dividende est 4\,000, le diviseur est 120, le quotient
est \opprint{q3} et le reste est \opprint{r3}.
\end{cursive}
\item \subitem{}\par
\opidiv[style=text]{7965}{86}\hfill\opidiv[style=text]{7045}{103}\hfill\opidiv[style=text]{4000}{120}
\end{myenumerate}
%@Commentaire: La difficulté vient du nombres de chiffres du diviseur.