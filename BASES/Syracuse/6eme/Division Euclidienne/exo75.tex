%@Auteur:\url{www.geombre.com/article-7359751.html}\par
La maman de la petite Lisette lui demande d'aller à la boulangerie lui
acheter des baguettes.
\par Elle lui donne un billet de 5~\textgreek{\euro} en lui disant d'en acheter le plus possible et de garder la monnaie pour se payer des bonbons.
\begin{myenumerate}
  \item Quand elle connaîtra le prix d'une baguette, dans le calcul
    que fera Lisette pour savoir combien de baguette elle peut acheter
    avec ses 5~\textgreek{\euro}:
    \begin{enumerate}
    \item quel est le résultat qui intéresse sa mère ?
    \item quel est le résultat qui intéresse Lisette ?
    \end{enumerate}
  \item Près de chez Lisette, il y a trois boulangeries, l'une vend le
    pain à 1,30~\textgreek{\euro} et la baguette à
    0,85~\textgreek{\euro}, l'autre vend le pain à
    1,35~\textgreek{\euro} et la baguette à 0,80~\textgreek{\euro}, la
    troisième vend le pain à 1,30~\textgreek{\euro} et la baguette à 0,75~\textgreek{\euro}.
    \begin{enumerate}
    \item Où ira Lisette si elle veut surtout faire des économies ?
    \item Où ira Lisette si elle veut surtout avoir le plus monnaie
      pour ses bonbons ?
    \end{enumerate}
\end{myenumerate}

