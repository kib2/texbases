%@P:exocorcp
%@Auteur: IREM Strasbourg
%@Dif:2
\par Observe la division suivante qui est {\em exacte} :
\[\opidiv{94}{6}\opidiv*{94}{6}{q}{r}\]
Utilise les nombres intervenant dans cette division pour compléter le texte suivant :
\begin{quote}
  Un fermier a récolté \ldots \oe ufs. Il a entièrement rempli \ldots boîtes. Il prend une boîte supplémentaire et y met \ldots \oe ufs pour compléter la dernière boîte. Il lui faut \ldots boîtes au total.
\end{quote}
%@Correction:
\begin{quote}
  Un fermier a récolté 94 \oe ufs. Il a entièrement rempli \opprint{q} boîtes. Il prend une boîte supplémentaire et y met 2 \oe ufs pour compléter la dernière boîte. Il lui faut \opadd*{q}{1}{q}\opprint{q} boîtes au total.
\end{quote}
%@Commentaire: Travail sur le sens de la division : que représente le dividende, le quotient, \ldots ?