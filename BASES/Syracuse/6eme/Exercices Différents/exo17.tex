%@Titre: Quelques énigmes. 
%@Auteur: François Meria\par
\begin{myenumerate}
  \item Deux pères et deux fils ont chacun tué un canard. Aucun n'a tiré
le même canard. Trois  canards seulement ont été abattus.\par Explique
cela. 
\item Messieurs Leblanc, Lenoir et Leroux sont des enseignants.

Il y a cinq sujets à enseigner: anglais, français, maths, histoire
et géographie.\\

\textbf{Indices}

\begin{enumerate}
    \item Monsieur Lenoir ne sait pas ce qu'est un angle.
    \item Monsieur Leblanc est le seul à savoir où sont les montagnes Rocheuses.
    \item Chacun enseigne trois matières.
    \item Aucune matière n'est enseignée par 3 personnes.
    \item Certaines matières sont enseignées par 2 personnes.
    \item Monsieur Leroux est bilingue.
\end{enumerate}

Qui enseigne quelles matières ? 
\item Messieurs Lenoir, Lebrun et Levert mangeaient ensemble au
restaurant. Tous portaient des cravates de couleurs. L'un portait
une cravate noire, l'autre portait une cravate brune et le
dernier portait une cravate verte.

Soudainement l'homme portant la cravate verte s'écria: \og
Réalisez -vous que chacun porte une cravate de la même couleur que
nos noms mais personne ne porte une cravate semblable à son nom.
\fg

\og C'est curieux \fg, s'exclama monsieur Lenoir.

Quelle couleur de cravate portait chaque homme ? 
\end{myenumerate}
