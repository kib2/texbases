%@metapost:mesureanglefigure.mp
%@Auteur: Régis Leclercq
%@Dif:2
Dans chacun des cas, à l'aide de ton rapporteur, mesure les angles de chacun des quadrilatères.
\par
\includegraphics[scale=0.8]{mesureanglefigure.1}\hfill\includegraphics[scale=0.8]{mesureanglefigure.2}\hfill\includegraphics[scale=0.8]{mesureanglefigure.3}
\par
\includegraphics[scale=0.8]{mesureanglefigure.4}\hfill\includegraphics[scale=0.8]{mesureanglefigure.5}\hfill\includegraphics[scale=0.8]{mesureanglefigure.6}
%@Commentaire: On abandonne les classiques triangles pour passer aux quadrilatères. Il y a plus d'angles à mesurer et la forme des quadrilatères impose une réflexion sur le maniement du rapporteur.