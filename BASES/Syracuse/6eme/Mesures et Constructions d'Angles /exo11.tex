%@metapost:mesureanglefigure.mp
%@Auteur: Régis Leclercq
%@Dif:3
Dans chacun des cas, à l'aide de ton rapporteur, mesure les angles formés par les diagonales de chacun des quadrilatères et code ensuite les angles qui te paraissent égaux.
\par
\includegraphics[scale=0.8]{mesureanglefigure.7}\hfill\includegraphics[scale=0.8]{mesureanglefigure.8}\hfill\includegraphics[scale=0.8]{mesureanglefigure.9}
\par
\includegraphics[scale=0.8]{mesureanglefigure.10}\hfill\includegraphics[scale=0.8]{mesureanglefigure.11}\hfill\includegraphics[scale=0.8]{mesureanglefigure.12}
%@Commentaire: On abandonne les classiques triangles pour passer aux quadrilatères. Il y a plus d'angles à mesurer et la forme des quadrilatères impose une réflexion sur le maniement du rapporteur.