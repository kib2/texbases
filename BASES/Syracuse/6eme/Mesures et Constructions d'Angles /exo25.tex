%@Auteur: François Meria
%@Dif:2
\dispo{1}{% Generated by eukleides 1.0.0
\psset{linecolor=black, linewidth=.5pt, arrowsize=2pt 4}
\psset{unit=1.0000cm}
\pspicture(-1.0000,0.0000)(7.0000,5.50000)
\pspolygon(0.0000,0.0000)(6.0000,0.0000)(0.0000,4.5000)
\pspolygon(0.0000,0.0000)(6.0000,0.0000)(3.0000,5.1962)
\psline(0.0000,0.0000)(4.5000,2.5981)
\psline(3.0000,5.1962)(3.6260,0.0000)
\uput{0.3000}[180.0000](0.0000,0.0000){$A$}
\uput{0.3000}[0.0000](6.0000,0.0000){$B$}
\uput{0.3000}[90.0000](0.0000,4.5000){$C$}
\uput{0.3000}[90.0000](3.0000,5.1962){$D$}
\uput{0.3000}[180.0000](1.8130,3.1402){$E$}
\uput{0.3000}[0.0000](4.5000,2.5981){$M$}
\uput{0.3000}[180.0000](3.3902,1.9573){$F$}
\uput{0.3000}[-90.0000](3.6260,0.0000){$G$}
\psarc(0.0000,0.0000){0.45000}{0.0000}{30.0000}
\psarc(0.0000,0.0000){0.55000}{0.0000}{30.0000}
\psarc(0.0000,0.0000){0.65000}{0.0000}{30.0000}
\psarc(3.0000,5.1962){0.4500}{-120.0000}{-83.1301}
\psarc(3.0000,5.1962){0.5500}{-120.0000}{-83.1301}
\psarc(3.3902,1.9573){0.5000}{-83.1301}{30.0000}
\endpspicture
% End of figure
}{Dans la figure ci-contre
\begin{myenumerate}
\item Nommer de deux façons possibles les angles marqués :
\begin{itemize}
 \item \dotfill
 \item \dotfill
 \item \dotfill
\end{itemize}
\item Marquer en bleu l'angle $\widehat{ACB}$, en rouge l'angle
$\widehat{EFM}$, en vert l'angle $\widehat{CDF}$.
\end{myenumerate}
}
