%@Titre:Pondichéry -- 2007
\emph{Les parties A et B sont indépendantes. La feuille annexe est à rendre avec votre copie}
\paragraph{Partie A}\hfill\newline
DVDLOC est un magasin qui propose différentes formules de location de DVD.
\begin{itemize}
\item Formule 1 : chaque DVD est loué 3,50~\textgreek{\euro}.
\item Formule 2 : on paye un abonnement annuel de 12~\textgreek{\euro}, puis 2~\textgreek{\euro} par DVD loué.
\end{itemize}
\begin{myenumerate}
\item  Compléter sur la feuille ANNEXE le tableau suivant :
\begin{center}\begin{tabularx}{0.8\linewidth}{|l|*{2}{>{\centering \arraybackslash}X|}}\hline
	Nombre de DVD loués&	2&	6\\ \hline
	Prix en euro avec la formule 1&&\\ \hline
	Prix en euro avec la formule 2 &&\\ \hline
	\end{tabularx}
	\end{center}
\item On note $x$ le nombre de DVD loués.
\begin{enumerate}
\item Exprimer, en fonction de $x$, le prix en euro à payer pour la location de $x$ DVD par la formule 1. 
\item Exprimer, en fonction de $x$, le prix en euro à payer pour la location de $x$ DVD par la formule 2.
\end{enumerate}
\item  \begin{enumerate}
\item Résoudre l'inéquation $2x + 12 \leqslant  3,5x$.
\item Déterminer le nombre de DVD à partir duquel la formule 2 est la plus avantageuse.
\end{enumerate}
\item Sur la feuille ANNEXE, tracer dans le repère les représentations graphiques des fonctions $f$ et $g$ définies par : $f(x) = 3,5x$ et $g(x) = 2x + 12$.
\item Carine ne possède pas de carte d'abonnement et elle dispose de 18~\textgreek{\euro}. Indiquer à l'aide du
graphique et en marquant en couleur les pointillés nécessaires, le nombre maximum de DVD qu'elle peut louer.
\end{myenumerate}
\paragraph{Partie B}\hfill\newline
\begin{myenumerate}
\item Romain se rend à vélo chez son ami David qui a loué un DVD chez DVDLOC.\\
Sachant qu'il a 3,75 kilomètres à parcourir et qu'il roule à la vitesse moyenne de 15 km/h, quel temps mettra-t-il pour faire ce trajet ?
\item Après avoir regardé le film, Romain propose à David d'aller rendre ce DVD au magasin de location. Sachant qu'il roule pendant 36 minutes, toujours à la vitesse moyenne de 15 km/h, déterminer la distance qui sépare le magasin du domicile de David.
\end{myenumerate}
%@metapost:pondichery2007.mp
%@Titre:Pondichéry -- 2007
\[\includegraphics{pondichery2007.2}\]