%@Titre:Polynésie -- 2006 (Sept.)
Le plan est muni d'un repère orthononné $(O;  I, J)$. L'unité choisie est le centimètre.
\begin{myenumerate}
\item En utilisant la feuille de papier millimétré jointe, placer les points
$A(3 ; 4)$, $B(- 1~ ;~- 4)$ et $C(-7 ~;~-1)$.
\item \begin{enumerate}
\item Montre que $AB= \sqrt{80}$, $AC= \sqrt{125}$ et $BC = \sqrt{45}$.
\item En déduire que $ABC$ est un triangle rectangle. Préciser l'angle droit.
\end{enumerate}
\item \begin{enumerate}
\item Construire le point $D$ tel que $\vecteur{CD} = \vecteur{BA}$.
\item Donner les coordonnées du point $D$ par lecture graphique.
\item Démontrer que $ABCD$ est un rectangle.
\item Calculer les coordonnées de $\vecteur{BA}$.
\end{enumerate}
\item \begin{enumerate}
\item Calculer les coordonnées du point $K$ milieu du segment $[AC]$.
\item Que représente le point $K$ pour le quadrilatère $ABCD$ ?
 \end{enumerate}
\item \begin{enumerate}
\item Construire le cercle $(\mathscr{C})$ circonscrit au triangle $ABC$ en précisant le centre et le rayon.
\item Montrer que le point $D$ est sur le cercle $(\mathscr{C})$.
\end{enumerate}
\end{myenumerate}