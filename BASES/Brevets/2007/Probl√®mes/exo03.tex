%@Titre:Amérique du Nord -- 2007
\textbf{Les parties A et B sont indépendantes.\\ La feuille ANNEXE  est à rendre avec la copie.}
 \paragraph{Partie A}\hfill\newline
Deux établissements scolaires ont financé des déplacements en car pour se rendre dans un musée, où une grande exposition de peinture se tient durant plusieurs mois.
\begin{myenumerate}
\item  L'établissement du premier groupe est situé à 250~km du musée. Le car a quitté le collège à 7~h~25 et roule à la vitesse moyenne de 100~km/h. Calculer l'heure d'arrivée au musée de ce premier groupe.
\item  Le second groupe a quitté son établissement à 8~h~00 pour arriver à 9~h~30. II a parcouru 120~km pour se rendre au musée. Calculer la vitesse moyenne, en km/h, du car transportant ce second groupe.
\end{myenumerate}
\paragraph{Partie B}\hfill\newline
Armelle souhaite travailler quelques heures par mois dans ce musée, afin de gagner un peu d'argent. \`A la suite d'un entretien, deux possibilités d'indemnisation lui sont proposées :
 \begin{itemize}
\item  Somme d'argent $S_{1}$ :  8~\textgreek{\euro} par heure. 
\item  Somme d'argent $S_{2}$ : versement de 90~\textgreek{\euro} en début de mois, puis 5~\textgreek{\euro} par heure.
\end{itemize}
 Ne sachant pas quelle forme d'indemnisation privilégier, elle décide d'étudier ces deux propositions.
\begin{myenumerate}
\item compléter le tableau :\\
\begin{tabularx}{\linewidth}{|c|p{1.5cm}|*{2}{>{\centering \arraybackslash}X|}}\cline{3-4}
\multicolumn{2}{c|}{}&\multicolumn{2}{c|}{Nombre d'heures effectuées par mois}\\  \cline{3-4}
\multicolumn{2}{c|}{}&20 heures  & 25 heures\\ \hline
{\small Somme d'argent per\c{c}ue}&~~~~~~~~$S_{1}$&&\\ \hline
par mois en \textgreek{\euro})&~~~~~~~~$S_{2}$&&\\ \hline
\end{tabularx}
\item Soit $x$ le nombre d'heures effectuées par Armelle pendant un mois dans ce musée. Exprimer en fonction de $x$ les sommes d'argent $S_{1}(x)$ et $S_{2}(x)$, versées à Armelle selon les deux formes d'indemnisation proposées.
\item Résoudre l'équation $8x = 5x + 90$. \`A quoi correspond la solution de cette équation ?
\item Sur le repère ci-dessous, représenter graphiquement les deux fonctions suivantes :
\[S_1~: x \longmapsto 8x~~	\text{et}~~	S_{2}~: x \longmapsto 5x+90\]
\item \begin{enumerate}
\item Utiliser une couleur pour marquer les traits qui permettent de déterminer graphiquement le résultat de la question 3. 
\item Utiliser une autre couleur pour marquer les traits qui permettent de déterminer graphiquement l'indemnisation la plus avantageuse pour Armelle si elle souhaite effectuer 35 heures par mois. Indiquer alors la somme d'argent perçue.
\end{enumerate}
\item En s'aidant du graphique, indiquer à Armelle l'indemnisation la plus avantageuse en fonction du nombre d'heures effectuées par mois dans ce musée.
\end{myenumerate}
%metapost:amnord2007.mp
\[\includegraphics{amnord2007.3}\]