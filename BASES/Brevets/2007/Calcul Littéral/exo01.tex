%@Titre: Métropole -- 2007
On donne un programme de calcul :
  \begin{center}
    \psframebox[fillstyle=solid,fillcolor=gray]{\begin{minipage}{0.75\linewidth}
        \begin{itemize}
        \item[\textbullet] Choisir un nombre.
        \item[\textbullet] Lui ajouter 4.
        \item[\textbullet] Multiplier la somme obtenue par le nombre choisi.
        \item[\textbullet] Ajouter 4 à ce produit.
        \item[\textbullet]\'Ecrire le résultat.
        \end{itemize}
\end{minipage}
}
  \end{center}
  \begin{myenumerate}
    \item\'Ecrire les calculs permettant de vérifier que si l'on fait fonctionner ce programme avec le nombre $-2$, on obtient 0.
    \item Donner le résultat fourni par le programme lorsque le nombre choisi est 5.
    \item
      \begin{enumerate}
      \item Faire deux autres essais en choisissant à chaque fois un nombre entier et écrire le résultat obtenu sous la forme d'un carré d'un autre nombre entier (les essais doivent figurer sur la copie).
      \item En est-il toujours ainsi lorsque l'on choisit un nombre entier au départ de ce programme de calcul ? Justifier la réponse.
      \end{enumerate}
    \item On souhaite obtenir 1 comme résultat. Quels nombres peut-on choisir au départ ?
  \end{myenumerate}
