%@metapost:brevet2007.mp
%@Titre: Métropole -- 2007
\compo{1}{brevet2007}{1}{Sur la figure ci-contre,
    \begin{itemize}
    \item $ABC$ est un triangle équilatéral;
    \item le point $O$ est le centre du cercle circonscrit au triangle $ABC$;
    \item le point $D$ est le point diamètralement opposé au point $B$ sur ce cercle.
    \end{itemize}
}
\begin{myenumerate}
  \item Quelle est la nature du triangle $ABD$ ? Justifier.
  \item Quelle est la mesure de l'angle $\widehat{ADB}$ ? Justifier.
  \item On désigne par $E$ l'image du point $D$ par la translation de vecteur $\vecteur{OC}$.\\Démontrer que les droites $(DC)$ et $(OE)$ sont perpendiculaires.
\end{myenumerate}
