%@Titre:Polynésie -- 2006 (Sept.)
Voici les notes de $200$ élèves regroupées dans le tableau reproduit ci-dessous.
\begin{myenumerate}
\item Montrer que le nombre d'élèves $x$ ayant obtenu une note comprise entre $12$ et $16$ ($16$ exclu) est égal à $64$.\\
\noindent \begin{tabularx}{\linewidth}{|p{1,3cm}|*{5}{>{\centering \arraybackslash}X|}}\hline
Notes $n$& {\footnotesize $	0 \leqslant  n < 4$}&{\footnotesize $4 \leqslant n < 8$}	&{\footnotesize $8 \leqslant  n < 12$}&{\footnotesize 	$12 \leqslant  n <16$}&{\footnotesize 	$16 \leqslant  n \leqslant  20$}\\ \hline
Nombre	d'élèves	&8	&48	&	56	&$x$	&	24\\ \hline
\end{tabularx}
\item Combien d'élèves ont obtenu une note strictement inférieure à 8 ?
\item Combien d'élèves ont obtenu au moins 12 ?
\item Calculer le pourcentage des élèves qui ont obtenu une note comprise entre 8 et 12 (12 exclu).
\end{myenumerate}