%@Titre:Pondichéry -- 2007
Voici les résultats au lancer de javelot lors d'un championnat d'athlétisme. Les longueurs sont exprimées en mètres.
 
\begin{center}
\psframebox{\begin{minipage}{0.7\linewidth}
\begin{center}
36 42 37 43 38 44 32 40	44 36 46 39 40\\
40 41 41 45 37 43 43 46 39 44 47 48\\
\end{center}
\end{minipage}
}
\end{center}
\begin{myenumerate}
\item Compléter le tableau suivant \emph{sur la feuille annexe}\\
\begin{tabularx}{\linewidth}{|p{2,5cm}|*{4}{>{\centering \arraybackslash}X|}c|}\hline
Longueur $\ell$ du	lancer (en mètres)&\footnotesize $30 \leqslant\ell <  35$&\footnotesize	$35 \leqslant \ell <40$&\footnotesize	$40  \leqslant\ell < 45$&\footnotesize	$45 \leqslant\ell < 50$&	Total\\ \hline
Nombre de sportifs	&		&7		&		&5		&\\  \hline
Fréquence			&0,04	&		&		&0,2	& \\ \hline
Valeur centrale		&32,5	&		&42,5	&		&\multicolumn{1}{|c}{} \\ \cline{1-5}
\end{tabularx}
\item En utilisant les valeurs centrales, calculer la longueur moyenne d'un lancer.
\item Quel est le pourcentage de sportifs ayant lancé au moins à 40 mètres ?
\end{myenumerate}