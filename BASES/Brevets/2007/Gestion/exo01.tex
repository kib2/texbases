%@Titre:Guadeloupe -- 2007
Le tableau ci-dessous (source: site national de la sécurité routière)
donne la répartition, par tranche d'âges, du nombre des victimes dans
des accidents dus à l'alcool, en 2005:
\begin{center}
  \begin{tabular}{|c|c|c|c|c|c|c|}
    \hline
    Tranches d'âges&0 ­ 17 ans&18 ­ 24 ans&25 ­ 44 ans&45 ­ 64 ans&65
    ans et plus&Age inconnu\\
    \hline
    Nombre de tués&68&384&557&&68&8\\
    \hline
  \end{tabular}
\end{center}
\begin{myenumerate}
  \item On sait de plus que le nombre total de tués dans des accidents
    dus à l'alcool en 2005 est de 1\,355.\\Compléter le tableau.
  \item Quelle est la tranche d'âges la plus touchée ?
  \item Parmi les victimes d'accidents dus à l'alcool, calculer le
    pourcentage de tués de moins de 25 ans ?\\Donner l'arrondi à l'unité.
  \item En 2005 , il y a eu 4\,718 tués dans des accidents de la
    circulation. Quel est le pourcentage des tués dans des accidents
    dus à l'alcool? On donnera l'arrondi à l'unité.
\end{myenumerate}