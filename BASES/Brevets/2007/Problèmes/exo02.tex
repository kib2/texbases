%@metapost:brevet2007.mp
%@Titre: Métropole -- 2007
\par\compo{2}{brevet2007}{1}{Dans le jardin de sa nouvelle maison, M. Durand a construit une terrasse rectangulaire qu'il désire recouvrir d'un toit.\\Pour cela, il réalise le croquis suivant où l'unité de longueur est le mètre.
  \begin{itemize}
  \item Le sol $ABCD$ et le toit $EFGH$ sont des rectangles.
  \item Le triangle $HIE$ est rectangle en $I$.
  \item Le quadrilatère $HEAB$ est un rectangle.
  \item La hauteur du sol au sommet du toit est $HB$.
  \end{itemize}
On donne $AB=2,25$; $AD=7,5$; $HB=5$.
}
\paragraph{Partie I}\hfill\newline
\compo{3}{brevet2007}{1}{On suppose dans cette partie que $AE=2$.
  \begin{myenumerate}
    \item Justifier que $HI=3$.
    \item Démontrer que $HE=3,75$.
    \item Calculer au degré près la mesure de l'angle $\widehat{IHE}$ du toit avec la maison.
  \end{myenumerate}
}
\paragraph{Partie II}\hfill\newline
\compo{4}{brevet2007}{1}{Dans cette partie, on suppose que $\widehat{IHE}=45\degres$\ et on désire déterminer $AE$.
  \begin{myenumerate}
    \item Quelle est la nature du triangle $HIE$ dans ce cas ? Justifier.
    \item En déduire $HI$ puis $AE$.
  \end{myenumerate}
}
\paragraph{Partie III}\hfill\newline
\compo{5}{brevet2007}{1}{Dans cette partie, on suppose que $\widehat{IHE}=60\degres$\ et on désire déterminer $AE$.
  \begin{myenumerate}
    \item Déterminer la valeur arrondie au cm de $HI$.
    \item En déduire la valeur arrondie au cm de $AE$.
  \end{myenumerate}
}

\paragraph{Partie IV}\hfill\newline
La courbe ci-dessous représente la hauteur $AE$ en fonction de la mesure de l'angle $\widehat{IHE}$.
\[\includegraphics{brevet2007.6}\]
M. Durand souhaite que la hauteur $AE$ soit comprise entre 3~m et 3,5~m. En utilisant le graphique, donner une mesure possible de l'angle $\widehat{IHE}$.