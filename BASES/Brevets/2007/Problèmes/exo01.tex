%@Titre:Guadeloupe -- 2007
On transfert le pétrole contenu dans un réservoir B vers un réservoir A à l'aide d'une pompe.\\
Après démarrage de la pompe, on constate que la hauteur de pétrole
dans le réservoir A augmente de 3~cm par minute. Le réservoir A est
vide au départ.
\paragraph{I. Remplissage du réservoir A.}\hfill\newline
\begin{myenumerate}
  \item Recopier et compléter le tableau suivant:
    \begin{center}
      \begin{tabular}{|l|c|c|c|c|c|}
        \hline
        Temps (en min)&0&10&20&30&40\\
        \hline
        Hauteur du pétrole dans le réservoir A (en cm)&0&&60&&\\
        \hline
      \end{tabular}
    \end{center}
  \item On appelle $x$ le temps (en minute) de fonctionnement de la
    pompe et $f(x)$ la hauteur du pétrole (en cm) dans le réservoir
    A.\\Parmi les trois fonctions suivantes, laquelle correspond à la
    fontion $f$ :
    \[x\mapsto-2x\kern0.15\linewidth x\mapsto3x+20\kern0.15\linewidth x\mapsto3x\]
  \item Représenter graphiquement la fonction $f$ ,pour $x$ variant de
    0 à 40, sur la feuille de papier millimétrée jointe.
    \\Les unités:
    \begin{itemize}
    \item[en abscisse:] 2~cm représenteront 5 minutes;
    \item[en ordonnée:] 1~cm représentera une hauteur de 10~cm de
      pétrole dans la cuve.
    \end{itemize}
  \item Déterminer graphiquement le temps nécessaire pour obtenir une
    hauteur de pétrole de 105~cm dans le réservoir A. On fera
    apparaître les tracés sur le graphique.
\end{myenumerate}
\paragraph{II. Vidange du réservoir B .}\hfill\newline
Sur la feuille de papier millimétrée, le segment $[CD]$ représente la
hauteur (en centimètre) de pétrole dans la cuve B en fonction du temps
(en minute).
\\Le segment $[CD]$ est représenté sur la feuille de papier
millimétrée distribuée aux élèves. Le point $C$ a pour coordonnées
$(0;200)$ et le point $D(40;0)$.
\\Les unités sont les mêmes que dans la première partie :
\begin{itemize}
  \item[en abscisse:] 2~cm représenteront 5 minutes;
  \item[en ordonnée:] 1~cm représentera une hauteur de 10~cm de
    pétrole dans la cuve.
\end{itemize}
\begin{myenumerate}
  \item Recopier le tableau ci-dessous.\\Le compéter en utilisant le graphique de la feuille millimétrée.
    \begin{center}
      \begin{tabular}{|l|c|c|c|c|}
        \hline
        Temps (en min)&0&10&&40\\
        \hline
        Hauteur du pétrole dans le réservoir B (en cm)&200&&80&\\
        \hline
      \end{tabular}
    \end{center}
  \item On appelle $x$ le temps (en minute) de fonctionnement de la
    pompe et $g(x)$ la hauteur du pétrole (en cm) dans le réservoir B.
    \\Parmi les trois fonctions suivantes, quelle est celle qui
    correspond à la fontion $g$ :
    \[x\mapsto-4x\kern0.15\linewidth x\mapsto3x+200\kern0.15\linewidth x\mapsto-5x+200\]
  \item Déterminer par le calcul le temps au bout duquel les hauteurs
    de pétrole dans les cuves A et B sont égales.
  \item Expliquer comment on peut retrouver graphiquement ce dernier résultat.
\end{myenumerate}