%@Titre: Métropole -- 2007
*Cet exercice est un questionnaire à choix multiples (QCM).\\Aucune justification n'est demandée.
\\Pour chacune des questions, trois réponses sont proposées, une seule est exacte.
\par\textbf{\em Pour chacune des cinq questions, indiquer sur la copie le numéro de la question et recopier la réponse exacte.}
\par
\begin{center}
\QCMvar{3}{2}{}{
\hline
\multicolumn{1}{|c|}{}&\multicolumn{1}{c|}{Réponse A}&\multicolumn{1}{c|}{Réponse B}&\multicolumn{1}{c|}{Réponse C}\\
\hline
Quelle est l'expression développée de $(3x+5)^2$ ?&$3x^2+25$&$9x^2+25$&$9x^2+30x+25$\\
Quelle est l'expression qui est égale à 10 si on choisit la valeur $x=4$ ?&$x(x+1)$&$(x+1)(x-2)$&$(x+1)^2$\\
Quelle est la valeur exacte de $\dfrac{\sqrt{48}}2$ ?&$\sqrt{24}$&$3,464$&$2\sqrt3$\\
Quel est le nombre qui est solution de l'équation $2x-(8+3x)=2$ ?&10&$-10$&2\\
En 3\ieme\ A, sur 30 élèves, il y 40\% de filles. En 3\ieme\ B, il y a 60\% de filles.\par Lorsque les deux classes sont réunies, quel est le pourcentage de filles dans le groupe ?&36\% de filles.&48\% de filles.&50\% de filles.\\
}
\end{center}