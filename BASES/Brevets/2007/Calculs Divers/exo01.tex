%@Titre:Guadeloupe -- 2007
\QCMvar{3}{2.5}{Pour chaque question, il n'y a qu'une bonne
réponse. On entourera la bonne réponse.}{%
\multicolumn{4}{|c|}{Barême: 1 point par bonne réponse, 0
autrement.}\\
\hline
\multicolumn{1}{|c|}{}&\multicolumn{1}{c|}{Réponse A}&\multicolumn{1}{c|}{Réponse B}&\multicolumn{1}{c|}{Réponse C}\\
\hline
Une solution de $3x^2-5x+2=0$ est
:&$-1$&$\dfrac{\strut2}3$&$\dfrac73$\\
Les solutions de  $(x-\dfrac12)(x+2)=0$ sont :&$-2$ et
$-\dfrac12$&$-2$ et $\dfrac12$&$2$ et $-\dfrac{\strut1}2$\\
Les solutions de $2x+1<4x-2$ sont :&$x<-\dfrac{\strut1}2$&$x>\dfrac32$&$x<-\dfrac32$\\
Le développement de $(x-1)(x+3)-(x-\dfrac12)(x+1)$ est
:&$x^2-3x+9$&$x^2+\dfrac{\strut3}2x+\dfrac52$&$\dfrac32x-\dfrac52$\\
La factorisation de $25x^2-16$ est
:&$(5x-4)^2$&$(5x-4)(5x+4)$&$(5x+4)^2$\\
La fraction irréductible égale à
$\dfrac{3-\dfrac5{\strut2}}{\dfrac{\strut2}{\strut7}-\dfrac72}$ est
:&1&$-\dfrac{45}{28}$&$\dfrac{-7}{45}$\\
L'écriture sous forme scientifique de
$\dfrac{49\times10^{-6}\times6\times10^5}{3\times10^4\times7\times10^{-2}}$
est :&$1,4\times10^{-2}$&$1,4\times10^{-1}$&$1,4\times10^2$\\
L'écriture sous la forme $a\sqrt5$ de
$\sqrt{180}-\sqrt{45}+3\sqrt{20}$ est :&$9\sqrt5$&$-3\sqrt5$&$3\sqrt5$\\
}